\documentclass{book}

\usepackage[portuguese]{babel}
\usepackage[letterpaper, top=2.5cm, bottom=2cm,left=2cm,right=2cm]{geometry}
\usepackage{hyperref}
\usepackage{outlines}
\setlength\parindent{0pt}

\hypersetup{
    colorlinks=true,
    linkcolor=blue,
    filecolor=magenta,      
    urlcolor=cyan,
    pdftitle={Overleaf Example},
    pdfpagemode=FullScreen,
    }

\urlstyle{same}

\title{Notas Forcera}
\author{Francisco Valente}

\begin{document}

\section*{Notas Forcera}

\hrule
\vspace{0.5cm}
\section*{Lista de sites :}
\hrule
\vspace{0.5cm}

\begin{itemize}
		\item \href{https://www.open-contracting.org/}{OCDS} : Open Contracting Data Standard 
		\item \href{https://www.open-contracting.org/wp-content/uploads/2016/11/OCP2016-Red-flags-for-integrityshared-1.pdf}{Manual sobre red flags}		
		\item \href{https://redflags.ai/}{Procurement Red Flags} 		
		\item \href{https://www.redflags.eu/}{Red Flags in Hungary}  		
		\item \href{https://k-monitor.hu/home}{K Monitor}             		
		\item \href{https://www.transparency.org/}{Transparency International}	
		\item \href{https://transparencia.pt/}{Transparência Internacional Portugal}
		\item \href{https://www.base.gov.pt/base4}{Base de Dados}				
\end{itemize}

\vspace{1cm}
\hrule
\subsection*{\href{https://www.open-contracting.org/}{Manual} : Red Flags for integrity}
\hrule
\vspace{1cm}

\subsection*{Parte I}

  
Existem 4 \textit{use cases} principais para \textit{open contracting} :

\begin{itemize}
  \item  Promover integridade
  \item  Greater value for money ( usar de forma mais eficiente o dinheiro ? )
  \item  Melhorar a competição e tornar ( os concursos  ? ) mais justos 
  \item  Melhoria do serviço contratado
\end{itemize}


Algumas definições : 

\begin{itemize}

		\item  \textit{Use cases :}
		
				\subitem descrições de como é que um usuário interage/lida com um software ou um produto. Estes \textit{use cases} são construídos de forma a establecer diferentes cenários 
				
				\subitem um possível cenário em que um sistema recebe um input de um usuário, por exemplo, e o sistema responde/reage a esse input 
				
		\item  \textit{Procurement : } ( aquisição ; obtenção )
		
				\subitem the process of getting supplies 
				
		\item   \textit{Government Procurement :} 
		
			\subitem consiste na aquisição de bens e serviços por parte de um governo. Estes bens e serviços são fornecidos por empresas e fornecedores. Existe um concurso público, as empresas candidatam-se e, consoante os critétios de elegibilidade, uma das empresas é, normalmente, selecionada. Estas transações são efetuadas utilizando dinheiro dos contribuintes.
		
		\item  \textit{Fraude} 
		\item  \textit{Corrupção}

\end{itemize}



\textbf{Problema :}  O maior risco de corrupção de um governo occore nos casos de contratação públice e \textit{procurement} \\


  
\textbf{Objetivo do Open Contracting :} 
\begin{itemize}
	
	\item detetar eventuais casos de fraude e corrupção antes da realização de um contrato
	\item detetar situações anómalas no processo de candidatura e seleção das empresas
	\item ao tornar o processo público, permite que um maior número de intervenientes possa verificar a integridade do contrato
	\item identificar situações suspeitas permite  identificar falhas no processo de *procurement* e sugerir mudanças/melhorias ao mesmo
	
\end{itemize}

  
  
\textbf{Como :} Dado que a natureza de todos os contratos realizados entre um governo e uma empresa não é a mesma, existe um espectro grande relativamente à definição de \textit{fraude} e \textit{corrupção}. Para tal, existem cerca de 150 indicadores de situações suspeitas/anómalas - denominadas, também, por \textbf{RED FLAGS} - que \textbf{podem} sugerir a presença de fraude e corrupção.\\
 
Estas situações podem ocorrer ao longo de todo o processo de \textit{public procurement}. \\
  
\textbf{Etapas do processo de *public procurement* :}
  
  \begin{enumerate}
		\item Planning ( Planeamento ) 
		\item Tender   ( Proposta ) 
		\item Award    ( Concessão )
		\item Contract
		\item Implementation
  \end{enumerate}

  
\textbf{O que É uma red flag :}
\begin{itemize}
	\item Situação anómala/suspeita que requer atenção e, eventualmente, intervenção
	\item Pode indicar o risco de fraude ou corrupção num contrato
	\item Pode alertar para padrões de erros cometidos  
	\item Realça/Destaca uma situação que deve ser melhor analisada e que cai numa de 3 categorias
		\subitem situação não ilícita nem indesejável
		\subitem situação não ilícta mas indesejável em termos de rentabilização de dinheiro, competitividade e qualidade de serviço
		\subitem situação ilícita
	\item Ajudam países a detetar *data gaps* e a prevenir situações dessas no futuro de modo a não comprometer a qualidade e disponibilidade dos dados	

\end{itemize}


\textbf{O que NÃO É uma red flag :}
\begin{itemize}
	\item Não prova que existe fraude ou corrupção numa transação / celebração de um contrato
	\item Não prova que existe comportamento ilícito 
\end{itemize}
  
  
  
  
  
\subsection*{Parte II}
  
Existem 3 pontos focais no que toca a \textit{flagging / sinalização} :

\begin{itemize}
	\item Contextualização
	\item Triangulação ( existem 3 entidades envolvidas )
	\item Qualidade dos dados e \textit{linking information}
\end{itemize}



\subsubsection*{Contextualização}

As maiores ameaças à integridade a nível global : \url{https://iacrc.org/training-courses/} \url{https://guide.iacrc.org/}


De uma forma geral, atos / comportamentos ilícitos pertencem a uma das seguintes \href{https://guide.iacrc.org/proof-of-common-schemes/}{categorias}: 

\begin{itemize}
	\item corrupção
	\item fraude em licitações
	\item conluio ( aliança ; coligalação ; combinação de dois ou mais para prejudicar outrem )
	\item fraude	
\end{itemize}
  
\textbf{Problema :} 

\begin{itemize}
	\item Construir indicadores de situações suspeitas - Red Flags - a nível global dada a situação económica, geográfica, social e cultural de diferentes países
	\item Evitar gerar red flags falsas
\end{itemize}


\textbf{Possível solução : }

\begin{itemize}
	\item Ferramentes de contextualização 
	\item Adaptar as flags ao problema - ter um conhecimento detalhado acerca da natureza e do contexto do trabalho em questão
	\item Tentar descobrir/entender o que é normal numa dada situação para detetar anormalidades
\end{itemize}


Exemplos : 

\begin{outline}[enumerate]
	
	\1  \textit{Período de proposta da candidatura muito \textbf{curto} : }
		\2  o que é um período de tempo curto ? 
		\2  como definir um perído de tempo aceitável - treshold - face à natureza do contrato ?
	
	\1  \textit{Número de licitações pequeno/grande :}
		\2 Para contratos que requerem mão de obra altamente especializada - aeroespacial por exemplo - espera-se um número pequeno de licitações. Pequeno pode ser 2 a 3 licitações.
		\2 Para contratos mais \textit{gerais}, tais como webdesign por exemplo, espera-se um número superior de licitações visto que existe um maior número de empresas a prestar este tipo de serviços. Se o número de licitações ronda as 2 ou 3, levanta suspeitas e requer atenção. Normalmente, espera-se que seja superior a um dado número - que é necessário definir. 
	
\end{outline}


\subsubsection*{Triangulação}
 
\textbf{Problema :} Evitar gerar red flags falsas


Possível Solução :

\begin{outline}[enumerate]
	
	 \1 Utilizar \href{https://digiwhist.eu/wp-content/uploads/2016/08/GTI_WP2016_3_Fazekas-Cingolani-Toth_Conceptualising-PP-corr_160821.pdf}{triangulação} 
	 	\2 agrupar múltiplos indicadores/proxies de corrupção
	 	\2 ao agrupar dados sobre a empresa envolvida e sobre o processo de *procurement* - se for possível - podemos definir um valor melhor de treshold para o problema em questão
	
\end{outline}


Exemplo : Gerar um flag agrupando 

\begin{itemize}
	\item a duração do período de licitação
	\item número baixo de licitadores 
	\item um elevado ou muito baixo número de especificações técnicas
\end{itemize}


Obstáculos : 

\begin{itemize}
	\item Indicadores triangulados são definidos localmente. Se definidos globalmente, corre-se o risco de gerar um elevado número de red flags falsas
	\item Os indicadores/proxies de risco só podem ser agrupados se pertencerem à mesma categoria de comportamento ilícito
\end{itemize}


\subsubsection*{Qualidade e disponibilidade dos dados}

\begin{itemize}
	\item Aumento da qualidade dos dados numa escala temporal grande -> melhorias no processo de flagging
\end{itemize}



Desafios e vantagens de possuir dados em quantidade e qualidade : 

\begin{itemize}
	
	\item Construir algoritmos específicos para um determinado contexto de modo a definir situações padrão/normais e identificar situações suspeitas
	\item Ligar e relacionar dados de fontes diferentes : dados da fase de *procurement* e informação relativa à empresa e redes de contactos entre adjudicantes e adjudicadores **ou** licitadores e compradores 
	\item Permitir que governos e cidadãos tenham um maior conhecimento acerca do que é que é feito, onde e por quem. 
	
\end{itemize}




\subsection*{Parte III}

Passos envolvidos na deteção de corrupção :

\begin{itemize}
	\item 1. Agrupar red flags que indicam risco de corrupção ao longo de toda a cadeia do processo de contratação
	\item 2. Desevolver indicadores associados a cada red flag
	\item 3. Definir cada flag usando cálculos replicáveis
\end{itemize}

Existe um mecanismo de mapeamento que faz a ligação entre red flags e o conjunto de dados ( data field? )\\

Passos para desenvolver indicadores :
\\


\textbf{Step 1 : Identificar indicadores}\\

A partir deste \href{https://guide.iacrc.org/the-red-flags-of-corruption-bid-rigging-collusive-bidding-and-fraud/}{documento} desenvolvido pelo \href{https://iacrc.org/}{IACRC}, foi compilada uma lista de indicadores que sugerem a presença de corrupção. Desta lista, selecionaram-se 60 red flags.\\

\textbf{Step 2 : Definir cada flag}\\

Ao identificar potencias red flags, estas são transformadas em indicadores discretos e quantificáveis.\\

Estes sites calculam risk scores e identificam flags em  propostas : [[14]](https://www.redflags.eu/) [[15]](https://opentender.net/)\\

\textbf{Step 3 :Atribuir a cada flag um esquema de corrupção}\\


Atribui-se a cada indicador/red flag um ou mais esquemas de corrupção : fraude, corrupção, coação e fraude em licitação.\\
 

Se num contrato são ativadas várias red flags e todas elas apontam para o mesma esquema de corrupção, existe uma maior probabilidade de existir corrupção. \\



\textbf{Step 4 : Mapear para o OCDS} \\


Não percebi bem esta secção. 

Ao utilizar o nosso conjunto de dados presumo que nem todos os campos do dataset ( data fields / as diferentes colunas do dataset ) irão ser necessários no processo de flagging. Será boa ideia estudar e procurar quais são os melhores campos a considerar do dataset de modo a conseguir ligar uma determinada flag a um contrato ; ver quais as melhores combinações de campos para tipos de contrato semelhantes.  \\


\textbf{Step 5 :  Atribuição a fases do contrato}\\


Já vimos que existem 5 fases ao longo do processo de contrato : 

\begin{enumerate}
	
	\item Planning
	\item Tender
	\item Award
	\item Contract
	\item Implementation
	
\end{enumerate}


Vimos também que existem 4 tipos de esquema ilícitos : 

\begin{enumerate}
	
	\item Corrupção
	\item Fraude
	\item Coação
	\item Fraude em licitação
	
\end{enumerate}


O objetivo é ligar/mapear cada flag a um tipo de esquema ilícito e, também, a cada uma das fases do processo do contrato. 
Presumo que de todas as flags existentes, umas se adaptem melhor a determinados tipos de esquema ilícito e, por sua vez, em determinadas fases do contrato. 

  
%  <h4 style='color:salmon'> Exemplo : </h4>
%  
%  
%  <h4 style='color:salmon'> Flag 1 - Proposta / Orçamento apenas com um licitador - Binária </h4>
%  
%  - É preciso ter em conta a especificidade do cargo. Se for muito específico, é expectável um reduzido número de candidaturas. 
%  - Esta flag deve ser ativada quando num concurso em que é esperado um número significativo de candidaturas - por exemplo > 5 - apenas se verifica 1
%  
%  
%  <h4 style='color:salmon'> Flag 2 - Tempo de licitação é muito curto - Binária </h4>
%  
%  - Se o período de tempo para fazer propostas é muito curto podemos estar na presença de um caso em que é favorecida uma empresa, o que implica um injustiça em termos de competitividade. Ao tornar o tempo de licitação curto, estamos a aumentar a probabilidade de empresas concorrentes não verem a proposta / não conseguirem  candidatar-se a tempo, dando a vantagem a outra empresa do nosso agrado. 
%  - Nesta situação, é preciso definir *perído de tempo curto*. Se período de licitação for inferior a $x$, a flag toma o valor 1. Caso contrário, valor 0. 
%  
%  
%  <h4 style='color:salmon'> Flag 3 - Licitador que nunca tinha participado é vencedor </h4>
%  
%  - Esta é uma flag que pode ou não levantar suspeitas. O facto de uma empresa/licitador novo ganhar pode significar que, ao tornar aberto o processo de procurement, exista maior equidade e justiça entre concorrentes. 
%  
%  
%  <h4 style='color:salmon'> Flag 4 - Oferta muito próxima do orçamento </h4>
%  
%  - Se o valor de uma oferta por parte duma empresa no processo de candidatura é muito próximo do valor do orçamento establecido pode sugerir que a empresa a candidatar-se pode ter informações priveligiadas. 
%  
%  
%  <h4 style='color:salmon'> Flag 5 - Apenas o vencedor foi elegível a fim de receber o contrato </h4>
%  
%  - Quando apenas a empresa licitadora que venceu é que cumpre os requisitos necessários da candidatura e, além do mais, das restantes empresas participantes existem casos de desclassificação e irrgularidades no processo, existe a possibilidade de existir o benificiamento da empresa vencedora ( fornecendo informação relevante no que diz respeito ao processo de candiatura )
%  

\subsection*{Parte IV - Red Flag Mapping Tool}

\begin{itemize}
	
	\item Converte ideias e conceitos de peritos na área de integridade em ferramentas. 
	
	\item Lista, em Excel, de todas as \href{https://docs.google.com/spreadsheets/d/19OTpHcl92GHixsX31IfpDbs-jCnNr9KyDqu1cr38xsY/edit#gid=0}{flags}. Existe, também, este \href{https://docs.google.com/spreadsheets/d/12PFkUlQH09jQvcnORjcbh9-8d-NnIuk4mAQwdGiXeSM/edit#gid=656314485}{ficheiro}, aparentemente, mais recente. Ambos contém bastantes referências. 
\end{itemize}


Breve resumo de cada coluna : 

\begin{outline}[enumerate]
	
	\1 \textbf{Phase} : Planemento - Proposta - Concessão - Contrato - Implementação
	\1 \textbf{Associated Scheme}: Fase do processo pode contemplar um, ou mais, dos 17 esquemas ilícitos definidos pelo \href{https://guide.iacrc.org/the-red-flags-of-corruption-bid-rigging-collusive-bidding-and-fraud/}{IACRC}. 
		
		\2 fraude em licitações 
		\2 abuso de pedido de alteração ( ? )
		\2 coação
		\2 corrupção
		\2 excluir licitadores qualificados
		\2 declarações e afirmações falsas
		\2 faturas falsas, inflacionadas ou duplicadas
		\2 contrato fictício
		\2 interesses ocultos
		\2  *leaking* de informações sobre a licitação
		\2  manipulação das licitações
		\2  substituição de produto
		\2  informação fraudada
		\2  split purchases ( ? )
		\2  licitações *unbalanced*
		\2  prémios injustificados de fonte única
		\2  failure to meet contract specs ( ? ) 
	
	\1 - Red Flag
	\1 - Descrição do Indicador : Explicação detalhada
	\1 - Outros campos : Possible: OCDS? ; OCDS Field(s) ; Other data needed ; Possible : Ukraine ? 
	
\end{outline}



Nem sempre é possível detetar red flags em tempo real. Não é possível detetar a red flag 1 - apenas um licitador - na fase de licitação. Só nas fases posteriores é que seria possível detetar esse sinal.  

\textit{Reler melhor as últimas 3/4 páginas.}


\section*{Anexos}

\href{https://www.open-contracting.org/}{OCDS} 

\href{https://developmentgateway.org/}{Development Gateway}

\href{https://iacrc.org/training-courses/}{IACRC} : International Anti-Corruption Resource Center fights against fraud and corruption in international development and humanitarian projects

\href{https://guide.iacrc.org/}{Guide} to combating corruption and fraude in infrastructure development projects - made by IACRC

\href{https://guide.iacrc.org/proof-of-common-schemes/}{Most common schemes} and steps of proof - made by IACRC

\href{https://digiwhist.eu/wp-content/uploads/2016/08/GTI_WP2016_3_Fazekas-Cingolani-Toth_Conceptualising-PP-corr_160821.pdf}{A comprehensive review of objective corruption proxies in public procurement: risky actors, transactions, and vehicles of rent extraction}

\href{https://www.crcb.eu/}{CRCB} : Corruption Research Center Budapest

\href{https://www.crcb.eu/wp-content/uploads/2016/08/crcb_2016_ijtoth_mhajdu_competitive_intensity_160820_.pdf}{Competitive Intensity and Corruption Risks in the Hungarian Public Procurement 2009-2015} 

\href{https://okfn.de/blog/2016/06/who-has-won-the-contract/}{Identifying the bidders of public procurement processes} 

\href{https://www.openopps.com/blog/post/21/why-good-procurement-data-does-more-than-fight-corruption/}{Why good procurement data does more than fight corruption}

\href{https://digiwhist.eu/}{DIGWHIST} : Digital Whistleblower

\href{https://spendnetwork.com/}{Spend Network}

\href{https://digiwhist.eu/publications/towards-a-comprehensive-mapping-of-information-on-public-procurement-tendering-and-its-actors-across-europe/}{Towards a comprehensive mapping of information on public procurement tendering and its actors across Europe} 

\href{https://www.govtransparency.eu/wp-content/uploads/2016/10/Fazekas-David-Barrett_Public-procurement-review_public_151113.pdf}{Corruption Risks in UK Public Procurement and New AntiCorruption Tools} 

\href{https://www.open-contracting.org/what-is-open-contracting/}{Transforming public contracting through open data \& smarter engagement}

\href{https://guide.iacrc.org/the-red-flags-of-corruption-bid-rigging-collusive-bidding-and-fraud/}{Red Flags of Corruption, Bid Rigging, Collusive Bidding and Fraud}

\href{https://guide.iacrc.org/the-red-flags-of-corruption-bid-rigging-collusive-bidding-and-fraud/}{Government Transparency Institute}

\href{https://bpp.worldbank.org/}{World Bank}

\href{https://www.redflags.eu/}{Red Flags Project}

\href{https://opentender.eu/start}{Opentender}
\href{https://docs.google.com/spreadsheets/d/19OTpHcl92GHixsX31IfpDbs-jCnNr9KyDqu1cr38xsY/edit#gid=0}{Spreadsheet} with 120 flags

\href{https://docs.google.com/spreadsheets/d/12PFkUlQH09jQvcnORjcbh9-8d-NnIuk4mAQwdGiXeSM/edit#gid=656314485}{Spreadsheet} updated 

\href{https://docs.google.com/document/d/14sKj037Mn93k__CVIgzfLaUMmwufa67_fyWIGLU6Vog/edit}{Lack of electoral accountability and public procurement corruption} 

\href{https://www.crcb.eu/wp-content/uploads/2014/11/Fazekas-Toth_State_capture_PP_2014Nov.pdf}{From corruption to state capture: A new analytical framework with empirical applications from Hungary}

\href{https://www.crcb.eu/wp-content/uploads/2013/12/Fazekas-Toth-King_Composite-indicator_v2_2013.pdf}{Anatomy of grand corruption: A composite corruption risk index based on objective data}

\href{https://standard.open-contracting.org/latest/en/guidance/design/user_needs/#value-for-money-in-procurement}{OCDS User needs}

\href{https://www.crcb.eu/wp-content/uploads/2015/04/Toth-et-al_CRCB_WP_v2_150413.pdf}{Toolkit for detecting collusive bidding in public procurement}

\href{https://www.crcb.eu/wp-content/uploads/2016/08/crcb_2016_ijtoth_mhajdu_competitive_intensity_160820_.pdf}{Competitive Intensity and Corruption Risks in the Hungarian Public Procurement 2009-2015}

\end{document}
