\section{Conclusions}
	In this work, we synthesized Ca$_{3}$Mn$_{2-x}$Ti$_{x}$O$_{7}$ $n=2$ Ruddlesden-Popper samples with various compositions $(x=0.1,\,0.25,\,0.3)$ via a common solid-state reaction method and studied its structural phase transitions and properties through different methods: X-Ray Powder Diffraction (XRPD), Raman Spectroscopy, Perturbed Angular Correlation (PAC), and Density Functional Theory (DFT) simulations, with special focus on the $x=0.25$ (C$_{2}$MTO) system.
	
	After the last annealing of the synthesis process, Rietveld refinements on the XRPD data were performed to check the purity of the formed $n=2$ phases. We observed an approximate $12\%$ presence of Ca$_{2}$Mn$_{0.875}$Ti$_{0.125}$O$_{4}$ ($n=1$ C$_{1}$MTO) in C$_{2}$MTO, however with $a-$ and $b-$lattice\-parameters considerably lower (by $\approx 0.6\AA$) of what can be found in the literature for the undoped Ca$_{2}$MnO$_{4}$ system.
	
	Raman and PAC spectroscopies, respectively in the $93-653$K and $74-1224$K temperature ranges, revealed a first order structural phase transition from what appears to be a the ferroelectric $A2_{1}am$ (s.g. 36) phase to an intermediate and paraelectric $Acaa$ (s.g. 68) symmetry within a $300-550$K temperature range, where both structural phases coexist. Thus, when comparing CMTO to the undoped Ca$_{3}$Mn$_{2}$O$_{7}$ (C$_{2}$MO), we report an increase in the percentage of the ferroelectric phase at room temperature by introducing the $x=0.25$ Ti$^{4+}$ doping. However, our results point to the possibility of the ground state not corresponding to the $A2_{1}am$ symmetry, but to one with inequivalent A-sites at the rock-salt of the RP system. PAC spectroscopy also revealed a second order structural phase transition from the intermediate orthorhombic $Acaa$ (s.g. 68) to the tetragonal $I4/mmm$ (s.g. 139) at 1150K.
	
	We came across an unexpected result at temperatures below 300K, where we clearly detected the probing of a second local environment, albeit without a clear physical origin. Interestingly, PAC measurements did not found evidence of the C$_{1}$MTO phase observed from XRPD refinements, and we believe this suggests that the $A2_{1}am$ space group might not correspond to the ground state symmetry for every composition of Ca$_{3}$Mn$_{2-x}$Ti$_{x}$O$_{7}$ mixed B-site systems. Phonon calculations on these doped systems in the $A2_{1}am$ symmetry could reveal if there are any lattice instabilities, shedding a light on whether the $A2_{1}am$ still corresponds to the ground state when Ti is added. 
	
	The C$_{2}$MO system was studied with DFT in it's ground state $A2_{1}am$ symmetry. We determined a Hubbard-$U$ correction by first principles $(U=6\text{ eV})$ to properly describe the Mn$-3d$ states, and then performed a series of structural and electronic properties calculations with our \textit{ab-initio} value and with $U=0$ and $U=3.9 \text{ eV}$, a value commonly found in the literature. We determined the cell parameters and bulk moduli for the mentioned $U$ values as well as multiple electronic properties, such as the projected density-of-states (PDOS), Bader charges, and the Electric Field Gradient at multiple sites. When we manage to define and adjust the most adequate parameters to compute the EFGs for this system, we intend to probe the Ti$^{4+}$ doped Ca$_{3}$Mn$_{2-x}$Ti$_{x}$O$_{7}$ structures, and analyse how the EFGs vary as a function of doping concentration.

\section{Future Work}
	In the future, we intend to deepen the discussion at several stages of this system's characterization. Regarding the observation of a C$_{1}$MTO phase impurity in XRPD experiments, and the underestimated lattice parameters, we intend to pursue a new refinement considering also the intermediate $Acaa$ phase. To complement this measurements, XAS and XPS spectroscopies would help us to confirm the oxidation states of Mn and Ti in the sample, to account for some deviations in the XRPD patterns. Additionally, neutron diffraction would confirm the structural and magnetic structural properties of these samples.
	
	More complex Raman Spectroscopy setups could be pursued to study the evolution of the softer modes that drive the $A2_1am-Acaa$ structural phase transition, and DFT calculations could assign the peak positions for each mode in the doped C$_{2}$MTO structure.
	
	The interpretation PAC results below 300K is still open to discussion, and we'll be pursing other methods to clarify them. Doped CMTO DFT calculations of the EFG parameters is a work in progress and the study of other doped systems is also scheduled, to later compare with undoped systems. In fact, $n=1$ Ca$_{2}$(Mn,Ti)O$_{4}$ RP compounds have already been measured by PAC and will be analysed in the future. Future phonon calculations on these B-site mixed systems may become a key point in understanding these systems.