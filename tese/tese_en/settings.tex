% !TEX root = main.tex

%----------------------------------------------------------------------------------------
%	PARAMETERS FOR FCUP THESIS TITLEPAGES/BOOK COVER
%----------------------------------------------------------------------------------------

\usepackage{titling,titlesec,pdfpages}
\usepackage[utf8]{inputenc}
%\usepackage[dvipsnames,prologue,table]{pstricks}% For defining images in the page background and other tweeks
\usepackage{babel}% The package manages cul­tur­ally-de­ter­mined ty­po­graph­i­cal (and other) rules,

\usepackage{amsmath,amssymb,amsfonts,amsthm}% Para permitir escrever acentos, caracteres especiais e outros em ambiente "MATH"  

\usepackage{ifthen}		%Allows for using conditions in latex text
\usepackage{iftex}		%Allows for testing whether PDFTeX, or XeTeX, or LuaTeX is being used for typesetting
\usepackage{calc}		%Para poder fazer cálculos de variáveis no codigo
\usepackage{contour}	%Para Definir o a espessura de contorno de letras (ex:PhD)
\usepackage{notoccite}	%Prevent trouble from citations in table of contents, etc
\usepackage{caption}	%pro­vides many ways to cus­tomise the cap­tions in float­ing en­vi­ron­ments

%%%%%\usepackage{subfig}
\usepackage{subcaption}
\captionsetup{margin=10pt,font={footnotesize},labelfont=bf}%,labelsep=endash}

\def\blankpage{%
	\clearpage%
	\thispagestyle{empty}%
	\addtocounter{page}{-1}%
	\null%
	\clearpage}


%\usepackage{relsize}
\usepackage{bm}
%\usepackage{braket}
%\usepackage[euler]{textgreek}
\usepackage{mathtools}
%\usepackage{siunitx}
%\usepackage{algpseudocode}

%\usepackage{tikz}
%\usetikzlibrary{matrix}

%
%\newcommand{\TN}{$T_{\text{N}}$~}
%\newcommand{\TC}{$T_{\text{C}}$~}
%
%
%
%\newcommand{\tj}[6]{ \begin{pmatrix}
%  #1 & #2 & #3 \\
%  #4 & #5 & #6
%\end{pmatrix}}

% ------------------------------------
%%%%%%%%%%%%%%%%%%%%%%%%%%%%%%%%%%%%%%%%%%%%%%%%%%%%%%%%%%%%%%%%%%%%%%%%%%%
%%%%%%%%%%%%%%%%%%%%	Definition of green check mark %%%%%%%%%%%%%%%%%%%
%%%%%%%%%%%%%%%%%%%%%%%%%%%%%%%%%%%%%%%%%%%%%%%%%%%%%%%%%%%%%%%%%%%%%%%%%%

\usepackage{pifont}% http://ctan.org/pkg/pifont
\newcommand{\cmark}{\textcolor{green}{\ding{51}}}%
\newcommand{\xmark}{\textcolor{red}{\ding{55}}}%


\newcommand{\greencheck}{}%
\DeclareRobustCommand{\greencheck}{%
  \tikz\fill[scale=0.4, color=green]
  (0,.35) -- (.25,0) -- (1,.7) -- (.25,.15) -- cycle;%
}
%%%%%%%%%%%%%%%%%%%%%%%%%%%%%%%%%%%%%%%%%%%%%%%%%%%%%%%%%%%%%%%%%%%%%%%%%%
%%%%%%%%%%%%%%%%%%%%	Definition of red cross mark %%%%%%%%%%%%%%%%%%%%%
%%%%%%%%%%%%%%%%%%%%%%%%%%%%%%%%%%%%%%%%%%%%%%%%%%%%%%%%%%%%%%%%%%%%%%%%%%
\newcommand{\redxmark}{}%
\DeclareRobustCommand{\redxmark}{%
  \tikz\fill[scale=0.4, color=red]
  (0,.35) -- (.25,0) -- (1,.7) -- (.25,.15) -- cycle;%
}  

% ------------------------------------
% +++++++++++++++++++++++++++++++++++++++++++++++++++++++++++++++++++++++++++++++++
%							Configurações do pacote geometry
% +++++++++++++++++++++++++++++++++++++++++++++++++++++++++++++++++++++++++++++++++
\usepackage{geometry}			%For Better control of page dimensions												
\geometry{
    a4paper,					%		A4 paper dimension
    includehead,				%		Include header when defining the dimensions
    includefoot,				%		Include footer when defining the dimensions
    twoside,					%		Para definir automaticamente lombadas diferentes nas paginas pares e impares
    %showframe,					%		For showing the different dimensions of the page
    %twoside=true,%				%		For two sided pages, different margins in odd and even pages
    left=100.0pt
    }
\geometry{
	left={3cm},
	right={3cm},
	top={3cm},
	bottom={2.8cm},
}



% +++++++++++++++++++++++++++++++++++++++++++++++++++++++++++++++++++++++++++++++++
% 						Dimensões e definições da página
% +++++++++++++++++++++++++++++++++++++++++++++++++++++++++++++++++++++++++++++++++
\paperwidth = 597.50787pt 
\paperheight = 845.04684pt

%%\oddsidemargin = 31.0pt
\topmargin = 36.135pt 

\voffset = -72.26999pt
\hoffset = 0.0pt

\textheight = 674pt
\textwidth = 418.25368pt 

\headheight = 40.0pt
\headsep = 21.68121pt 
%%\headsep = 0.3in

\footskip = 27.46295pt 

\marginparsep = 7.0pt 
\marginparwidth = 116.0pt 

\marginparpush = 5.0pt 



% +++++++++++++++++++++++++++++++++++++++++++++++++++++++++++++++++++++++++++++++++
%						Definir o espaçamento entre linhas
% +++++++++++++++++++++++++++++++++++++++++++++++++++++++++++++++++++++++++++++++++
\usepackage{setspace}		% For defining line spacing 
%%\doublespacing : this is on option. The other one is the one below : 
\onehalfspacing

% ------------------------------------
%\usepackage[no-math]{fontspec}
%%%%%%%%%%%%%%%%%%%%%%%%%%%%%%%%%%%%%%%%%%%%%%%%%%%%%%%%%%%%%%%%%%%%%%%%%%%
%%%%%%%%%%%%%%%%%%%%%%% PARA DEFINIR O TIPO DE LETRA  %%%%%%%%%%%%%%%%%%%%
%%%%%%%%%%%%%%%%%%%%%%%%%%%%%%%%%%%%%%%%%%%%%%%%%%%%%%%%%%%%%%%%%%%%%%%%%%

\usepackage[no-math]{fontspec}	%Para permitir personalizações ao tipo de fonte no OSX
\setmainfont[Ligatures=TeX]{Arial}

\setmainfont{Arial}

\setsansfont{Arial}[Scale=MatchLowercase]
\setmonofont{Arial}[Scale=MatchLowercase]

\sffamily
\renewcommand{\familydefault}{\sfdefault}

\newfontfamily\headfont{Arial}
\newcommand\texthead[1]{\headfont #1}

%%%%%%%%%%%%%%%%%%%%%%%%%%%%%%%%%%%%%%%%%%%%%%%%%%%%%%%%%%%%%%%%%%%%%%%%%%
%%%%%%%%%%%%%%%%%%% PARA DEFINIR O TIPO DE LETRA MATHMODE %%%%%%%%%%%%%%%%
%%%%%%%%%%%%%%%%%%%%%%%%%%%%%%%%%%%%%%%%%%%%%%%%%%%%%%%%%%%%%%%%%%%%%%%%%%

%\usepackage[cmintegrals,	%makes use of integrals drawn from Computer Modern rather than the more upright, 
%							%but unattractive, txfonts integrals.
%scaled=1.00,				%allows you to scale all fonts in this math package to match a chosen text font family
%nosymbolsc,					%saves you a math group of mostly rather obscure symbols
%noamssymbols,				%saves you one or two math groups (the AMS symbols) if you have no need of them
%						    %uprightGreek,													
%						    %specifies the use of upright rather than slanted Greek symbols for upper case only.
%						    %frenchmath														
%						    %forces uppercase and lowercase Greek to upright shape and makes uppercase math
%							% Roman letters render in upright rather than slanted shape
%]{newtxsf}
%
%\usepackage[italic,defaultmathsizes,selfGreek]{mathastext}

% ------------------------------------
%%%%%%%%%%%%%%%%%%%%%%%%%%%%%%%%%%%%%%%%%%%%%%%%%%%%%%%%%%%%%%%%%%%%%%%%%%%
%%%%%%%%%%%%%%%%%%%%%%%%  Special symbols %%%%%%%%%%%%%%%%%%%%%%%%%%%%%%%%
%%%%%%%%%%%%%%%%%%%%%%%%%%%%%%%%%%%%%%%%%%%%%%%%%%%%%%%%%%%%%%%%%%%%%%%%%%

\DeclareSymbolFont{symbolsC}{U}{pxsyc}{m}{n}
\SetSymbolFont{symbolsC}{bold}{U}{pxsyc}{bx}{n}
\DeclareFontSubstitution{U}{pxsyc}{m}{n}
\DeclareMathSymbol{\medcirc}{\mathbin}{symbolsC}{7}

\DeclareSymbolFont{symbolsC}{U}{pxsyc}{m}{n}
\SetSymbolFont{symbolsC}{bold}{U}{pxsyc}{bx}{n}
\DeclareFontSubstitution{U}{pxsyc}{m}{n}
\DeclareMathSymbol{\medbullet}{\mathbin}{symbolsC}{8}


% ------------------------------------
%%%%%%%%%%%%%%%%%%%%%%%%%%%%%V%%%%%%%%%%%%%%%%%%%%%%%%%%%%%%%%%%%%%%%%%%%%%
%%%%%%%%%%%%%%%%%%%  DEFINIÇÃO DOS CABEÇALHOS   %%%%%%%%%%%%%%%%%%%%%%%%%%
%pra retirar cabeçalhos nas páginas brancas antes de inicio de capítulo
%%%%%%%%%%%%%%%%%%%%%%%%%%%%%%%%%%%%%%%%%%%%%%%%%%%%%%%%%%%%%%%%%%%%%%%%%%

\usepackage{fancyhdr}%Para cabeçalhos personalizados			Options: Sonny, Lenny, Glenn, Conny, Rejne, Bjarne, Bjornstrup
\pagestyle{fancy}
\fancyhf{}

\makeatletter
\def\cleardoublepage{\clearpage\if@twoside \ifodd\c@page\else                   
    \hbox{}
    \thispagestyle{empty}
    \newpage
    \if@twocolumn\hbox{}\newpage\fi\fi\fi}
\makeatother

%%%%%%%%%%%%%%%%%%%%%%%%%%%%%%%%%%%%%%%%%%%%%%%%%%%%%%%%%%%%%%%%%%%%%%%%%%
%%%% para definir o nome dos capitulos em minusculas no cabeçalho %%%%%%%
%%%%%%%%%%%%%%%%%%%%%%%%%%%%%%%%%%%%%%%%%%%%%%%%%%%%%%%%%%%%%%%%%%%%%%%%%%
                        
\renewcommand{\chaptermark}[1]{\markboth{#1}{#1}}     
\renewcommand{\sectionmark}[1]{\markright{\thesection\ #1}}


%%%%%%%%%%%%%%%%%%%%%%%%%%%%%%%%%%%%%%%%%%%%%%%%%%%%%%%%%%%%%%%%%%%%%%%%%%
%%%%%% para definir a espessura da linha para zero. Tirar a linha %%%%%%%%
%%%%%%%%%%%%%%%%%%%%%%%%%%%%%%%%%%%%%%%%%%%%%%%%%%%%%%%%%%%%%%%%%%%%%%%%%%

\renewcommand{\headrulewidth}{0pt}
\renewcommand{\footrulewidth}{0pt}



%%%%%%%%%%%%%%%%%%%%%%%%%%%%%%%%%%%%%%%%%%%%%%%%%%%%%%%%%%%%%%%%%%%%%%%%%%
%%%%Cabeçalhos para para as páginas impares do lado direito
%%%%%%%%%%%%%%%%%%%%%%%%%%%%%%%%%%%%%%%%%%%%%%%%%%%%%%%%%%%%%%%%%%%%%%%%%%

%\fancyhead[RO]{
%\begin{tabular}{r|c}
%{\fontsize{8pt}{8pt}\selectfont FCUP}&\multirow{2}{*}{\fontsize{8pt}{10pt}\selectfont \thepage}\\
%{\fontsize{8pt}{8pt}\selectfont\nouppercase\rightmark}&\\
%          \end{tabular}%
%}

\fancyhead[RO,RE]{
	\begin{tabular}{r|c}
		{\fontsize{8pt}{8pt}\selectfont FCUP}&\multirow{2}{*}{\fontsize{8pt}{10pt}\selectfont \thepage}\\
		{\fontsize{8pt}{8pt}\selectfont\nouppercase\rightmark}&\\
	\end{tabular}%
}

%%%%%%%%%%%%%%%%%%%%%%%%%%%%%%%%%%%%%%%%%%%%%%%%%%%%%%%%%%%%%%%%%%%%%%%%%%
%%%%Cabeçalhos para para as páginas pares do lado esquerdo
%%%%%%%%%%%%%%%%%%%%%%%%%%%%%%%%%%%%%%%%%%%%%%%%%%%%%%%%%%%%%%%%%%%%%%%%%%

%\fancyhead[LE]{
%\begin{tabular}{c|l}
%\multirow{2}{*}{\fontsize{8pt}{10pt}\selectfont \thepage}&{\fontsize{8pt}{8pt}\selectfont FCUP}\\
%&{\fontsize{8pt}{8pt}\selectfont\nouppercase\leftmark}\\
%          \end{tabular}%
%}



%%%%%%%%%%%%%%%%%%%%%%%%%%%%%%%%%%%%%%%%%%%%%%%%%%%%%%%%%%%%%%%%%%%%%%%%%%
%%%%Cabeçalhos para para as páginas pares do lado direito e impares do lado esquerdo
%%%%%%%%%%%%%%%%%%%%%%%%%%%%%%%%%%%%%%%%%%%%%%%%%%%%%%%%%%%%%%%%%%%%%%%%%%

\fancyhead[LO,LE]{\raisebox{-0.4\height}{\includegraphics[height=25pt, keepaspectratio=true]{front-matter/fcup_clipped.png}}
}


%%%%%%%%%%%%%%%%%%%%%%%%%%%%%%%%%%%%%%%%%%%%%%%%%%%%%%%%%%%%%%%%%%%%%%%%%%
%%%%%%%%%%%%%%%%%%  PERSONALIZAÇÂO DE CABEÇALHOS %%%%%%%%%%%%%%%%%%%%%%%%%
%%%%%%%%%%%%%%%%%%%%%%%%%%%%%%%%%%%%%%%%%%%%%%%%%%%%%%%%%%%%%%%%%%%%%%%%%%
\usepackage{sectsty}         %Para definir o tamanho letra sections (tem que ser antes do fancychap)
\usepackage{fncychap}		%Para personalizar letras capitulos		

% ------------------------------------
% %%%%%%%%%%%%%%%%%%%%%%%%%%%%%%%%%%%%%%%%%%%%%%%%%%%%%%%%%%%%%%%%%%%%%%%%%%
%%ALTERAR A DISTANCIA DO TOPO DA PAGINA ATAO MODULO COM IFORMAcao CAPITULO
%%%%%%%%%%%%%%%%%%%%%%%%%%%%%%%%%%%%%%%%%%%%%%%%%%%%%%%%%%%%%%%%%%%%%%%%%%

%%%%%%%%  %%%% For the case \chapter:
%%%%%%%%  \makeatletter
%%%%%%%%  \renewcommand*{\@makechapterhead}[1]{%
%%%%%%%%    \vspace*{1\p@}%
%%%%%%%%    {\parindent \z@ \raggedright \normalfont
%%%%%%%%      \ifnum \c@secnumdepth >\m@ne
%%%%%%%%        \if@mainmatter%%%%% Fix for frontmatter, mainmatter, and backmatter 040920
%%%%%%%%          \DOCH
%%%%%%%%        \fi
%%%%%%%%      \fi
%%%%%%%%      \interlinepenalty\@M
%%%%%%%%      \if@mainmatter%%%%% Fix for frontmatter, mainmatter, and backmatter 060424
%%%%%%%%        \DOTI{#1}%
%%%%%%%%      \else%
%%%%%%%%        \DOTIS{#1}%
%%%%%%%%      \fi
%%%%%%%%    }}
%%%%%%%%    
%%%%%%%%  %%%% For the case \chapter*:
%%%%%%%%  \renewcommand*{\@makeschapterhead}[1]{%
%%%%%%%%    \vspace*{1\p@}%
%%%%%%%%    {\parindent \z@ \raggedright
%%%%%%%%      \normalfont
%%%%%%%%      \interlinepenalty\@M
%%%%%%%%      \DOTIS{#1}
%%%%%%%%      \vskip 10\p@
%%%%%%%%    }}
%%%%%%%%  \makeatother










%%%%%%%%%%%%%%%%%%%%%%%%%%%%%%%%%%%%%%%%%%%%%%%%%%%%%%%%%%%%%%%%%%%%%%%%%%
%%%%COMANDO PARA DEFINIR A PROFUNDIDADE DA NUMERACAO DOS CAP,SEC,...-
%%%%%%%%%%%%%%%%%%%%%%%%%%%%%%%%%%%%%%%%%%%%%%%%%%%%%%%%%%%%%%%%%%%%%%%%%%

\setcounter{secnumdepth}{6}
\setcounter{tocdepth}{6}

%%%%%%%%%%%%%%%%%%%%%%%%%%%%%%%%%%%%%%%%%%%%%%%%%%%%%%%%%%%%%%%%%%%%%%%%%%
%%%%Para definir a quebrea de linhas e palavras
%%%%%%%%%%%%%%%%%%%%%%%%%%%%%%%%%%%%%%%%%%%%%%%%%%%%%%%%%%%%%%%%%%%%%%%%%%

\hyphenpenalty=50 % default 50
\tolerance=200      % default 200

\flushbottom

%%%%%%%%%%%%%%%%%%%%%%%%%%%%%%%%%%%%%%%%%%%%%%%%%%%%%%%%%%%%%%%%%%%%%%%%%%
%%%%PARA DEFINIR O DETALHE NA NUMERAcao AUTOMATICA%%%%%%%%%%%%%-
%%%%%%%%%%%%%%%%%%%%%%%%%%%%%%%%%%%%%%%%%%%%%%%%%%%%%%%%%%%%%%%%%%%%%%%%%%

\numberwithin{equation}{chapter}
\numberwithin{figure}{chapter}
\numberwithin{table}{chapter}

% ------------------------------------
% +++++++++++++++++++++++++++++++++++++++++++++++++++++++++++++++++++++++++++++++++
%								DEFINIR O TAMANHO LETRA
% +++++++++++++++++++++++++++++++++++++++++++++++++++++++++++++++++++++++++++++++++

%\usepackage{fontspec}
%\setmainfont{Times New Roman}

%%%% Tamanho da fonte do CHAPTER
\ChNameVar{\fontsize{14}{16}\usefont{OT1}{phv}{m}{n}\selectfont} \ChNumVar{\fontsize{60}{62}\usefont{OT1}{ptm}{m}{n}\selectfont} \ChTitleVar{\Huge\bfseries\rm} \ChRuleWidth{1pt}

%%%% Tamanho da fonte do X
\ChNumVar{\raggedleft\fontsize{20pt}{25pt}\usefont{OT1}{ptm}{m}{n}\selectfont}
\ChTitleVar{\raggedright\rm\fontsize{20pt}{25pt}\bfseries\selectfont}
\ChNameAsIs
\ChNameVar{[1]}

%%%% \chapterfont{\Large}

\ChNameUpperCase% Definir o estilo de CAPITULO X
\ChNameVar{\Huge\sf\bf\flushleft}% CAPITULO
\ChNumVar{\Huge\sf\bf}% X
\ChTitleVar{\huge\bf\raggedright}% Mudar o tamanho, tipo de letra, e outras coisas no %%%% nome do capitulo

\sectionfont{\Large\raggedright}% Mudar o tipo de letra das SECTIONS
\subsectionfont{\large\raggedright}% Mudar o tipo de letra das SUBSECTIONS
\subsubsectionfont{\normalsize\raggedright}% Mudar o tipo de letra das SUBSUBSECTIONS

% ------------------------------------
% +++++++++++++++++++++++++++++++++++++++++++++++++++++++++++++++++++++++++++++++++
%		  COMANDO PARA INSERIR PAGINAS EM BRANCO E COMECAR EM PAGINA IMPAR 
% +++++++++++++++++++++++++++++++++++++++++++++++++++++++++++++++++++++++++++++++++

\newcommand{\newevenside}{
	\ifthenelse{\isodd{}}{\newpage}{
	\newpage
\thispagestyle{fancy}
\fancyhf{empty}
\fancyhead{empty}
	\textcolor{white}{placeholder}
	\thispagestyle{empty}
	\newpage
	}
}



% +++++++++++++++++++++++++++++++++++++++++++++++++++++++++++++++++++++++++++++++++
% 					  COMANDO PARA INSERIR PAGINAS EM BRANCO
% +++++++++++++++++++++++++++++++++++++++++++++++++++++++++++++++++++++++++++++++++
\newcommand{\clearemptydoublepage}{\newpage{\pagestyle{empty}\cleardoublepage}}





% %%%%%%%%%%%%%%%%%%%%%%%%%%%%%%
%\usepackage{tocloft}% Control table of contents, figures, etc

\usepackage{graphicx}
\usepackage{wrapfig}% Para colocar figuras rodeadas de texto
\usepackage{float}% Para permitir o movimento the certos elementos
\usepackage{enumerate}% Para permitir enumerações personalizadas i), a), 1), ...
%\usepackage{xecolor}% Para permitir texto a cores
\definecolor{fcupcolor}{RGB}{153,205,255}

\newlength{\spinew}
\newlength{\coverwidth}
\newlength{\sidew}

\usepackage{multirow}       %Para permitir multi linhas em tabelas
\usepackage{multicol}		%Para permitir multi colunas em tabelas
\usepackage{booktabs}		%Enhances the quality of tables
\usepackage[figuresright]{rotating}       %Para permitir rotação de tabelas e texto em tabelas
%\usepackage{tablefootnote}
%\usepackage{colortbl}

%%\usepackage{genmpage}
%%\usepackage{lscape}



\usepackage{todonotes}                          						%Para permitir  to do notes.
\usepackage{makeidx}                            						%Para poder fazer um index
\usepackage{footnote}


%%%%%%%%%%%%%%%%%%%%%%%%%%%%%%%%%%%%%%%%%%%%%%%%%%%%%%%%%%%%%%%%%%%%%%%%%%
%%%%    PARA DEFINIR O ESTILO DO NUMERO NA NOTA DE RODAP--
%%%%%%%%%%%%%%%%%%%%%%%%%%%%%%%%%%%%%%%%%%%%%%%%%%%%%%%%%%%%%%%%%%%%%%%%%%

%\renewcommand{\thefootnote}{\roman{footnote}}
%\renewcommand{\thefootnote}{\arabic{footnote}}  %Para as notas de rodapé
%\interfootnotelinepenalty=10000
%%%%\renewcommand{\thefootnote}{\fnsymbol{footnote}}
%%%%\renewcommand{\thefootnote}{\roman{footnote}}

%\usepackage{cancel}   			%para permitir rasuras nas formulas matemáticas
\usepackage{gensymb}

%%%%%%%%%%%%%%%%%%%%%%%%%%%%%%%%%%%%%%%%%%%%%%%%%%%%%%%%%%%%%%%%%%%%%%%%%%%
%%%% DEFINI RELACIONADAS COM A BIBLIOGRAFIA
%%%%%%%%%%%%%%%%%%%%%%%%%%%%%%%%%%%%%%%%%%%%%%%%%%%%%%%%%%%%%%%%%%%%%%%%%%

\usepackage[square,					%Sqaure brackets around the numbers
comma,								%Commas separating the numbers
numbers,							%numbers and not names of references
sort&compress,						%group consecutive references
%url=false,											%
super]{natbib}                      %Para poder personalizar bibliografia
\setlength{\bibsep}{5.0pt}			%Separação entre cada item da bibliografia
\usepackage{doi}					% Para poder ter mais controlo no campo do DOI na bilbiografia
%\bibliographystyle{msc_style}		% Ficheiro de estilo da bibliografia. Alterar aqui o que fica a negrito, a italico, etc

\usepackage[noprefix]{nomencl}		%Para poder criar listas de Abreviaes ou nomenclaturas
\usepackage[version=4,arrows=pgf]{mhchem} 
\usepackage{chemformula}

\usepackage{xfrac}



%%%%%%%%%%%%%%%%%%%%%%%%%%%%%%%%%%%%%%%%%%%%%%%%%%%%%%%%%%%%%%%%%%%%%%%%%%
%%%%%%%%%%%%%%%%%%%%%%%%%%%%%%%%%%%%%%%%%%%%%%%%%%%%%%%%%%%%%%%%%%%%%%%%%%
%%%%%%%%%%%%%%%%%%%%%%%%%%%%%%

\usepackage{mathptmx,cite,lipsum,physics,xcolor,graphics,colortbl,pgfplotstable}
\definecolor{Gray}{gray}{0.92}

\usepackage{hyperref}               %Para as hyperligações ao longo do texto
\hypersetup{final=true,			%
            raiselinks=true,		%
            pdftoolbar=true,		%show or hide Acrobatas toolbar
            pdfmenubar=true,		%show or hide Acrobaat menu
            pdffitwindow=true,		%
            pdftitle={MSc-Thesis},	%	define the title that gets displayed in the "Document Info" window of Acrobat
            pdfauthor={Francisco Valente},%	the name of the PDF author, it works like the one above
            colorlinks=true,		%	surround the links by color frames (false) or colors the text of the links (true).
            %The color of these links can be configured using the following options (default colors are shown):
            linkcolor=blue,		%color of internal links (sections, pages, etc.)
            linktoc=all,        	%	defines which part of an entry in the table of contents is made into a link (=none,section,page,all	)
            unicode=true,		%	allows to use characters of non-Latin based languages in Acrobat bookmarks
            citecolor=blue,		%	color of citation links (bibliography),
            filecolor=blue,
            urlcolor=blue,
            anchorcolor = blue,
            %allcolors=blue,
            %hidelinks=false
}

%\usepackage{siunitx}


\usepackage[nameinlink,capitalize]{cleveref}
\usepackage[bottom]{footmisc}

\usepackage{doi}

%%%%%%%%%%%%%%%%%%%%%%%%%%%%%%%%%%%%%%%%%%%%%%%%%%%%%%%%%%%%%%%%%%%%%%%%%%
%%%%%%%%%%%%%%%%%%%%%% To create a sorted list items  %%%%%%%%%%%%%%%%%%%%
%%%%%%%%%%%%%%%%%%%%%%%%%%%%%%%%%%%%%%%%%%%%%%%%%%%%%%%%%%%%%%%%%%%%%%%%%%
%%%%%%%%%%%%%%%%%%%%%%%%%%%%%%
\usepackage{acronym}
\usepackage{datatool}							% http://ctan.org/pkg/datatool
\newcommand{\sortitem}[2][\relax]{%	%			%
  \DTLnewrow{list}%	%							% Create a new entry
  \ifx#1\relax%
    \DTLnewdbentry{list}{sortlabel}{#2}%			% Add entry sortlabel (no optional argument)
  \else%
  	\DTLnewdbentry{list}{sortlabel}{#1}%			% Add entry sortlabel (optional argument)
  \fi%											%
  \DTLnewdbentry{list}{description}{#2}%			% Add entry description
}%
\newenvironment{sortedlist}{\DTLifdbexists{list}{\DTLcleardb{list}}{\DTLnewdb{list}}}{
  \DTLsort{sortlabel}{list}%						% Sort list
  \begin{itemize}%								%
    \DTLforeach*{list}{\theDesc=description}{\item[]\theDesc}%
  \end{itemize}%								%
}%
%
%\newcommand{\sortitem}[2]{%
%	\DTLnewrow{list}%
%	\DTLnewdbentry{list}{label}{#1}%
%	\DTLnewdbentry{list}{description}{#2}%
%}
%
%\newenvironment{sortedlist}{%
%	\DTLifdbexists{list}{\DTLcleardb{list}}{\DTLnewdb{list}}%
%}{%
%	\DTLsort{label}{list}%
%	\begin{description}%
%		\DTLforeach*{list}{\theLabel=label,\theDesc=description}{%
%			\item[\theLabel] \theDesc }%
%	\end{description}%
%}

%%%%%%%%%%%%%%%%%%%%%%%%%%%%%%%%%%%%%%%%%%%%%%%%%%%%%%%%%%%%%%%%%%%%%%%%%%
%%%%%%%%%%%%% New style of enumerate for abbreviations page  %%%%%%%%%%%%%
%%%%%%%%%%%%%%%%%%%%%%%%%%%%%%%%%%%%%%%%%%%%%%%%%%%%%%%%%%%%%%%%%%%%%%%%%%

\newenvironment{my_enumerate}
{\begin{enumerate}
  \setlength{\itemsep}{0.2pt}
  \setlength{\parskip}{0pt}
  \setlength{\parsep}{0pt}}
{\end{enumerate}}


























% % Author name
% \authorname[mailto:example@fc.up.pt]{Ricardo Manuel Alves Pacheco Moreira}

% % Affiliation number 2
% % USAGE: \otheraffiliation[url]{relative/path/to/}{INITIAL}{University name}
% %\otheraffiliation[http://uni2.pt]{logos/logo2}{UNI2}{Universidade/Faculdade 2}

% % Affiliation number 3
% % USAGE: \extraaffiliation[url]{relative/path/to/logo}{INITIAL}{University name}
% %\extraaffiliation[http://uni3.pt]{logos/logo3}{UNI3}{Universidade/Faculdade 3}


% % Degrre name
% \degreename{Mestrado em Física}


% % Field of science
% \sciencefield{Física}


% % Department name
% \department[http://dfa.fc.up.pt/]{Departamento de Física e Astronomia}


% % Supervisor info
% \supervisor[mailto:example@fc.up.pt]{Dra. Armandina Maria Lima Lopes} % Supervisor name
% %\supervisorposition{Categoria} % Supervisor name position/Category (comment out to hide this field)
% \supervisoraffiliation[]{Faculdade de Ciências} % Supervisor university/faculty % Supervisor university/faculty
% %\supervisoraffiliation[]{Instituto de Engenharia de Sistemas e Computadores, Tecnologia e Ciência}


% % Cosupervisor info ----- Comment out if not needed
% \cosupervisor[mailto:example@fc.up.pt]{Prof. João Pedro Esteves de Araújo} % Cosupervisor name
% %\cosupervisorposition{Categoria} % Cosupervisor name position/Category (comment out to hide this field)
% \cosupervisoraffiliation[]{Faculdade de Ciências} % Supervisor university/faculty
