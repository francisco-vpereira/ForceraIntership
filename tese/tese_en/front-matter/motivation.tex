\chapter*{Motivation}
\label{ch:motivation}
\addcontentsline{toc}{chapter}{Motivation}

%Hybrid Improper Ferroelectric (HIF) systems constitute an encouraging set of materials in the search for room temperature ferroelectrics and multiferroics\cite{benedekHybridImproperFerroelectricity2011}, the latter being remarkably rare and sought after\cite{hillWhyAreThere2000}, since the combination of ferroelectricity and ferromagnetism in the same phase may lead to incredible possibilities and practical breakthroughs, such as the long awaited electric field controlled magnetic data storages\cite{spaldinRenaissanceMagnetoelectricMultiferroics2005}.

%Systems that present HIF offer a wide range of applications as magneto-electrics or ferroelectric random-access memories (FRAMs)\cite{panPerovskiteMaterialsSynthesis2016}, with recent studies aiming to fine tune their properties, such as polarizability \cite{sennNegativeThermalExpansion2015}. Ruddlesden-Popper (RP) perovskites, in particular $n=2$ Ca$_{3}(\text{Mn,Ti})_{2}$O$_{7}$ systems, have been very promising in this regard, due to their naturally layered structure and the presence of Mn ions, which allow for an unconventional condensation of a multiferroic phase below 110K\cite{benedekWhyAreThere2013,gaoInterrelationDomainStructures2017}. This is still far from room temperature, nothing like promised above. However, the origin of these system's functional properties and, most notably, the structural phase transitions they take from the high symmetry to the ground state, have been the subject of much controversy\cite{sennNegativeThermalExpansion2015,glamazdaSoftTiltRotational2018,rocha-rodriguesCa3Mn2O7StructuralPath2020}. Hence, a good understanding of the mechanisms behind structural phase transitions in these and other systems presenting HIF may pave the way for important discoveries in the quest for room temperature multiferroics. 

%That is precisely the context in which we developed this work. In addition, we intend to explore and systematize a deeper connection between the interpretation of PAC experiments with first principles calculations of Electric Field Gradients (EFGs) via DFT\cite{petrilliElectricfieldgradientCalculationsUsing1998}, aiming to ally two very powerful techniques in the characterization of solid state systems.