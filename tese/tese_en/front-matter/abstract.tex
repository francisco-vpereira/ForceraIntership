\chapter*{Abstract}
\label{chap:abstract}
\addcontentsline{toc}{chapter}{Abstract}

%In this work, Ca$_{3}$Mn$_{2-x}$Ti$_{x}$O$_{7}$ $n=2$ Ruddlesden-Popper (RP) systems were synthesized and characterized in detail at the macroscopic and local scales. In particular, the $x=0.25$ composition (CMTO) was examined with X-Ray Powder Diffraction (XRPD), Raman Spectroscopy, and Perturbed Angular Correlation (PAC). Density Functional Theory (DFT) calculations in the $x=0$ undoped system (CMO) were also carried.
%Such systems belong to the $n=2$ family of Ruddlesden-Popper layered perovskites, which have attracted much interest due to the presence of Hybrid Improper Ferroelectricity. 

%The samples were synthesised via a common solid-state reaction method and the phase purity checked by XRPD measurements at 300K.

%With Raman ($93-653$K) and PAC ($74-1224$K) spectroscopies, we found a first order structural phase transition from what appears to be the ferroelectric $A2_{1}am$ (s.g. 36) phase to an intermediate and paraelectric $Acaa$ (s.g. 68) symmetry within a $300-550$K temperature range, where both phases coexist. Thus, in that case, when compared to CMO, we report an increase in the percentage of the ferroelectric phase at room temperature due to the Ti$^{4+}$ concentration. However, our results point to the possibility of the ground state not corresponding to the $A2_{1}am$ symmetry, but to one with inequivalent A-sites at the rock-salt of the RP system. Such statement will be examined in future works with DFT considerations. PAC spectroscopy also revealed a second order structural phase transition from the intermediate orthorhombic $Acaa$ (s.g. 68) to the tetragonal $I4/mmm$ (s.g. 139) at 1150K, as was previously reported for CMO.

%The Ca$_{3}$Mn$_{2}$O$_{7}-A2_{1}am$ system was studied with DFT. We determined a Hubbard-$U$ correction by first principles $(U=6\text{ eV})$ to properly describe the Mn $3d-$orbitals, and then compared the structural and electronic properties with $U=0$ and with a value commonly found in the literature $(U=3.9 \text{ eV})$. The cell parameters and bulk moduli were calculated for the mentioned $U$ values, as well each system's electronic properties, such as the projected density-of-states (PDOS), Bader charges, and the Electric Field Gradient at multiple lattice sites.