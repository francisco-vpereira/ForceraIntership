
\chapter*{Resumo}
\label{Chp:Resumo}
\addcontentsline{toc}{chapter}{Resumo}

%Neste trabalho em perovskites do tipo Ruddlesden-Popper (RP), amostras de Ca$_{3}$Mn$_{2-x}$Ti$_{x}$O$_{7}$ foram sintetizadas e caracterizadas em detalhe à escala macroscópica e à escala local. Em particular, a composição com $x=0.25$ (CMTO) foi examinada por Difracção de Raios-X (XRPD), Espectroscopia de Raman, e por Perturbação Angular Correlacionada (PAC). Foram também levadas a cabo simulações de Teoria do Funcional da Densidade (DFT) no sistema com $x=0$ (CMO).
%As amostras foram sintetizadas por reacção de estado sólido e a pureza da fase de interesse estudada com medições de XRPD a 300K.
%Através das espectroscopias de Raman ($93-653$K) e PAC ($74-1224$K), observamos uma transição de fase estrutural de primeira ordem, de uma fase que aparenta corresponder a uma ferroeléctrica de simetria $A2_{1}am$ (s.g. 36), para uma paraeléctrica intermédia, $Acaa$ (s.g. 68), num gama de temperaturas ente 300 e 550K, onde ambas as fases coexistem. Portanto, nesse caso, reportamos um aumento da percentagem da fase ferroelétrica à temperatura ambiente, quando comparado com CMO, devido à introdução de iões Ti$^{4+}$. No entanto, os nossos resultados de PAC apontam para a possibilidade de o estado fundamental no CMTO não corresponder à simetria $A2_{1}am$, mas a uma outra com sítios-A inequivalentes na \textit{rock-salt} da estrutura. Esta hipótese será explorada em trabalhos futuros através de cálculos de DFT. Espectroscopia PAC também revelou uma transição de fase estrutural de segunda ordem a 1150K da simetria $Acaa$ (s.g. 68) ortorrômbica para uma estrutura tetragonal $I4/mmm$ (s.g. 139).
%O sistema não dopado de Ca$_{3}$Mn$_{2}$O$_{7}$ na simetria $A2_{1}am$ foi estudado com DFT. Determinou-se o parâmetro $U$ de Hubbard por primeiros princípios $(U=6\text{ eV})$ para procurar descrever corretamente os estados Mn$-3d$. As propriedades estruturais e electrónicas calculadas foram comparadas com $U=0$ e com um valor comumente encontrado na literatura $(U=3.9 \text{ eV})$. Foram determinados os parâmetros de rede e \textit{bulk-modulus} para os valores de $U$ mencionados, além das propriedades electrónicas como as Densidades de Estados Projetadas (PDOS) do sistema, cargas de Bader, e Gradiantes de Campo Elétrico em vários sítios da rede. 

