% +++++++++++++++++++++++++++++++++++++++++++++++++++++++++++++++++++++++++++++++++
%							    Configurações Iniciais 
% +++++++++++++++++++++++++++++++++++++++++++++++++++++++++++++++++++++++++++++++++
\documentclass[11pt,openright]{book}



% +++++++++++++++++++++++++++++++++++++++++++++++++++++++++++++++++++++++++++++++++
% 						Ficheiro .tex com packages necessários 
%
%  			Este ficheiro, por sua vez, dá import a outro ficheiros .tex que se 
%  						encontram no diretório settings
%
% +++++++++++++++++++++++++++++++++++++++++++++++++++++++++++++++++++++++++++++++++
% !TEX root = main.tex

%----------------------------------------------------------------------------------------
%	PARAMETERS FOR FCUP THESIS TITLEPAGES/BOOK COVER
%----------------------------------------------------------------------------------------

\usepackage{titling,titlesec,pdfpages}
\usepackage[utf8]{inputenc}
%\usepackage[dvipsnames,prologue,table]{pstricks}% For defining images in the page background and other tweeks
\usepackage{babel}% The package manages cul­tur­ally-de­ter­mined ty­po­graph­i­cal (and other) rules,

\usepackage{amsmath,amssymb,amsfonts,amsthm}% Para permitir escrever acentos, caracteres especiais e outros em ambiente "MATH"  

\usepackage{ifthen}		%Allows for using conditions in latex text
\usepackage{iftex}		%Allows for testing whether PDFTeX, or XeTeX, or LuaTeX is being used for typesetting
\usepackage{calc}		%Para poder fazer cálculos de variáveis no codigo
\usepackage{contour}	%Para Definir o a espessura de contorno de letras (ex:PhD)
\usepackage{notoccite}	%Prevent trouble from citations in table of contents, etc
\usepackage{caption}	%pro­vides many ways to cus­tomise the cap­tions in float­ing en­vi­ron­ments

%%%%%\usepackage{subfig}
\usepackage{subcaption}
\captionsetup{margin=10pt,font={footnotesize},labelfont=bf}%,labelsep=endash}

\def\blankpage{%
	\clearpage%
	\thispagestyle{empty}%
	\addtocounter{page}{-1}%
	\null%
	\clearpage}


%\usepackage{relsize}
\usepackage{bm}
%\usepackage{braket}
%\usepackage[euler]{textgreek}
\usepackage{mathtools}
%\usepackage{siunitx}
%\usepackage{algpseudocode}

%\usepackage{tikz}
%\usetikzlibrary{matrix}

%
%\newcommand{\TN}{$T_{\text{N}}$~}
%\newcommand{\TC}{$T_{\text{C}}$~}
%
%
%
%\newcommand{\tj}[6]{ \begin{pmatrix}
%  #1 & #2 & #3 \\
%  #4 & #5 & #6
%\end{pmatrix}}

% ------------------------------------
%%%%%%%%%%%%%%%%%%%%%%%%%%%%%%%%%%%%%%%%%%%%%%%%%%%%%%%%%%%%%%%%%%%%%%%%%%%
%%%%%%%%%%%%%%%%%%%%	Definition of green check mark %%%%%%%%%%%%%%%%%%%
%%%%%%%%%%%%%%%%%%%%%%%%%%%%%%%%%%%%%%%%%%%%%%%%%%%%%%%%%%%%%%%%%%%%%%%%%%

\usepackage{pifont}% http://ctan.org/pkg/pifont
\newcommand{\cmark}{\textcolor{green}{\ding{51}}}%
\newcommand{\xmark}{\textcolor{red}{\ding{55}}}%


\newcommand{\greencheck}{}%
\DeclareRobustCommand{\greencheck}{%
  \tikz\fill[scale=0.4, color=green]
  (0,.35) -- (.25,0) -- (1,.7) -- (.25,.15) -- cycle;%
}
%%%%%%%%%%%%%%%%%%%%%%%%%%%%%%%%%%%%%%%%%%%%%%%%%%%%%%%%%%%%%%%%%%%%%%%%%%
%%%%%%%%%%%%%%%%%%%%	Definition of red cross mark %%%%%%%%%%%%%%%%%%%%%
%%%%%%%%%%%%%%%%%%%%%%%%%%%%%%%%%%%%%%%%%%%%%%%%%%%%%%%%%%%%%%%%%%%%%%%%%%
\newcommand{\redxmark}{}%
\DeclareRobustCommand{\redxmark}{%
  \tikz\fill[scale=0.4, color=red]
  (0,.35) -- (.25,0) -- (1,.7) -- (.25,.15) -- cycle;%
}  

% ------------------------------------
% +++++++++++++++++++++++++++++++++++++++++++++++++++++++++++++++++++++++++++++++++
%							Configurações do pacote geometry
% +++++++++++++++++++++++++++++++++++++++++++++++++++++++++++++++++++++++++++++++++
\usepackage{geometry}			%For Better control of page dimensions												
\geometry{
    a4paper,					%		A4 paper dimension
    includehead,				%		Include header when defining the dimensions
    includefoot,				%		Include footer when defining the dimensions
    twoside,					%		Para definir automaticamente lombadas diferentes nas paginas pares e impares
    %showframe,					%		For showing the different dimensions of the page
    %twoside=true,%				%		For two sided pages, different margins in odd and even pages
    left=100.0pt
    }
\geometry{
	left={3cm},
	right={3cm},
	top={3cm},
	bottom={2.8cm},
}



% +++++++++++++++++++++++++++++++++++++++++++++++++++++++++++++++++++++++++++++++++
% 						Dimensões e definições da página
% +++++++++++++++++++++++++++++++++++++++++++++++++++++++++++++++++++++++++++++++++
\paperwidth = 597.50787pt 
\paperheight = 845.04684pt

%%\oddsidemargin = 31.0pt
\topmargin = 36.135pt 

\voffset = -72.26999pt
\hoffset = 0.0pt

\textheight = 674pt
\textwidth = 418.25368pt 

\headheight = 40.0pt
\headsep = 21.68121pt 
%%\headsep = 0.3in

\footskip = 27.46295pt 

\marginparsep = 7.0pt 
\marginparwidth = 116.0pt 

\marginparpush = 5.0pt 



% +++++++++++++++++++++++++++++++++++++++++++++++++++++++++++++++++++++++++++++++++
%						Definir o espaçamento entre linhas
% +++++++++++++++++++++++++++++++++++++++++++++++++++++++++++++++++++++++++++++++++
\usepackage{setspace}		% For defining line spacing 
%%\doublespacing : this is on option. The other one is the one below : 
\onehalfspacing

% ------------------------------------
%\usepackage[no-math]{fontspec}
%%%%%%%%%%%%%%%%%%%%%%%%%%%%%%%%%%%%%%%%%%%%%%%%%%%%%%%%%%%%%%%%%%%%%%%%%%%
%%%%%%%%%%%%%%%%%%%%%%% PARA DEFINIR O TIPO DE LETRA  %%%%%%%%%%%%%%%%%%%%
%%%%%%%%%%%%%%%%%%%%%%%%%%%%%%%%%%%%%%%%%%%%%%%%%%%%%%%%%%%%%%%%%%%%%%%%%%

\usepackage[no-math]{fontspec}	%Para permitir personalizações ao tipo de fonte no OSX
\setmainfont[Ligatures=TeX]{Arial}

\setmainfont{Arial}

\setsansfont{Arial}[Scale=MatchLowercase]
\setmonofont{Arial}[Scale=MatchLowercase]

\sffamily
\renewcommand{\familydefault}{\sfdefault}

\newfontfamily\headfont{Arial}
\newcommand\texthead[1]{\headfont #1}

%%%%%%%%%%%%%%%%%%%%%%%%%%%%%%%%%%%%%%%%%%%%%%%%%%%%%%%%%%%%%%%%%%%%%%%%%%
%%%%%%%%%%%%%%%%%%% PARA DEFINIR O TIPO DE LETRA MATHMODE %%%%%%%%%%%%%%%%
%%%%%%%%%%%%%%%%%%%%%%%%%%%%%%%%%%%%%%%%%%%%%%%%%%%%%%%%%%%%%%%%%%%%%%%%%%

%\usepackage[cmintegrals,	%makes use of integrals drawn from Computer Modern rather than the more upright, 
%							%but unattractive, txfonts integrals.
%scaled=1.00,				%allows you to scale all fonts in this math package to match a chosen text font family
%nosymbolsc,					%saves you a math group of mostly rather obscure symbols
%noamssymbols,				%saves you one or two math groups (the AMS symbols) if you have no need of them
%						    %uprightGreek,													
%						    %specifies the use of upright rather than slanted Greek symbols for upper case only.
%						    %frenchmath														
%						    %forces uppercase and lowercase Greek to upright shape and makes uppercase math
%							% Roman letters render in upright rather than slanted shape
%]{newtxsf}
%
%\usepackage[italic,defaultmathsizes,selfGreek]{mathastext}

% ------------------------------------
%%%%%%%%%%%%%%%%%%%%%%%%%%%%%%%%%%%%%%%%%%%%%%%%%%%%%%%%%%%%%%%%%%%%%%%%%%%
%%%%%%%%%%%%%%%%%%%%%%%%  Special symbols %%%%%%%%%%%%%%%%%%%%%%%%%%%%%%%%
%%%%%%%%%%%%%%%%%%%%%%%%%%%%%%%%%%%%%%%%%%%%%%%%%%%%%%%%%%%%%%%%%%%%%%%%%%

\DeclareSymbolFont{symbolsC}{U}{pxsyc}{m}{n}
\SetSymbolFont{symbolsC}{bold}{U}{pxsyc}{bx}{n}
\DeclareFontSubstitution{U}{pxsyc}{m}{n}
\DeclareMathSymbol{\medcirc}{\mathbin}{symbolsC}{7}

\DeclareSymbolFont{symbolsC}{U}{pxsyc}{m}{n}
\SetSymbolFont{symbolsC}{bold}{U}{pxsyc}{bx}{n}
\DeclareFontSubstitution{U}{pxsyc}{m}{n}
\DeclareMathSymbol{\medbullet}{\mathbin}{symbolsC}{8}


% ------------------------------------
%%%%%%%%%%%%%%%%%%%%%%%%%%%%%V%%%%%%%%%%%%%%%%%%%%%%%%%%%%%%%%%%%%%%%%%%%%%
%%%%%%%%%%%%%%%%%%%  DEFINIÇÃO DOS CABEÇALHOS   %%%%%%%%%%%%%%%%%%%%%%%%%%
%pra retirar cabeçalhos nas páginas brancas antes de inicio de capítulo
%%%%%%%%%%%%%%%%%%%%%%%%%%%%%%%%%%%%%%%%%%%%%%%%%%%%%%%%%%%%%%%%%%%%%%%%%%

\usepackage{fancyhdr}%Para cabeçalhos personalizados			Options: Sonny, Lenny, Glenn, Conny, Rejne, Bjarne, Bjornstrup
\pagestyle{fancy}
\fancyhf{}

\makeatletter
\def\cleardoublepage{\clearpage\if@twoside \ifodd\c@page\else                   
    \hbox{}
    \thispagestyle{empty}
    \newpage
    \if@twocolumn\hbox{}\newpage\fi\fi\fi}
\makeatother

%%%%%%%%%%%%%%%%%%%%%%%%%%%%%%%%%%%%%%%%%%%%%%%%%%%%%%%%%%%%%%%%%%%%%%%%%%
%%%% para definir o nome dos capitulos em minusculas no cabeçalho %%%%%%%
%%%%%%%%%%%%%%%%%%%%%%%%%%%%%%%%%%%%%%%%%%%%%%%%%%%%%%%%%%%%%%%%%%%%%%%%%%
                        
\renewcommand{\chaptermark}[1]{\markboth{#1}{#1}}     
\renewcommand{\sectionmark}[1]{\markright{\thesection\ #1}}


%%%%%%%%%%%%%%%%%%%%%%%%%%%%%%%%%%%%%%%%%%%%%%%%%%%%%%%%%%%%%%%%%%%%%%%%%%
%%%%%% para definir a espessura da linha para zero. Tirar a linha %%%%%%%%
%%%%%%%%%%%%%%%%%%%%%%%%%%%%%%%%%%%%%%%%%%%%%%%%%%%%%%%%%%%%%%%%%%%%%%%%%%

\renewcommand{\headrulewidth}{0pt}
\renewcommand{\footrulewidth}{0pt}



%%%%%%%%%%%%%%%%%%%%%%%%%%%%%%%%%%%%%%%%%%%%%%%%%%%%%%%%%%%%%%%%%%%%%%%%%%
%%%%Cabeçalhos para para as páginas impares do lado direito
%%%%%%%%%%%%%%%%%%%%%%%%%%%%%%%%%%%%%%%%%%%%%%%%%%%%%%%%%%%%%%%%%%%%%%%%%%

%\fancyhead[RO]{
%\begin{tabular}{r|c}
%{\fontsize{8pt}{8pt}\selectfont FCUP}&\multirow{2}{*}{\fontsize{8pt}{10pt}\selectfont \thepage}\\
%{\fontsize{8pt}{8pt}\selectfont\nouppercase\rightmark}&\\
%          \end{tabular}%
%}

\fancyhead[RO,RE]{
	\begin{tabular}{r|c}
		{\fontsize{8pt}{8pt}\selectfont FCUP}&\multirow{2}{*}{\fontsize{8pt}{10pt}\selectfont \thepage}\\
		{\fontsize{8pt}{8pt}\selectfont\nouppercase\rightmark}&\\
	\end{tabular}%
}

%%%%%%%%%%%%%%%%%%%%%%%%%%%%%%%%%%%%%%%%%%%%%%%%%%%%%%%%%%%%%%%%%%%%%%%%%%
%%%%Cabeçalhos para para as páginas pares do lado esquerdo
%%%%%%%%%%%%%%%%%%%%%%%%%%%%%%%%%%%%%%%%%%%%%%%%%%%%%%%%%%%%%%%%%%%%%%%%%%

%\fancyhead[LE]{
%\begin{tabular}{c|l}
%\multirow{2}{*}{\fontsize{8pt}{10pt}\selectfont \thepage}&{\fontsize{8pt}{8pt}\selectfont FCUP}\\
%&{\fontsize{8pt}{8pt}\selectfont\nouppercase\leftmark}\\
%          \end{tabular}%
%}



%%%%%%%%%%%%%%%%%%%%%%%%%%%%%%%%%%%%%%%%%%%%%%%%%%%%%%%%%%%%%%%%%%%%%%%%%%
%%%%Cabeçalhos para para as páginas pares do lado direito e impares do lado esquerdo
%%%%%%%%%%%%%%%%%%%%%%%%%%%%%%%%%%%%%%%%%%%%%%%%%%%%%%%%%%%%%%%%%%%%%%%%%%

\fancyhead[LO,LE]{\raisebox{-0.4\height}{\includegraphics[height=25pt, keepaspectratio=true]{front-matter/fcup_clipped.png}}
}


%%%%%%%%%%%%%%%%%%%%%%%%%%%%%%%%%%%%%%%%%%%%%%%%%%%%%%%%%%%%%%%%%%%%%%%%%%
%%%%%%%%%%%%%%%%%%  PERSONALIZAÇÂO DE CABEÇALHOS %%%%%%%%%%%%%%%%%%%%%%%%%
%%%%%%%%%%%%%%%%%%%%%%%%%%%%%%%%%%%%%%%%%%%%%%%%%%%%%%%%%%%%%%%%%%%%%%%%%%
\usepackage{sectsty}         %Para definir o tamanho letra sections (tem que ser antes do fancychap)
\usepackage{fncychap}		%Para personalizar letras capitulos		

% ------------------------------------
% %%%%%%%%%%%%%%%%%%%%%%%%%%%%%%%%%%%%%%%%%%%%%%%%%%%%%%%%%%%%%%%%%%%%%%%%%%
%%ALTERAR A DISTANCIA DO TOPO DA PAGINA ATAO MODULO COM IFORMAcao CAPITULO
%%%%%%%%%%%%%%%%%%%%%%%%%%%%%%%%%%%%%%%%%%%%%%%%%%%%%%%%%%%%%%%%%%%%%%%%%%

%%%%%%%%  %%%% For the case \chapter:
%%%%%%%%  \makeatletter
%%%%%%%%  \renewcommand*{\@makechapterhead}[1]{%
%%%%%%%%    \vspace*{1\p@}%
%%%%%%%%    {\parindent \z@ \raggedright \normalfont
%%%%%%%%      \ifnum \c@secnumdepth >\m@ne
%%%%%%%%        \if@mainmatter%%%%% Fix for frontmatter, mainmatter, and backmatter 040920
%%%%%%%%          \DOCH
%%%%%%%%        \fi
%%%%%%%%      \fi
%%%%%%%%      \interlinepenalty\@M
%%%%%%%%      \if@mainmatter%%%%% Fix for frontmatter, mainmatter, and backmatter 060424
%%%%%%%%        \DOTI{#1}%
%%%%%%%%      \else%
%%%%%%%%        \DOTIS{#1}%
%%%%%%%%      \fi
%%%%%%%%    }}
%%%%%%%%    
%%%%%%%%  %%%% For the case \chapter*:
%%%%%%%%  \renewcommand*{\@makeschapterhead}[1]{%
%%%%%%%%    \vspace*{1\p@}%
%%%%%%%%    {\parindent \z@ \raggedright
%%%%%%%%      \normalfont
%%%%%%%%      \interlinepenalty\@M
%%%%%%%%      \DOTIS{#1}
%%%%%%%%      \vskip 10\p@
%%%%%%%%    }}
%%%%%%%%  \makeatother










%%%%%%%%%%%%%%%%%%%%%%%%%%%%%%%%%%%%%%%%%%%%%%%%%%%%%%%%%%%%%%%%%%%%%%%%%%
%%%%COMANDO PARA DEFINIR A PROFUNDIDADE DA NUMERACAO DOS CAP,SEC,...-
%%%%%%%%%%%%%%%%%%%%%%%%%%%%%%%%%%%%%%%%%%%%%%%%%%%%%%%%%%%%%%%%%%%%%%%%%%

\setcounter{secnumdepth}{6}
\setcounter{tocdepth}{6}

%%%%%%%%%%%%%%%%%%%%%%%%%%%%%%%%%%%%%%%%%%%%%%%%%%%%%%%%%%%%%%%%%%%%%%%%%%
%%%%Para definir a quebrea de linhas e palavras
%%%%%%%%%%%%%%%%%%%%%%%%%%%%%%%%%%%%%%%%%%%%%%%%%%%%%%%%%%%%%%%%%%%%%%%%%%

\hyphenpenalty=50 % default 50
\tolerance=200      % default 200

\flushbottom

%%%%%%%%%%%%%%%%%%%%%%%%%%%%%%%%%%%%%%%%%%%%%%%%%%%%%%%%%%%%%%%%%%%%%%%%%%
%%%%PARA DEFINIR O DETALHE NA NUMERAcao AUTOMATICA%%%%%%%%%%%%%-
%%%%%%%%%%%%%%%%%%%%%%%%%%%%%%%%%%%%%%%%%%%%%%%%%%%%%%%%%%%%%%%%%%%%%%%%%%

\numberwithin{equation}{chapter}
\numberwithin{figure}{chapter}
\numberwithin{table}{chapter}

% ------------------------------------
% +++++++++++++++++++++++++++++++++++++++++++++++++++++++++++++++++++++++++++++++++
%								DEFINIR O TAMANHO LETRA
% +++++++++++++++++++++++++++++++++++++++++++++++++++++++++++++++++++++++++++++++++

%\usepackage{fontspec}
%\setmainfont{Times New Roman}

%%%% Tamanho da fonte do CHAPTER
\ChNameVar{\fontsize{14}{16}\usefont{OT1}{phv}{m}{n}\selectfont} \ChNumVar{\fontsize{60}{62}\usefont{OT1}{ptm}{m}{n}\selectfont} \ChTitleVar{\Huge\bfseries\rm} \ChRuleWidth{1pt}

%%%% Tamanho da fonte do X
\ChNumVar{\raggedleft\fontsize{20pt}{25pt}\usefont{OT1}{ptm}{m}{n}\selectfont}
\ChTitleVar{\raggedright\rm\fontsize{20pt}{25pt}\bfseries\selectfont}
\ChNameAsIs
\ChNameVar{[1]}

%%%% \chapterfont{\Large}

\ChNameUpperCase% Definir o estilo de CAPITULO X
\ChNameVar{\Huge\sf\bf\flushleft}% CAPITULO
\ChNumVar{\Huge\sf\bf}% X
\ChTitleVar{\huge\bf\raggedright}% Mudar o tamanho, tipo de letra, e outras coisas no %%%% nome do capitulo

\sectionfont{\Large\raggedright}% Mudar o tipo de letra das SECTIONS
\subsectionfont{\large\raggedright}% Mudar o tipo de letra das SUBSECTIONS
\subsubsectionfont{\normalsize\raggedright}% Mudar o tipo de letra das SUBSUBSECTIONS

% ------------------------------------
% +++++++++++++++++++++++++++++++++++++++++++++++++++++++++++++++++++++++++++++++++
%		  COMANDO PARA INSERIR PAGINAS EM BRANCO E COMECAR EM PAGINA IMPAR 
% +++++++++++++++++++++++++++++++++++++++++++++++++++++++++++++++++++++++++++++++++

\newcommand{\newevenside}{
	\ifthenelse{\isodd{}}{\newpage}{
	\newpage
\thispagestyle{fancy}
\fancyhf{empty}
\fancyhead{empty}
	\textcolor{white}{placeholder}
	\thispagestyle{empty}
	\newpage
	}
}



% +++++++++++++++++++++++++++++++++++++++++++++++++++++++++++++++++++++++++++++++++
% 					  COMANDO PARA INSERIR PAGINAS EM BRANCO
% +++++++++++++++++++++++++++++++++++++++++++++++++++++++++++++++++++++++++++++++++
\newcommand{\clearemptydoublepage}{\newpage{\pagestyle{empty}\cleardoublepage}}





% %%%%%%%%%%%%%%%%%%%%%%%%%%%%%%
%\usepackage{tocloft}% Control table of contents, figures, etc

\usepackage{graphicx}
\usepackage{wrapfig}% Para colocar figuras rodeadas de texto
\usepackage{float}% Para permitir o movimento the certos elementos
\usepackage{enumerate}% Para permitir enumerações personalizadas i), a), 1), ...
%\usepackage{xecolor}% Para permitir texto a cores
\definecolor{fcupcolor}{RGB}{153,205,255}

\newlength{\spinew}
\newlength{\coverwidth}
\newlength{\sidew}

\usepackage{multirow}       %Para permitir multi linhas em tabelas
\usepackage{multicol}		%Para permitir multi colunas em tabelas
\usepackage{booktabs}		%Enhances the quality of tables
\usepackage[figuresright]{rotating}       %Para permitir rotação de tabelas e texto em tabelas
%\usepackage{tablefootnote}
%\usepackage{colortbl}

%%\usepackage{genmpage}
%%\usepackage{lscape}



\usepackage{todonotes}                          						%Para permitir  to do notes.
\usepackage{makeidx}                            						%Para poder fazer um index
\usepackage{footnote}


%%%%%%%%%%%%%%%%%%%%%%%%%%%%%%%%%%%%%%%%%%%%%%%%%%%%%%%%%%%%%%%%%%%%%%%%%%
%%%%    PARA DEFINIR O ESTILO DO NUMERO NA NOTA DE RODAP--
%%%%%%%%%%%%%%%%%%%%%%%%%%%%%%%%%%%%%%%%%%%%%%%%%%%%%%%%%%%%%%%%%%%%%%%%%%

%\renewcommand{\thefootnote}{\roman{footnote}}
%\renewcommand{\thefootnote}{\arabic{footnote}}  %Para as notas de rodapé
%\interfootnotelinepenalty=10000
%%%%\renewcommand{\thefootnote}{\fnsymbol{footnote}}
%%%%\renewcommand{\thefootnote}{\roman{footnote}}

%\usepackage{cancel}   			%para permitir rasuras nas formulas matemáticas
\usepackage{gensymb}

%%%%%%%%%%%%%%%%%%%%%%%%%%%%%%%%%%%%%%%%%%%%%%%%%%%%%%%%%%%%%%%%%%%%%%%%%%%
%%%% DEFINI RELACIONADAS COM A BIBLIOGRAFIA
%%%%%%%%%%%%%%%%%%%%%%%%%%%%%%%%%%%%%%%%%%%%%%%%%%%%%%%%%%%%%%%%%%%%%%%%%%

\usepackage[square,					%Sqaure brackets around the numbers
comma,								%Commas separating the numbers
numbers,							%numbers and not names of references
sort&compress,						%group consecutive references
%url=false,											%
super]{natbib}                      %Para poder personalizar bibliografia
\setlength{\bibsep}{5.0pt}			%Separação entre cada item da bibliografia
\usepackage{doi}					% Para poder ter mais controlo no campo do DOI na bilbiografia
%\bibliographystyle{msc_style}		% Ficheiro de estilo da bibliografia. Alterar aqui o que fica a negrito, a italico, etc

\usepackage[noprefix]{nomencl}		%Para poder criar listas de Abreviaes ou nomenclaturas
\usepackage[version=4,arrows=pgf]{mhchem} 
\usepackage{chemformula}

\usepackage{xfrac}



%%%%%%%%%%%%%%%%%%%%%%%%%%%%%%%%%%%%%%%%%%%%%%%%%%%%%%%%%%%%%%%%%%%%%%%%%%
%%%%%%%%%%%%%%%%%%%%%%%%%%%%%%%%%%%%%%%%%%%%%%%%%%%%%%%%%%%%%%%%%%%%%%%%%%
%%%%%%%%%%%%%%%%%%%%%%%%%%%%%%

\usepackage{mathptmx,cite,lipsum,physics,xcolor,graphics,colortbl,pgfplotstable}
\definecolor{Gray}{gray}{0.92}

\usepackage{hyperref}               %Para as hyperligações ao longo do texto
\hypersetup{final=true,			%
            raiselinks=true,		%
            pdftoolbar=true,		%show or hide Acrobatas toolbar
            pdfmenubar=true,		%show or hide Acrobaat menu
            pdffitwindow=true,		%
            pdftitle={MSc-Thesis},	%	define the title that gets displayed in the "Document Info" window of Acrobat
            pdfauthor={Francisco Valente},%	the name of the PDF author, it works like the one above
            colorlinks=true,		%	surround the links by color frames (false) or colors the text of the links (true).
            %The color of these links can be configured using the following options (default colors are shown):
            linkcolor=blue,		%color of internal links (sections, pages, etc.)
            linktoc=all,        	%	defines which part of an entry in the table of contents is made into a link (=none,section,page,all	)
            unicode=true,		%	allows to use characters of non-Latin based languages in Acrobat bookmarks
            citecolor=blue,		%	color of citation links (bibliography),
            filecolor=blue,
            urlcolor=blue,
            anchorcolor = blue,
            %allcolors=blue,
            %hidelinks=false
}

%\usepackage{siunitx}


\usepackage[nameinlink,capitalize]{cleveref}
\usepackage[bottom]{footmisc}

\usepackage{doi}

%%%%%%%%%%%%%%%%%%%%%%%%%%%%%%%%%%%%%%%%%%%%%%%%%%%%%%%%%%%%%%%%%%%%%%%%%%
%%%%%%%%%%%%%%%%%%%%%% To create a sorted list items  %%%%%%%%%%%%%%%%%%%%
%%%%%%%%%%%%%%%%%%%%%%%%%%%%%%%%%%%%%%%%%%%%%%%%%%%%%%%%%%%%%%%%%%%%%%%%%%
%%%%%%%%%%%%%%%%%%%%%%%%%%%%%%
\usepackage{acronym}
\usepackage{datatool}							% http://ctan.org/pkg/datatool
\newcommand{\sortitem}[2][\relax]{%	%			%
  \DTLnewrow{list}%	%							% Create a new entry
  \ifx#1\relax%
    \DTLnewdbentry{list}{sortlabel}{#2}%			% Add entry sortlabel (no optional argument)
  \else%
  	\DTLnewdbentry{list}{sortlabel}{#1}%			% Add entry sortlabel (optional argument)
  \fi%											%
  \DTLnewdbentry{list}{description}{#2}%			% Add entry description
}%
\newenvironment{sortedlist}{\DTLifdbexists{list}{\DTLcleardb{list}}{\DTLnewdb{list}}}{
  \DTLsort{sortlabel}{list}%						% Sort list
  \begin{itemize}%								%
    \DTLforeach*{list}{\theDesc=description}{\item[]\theDesc}%
  \end{itemize}%								%
}%
%
%\newcommand{\sortitem}[2]{%
%	\DTLnewrow{list}%
%	\DTLnewdbentry{list}{label}{#1}%
%	\DTLnewdbentry{list}{description}{#2}%
%}
%
%\newenvironment{sortedlist}{%
%	\DTLifdbexists{list}{\DTLcleardb{list}}{\DTLnewdb{list}}%
%}{%
%	\DTLsort{label}{list}%
%	\begin{description}%
%		\DTLforeach*{list}{\theLabel=label,\theDesc=description}{%
%			\item[\theLabel] \theDesc }%
%	\end{description}%
%}

%%%%%%%%%%%%%%%%%%%%%%%%%%%%%%%%%%%%%%%%%%%%%%%%%%%%%%%%%%%%%%%%%%%%%%%%%%
%%%%%%%%%%%%% New style of enumerate for abbreviations page  %%%%%%%%%%%%%
%%%%%%%%%%%%%%%%%%%%%%%%%%%%%%%%%%%%%%%%%%%%%%%%%%%%%%%%%%%%%%%%%%%%%%%%%%

\newenvironment{my_enumerate}
{\begin{enumerate}
  \setlength{\itemsep}{0.2pt}
  \setlength{\parskip}{0pt}
  \setlength{\parsep}{0pt}}
{\end{enumerate}}


























% % Author name
% \authorname[mailto:example@fc.up.pt]{Ricardo Manuel Alves Pacheco Moreira}

% % Affiliation number 2
% % USAGE: \otheraffiliation[url]{relative/path/to/}{INITIAL}{University name}
% %\otheraffiliation[http://uni2.pt]{logos/logo2}{UNI2}{Universidade/Faculdade 2}

% % Affiliation number 3
% % USAGE: \extraaffiliation[url]{relative/path/to/logo}{INITIAL}{University name}
% %\extraaffiliation[http://uni3.pt]{logos/logo3}{UNI3}{Universidade/Faculdade 3}


% % Degrre name
% \degreename{Mestrado em Física}


% % Field of science
% \sciencefield{Física}


% % Department name
% \department[http://dfa.fc.up.pt/]{Departamento de Física e Astronomia}


% % Supervisor info
% \supervisor[mailto:example@fc.up.pt]{Dra. Armandina Maria Lima Lopes} % Supervisor name
% %\supervisorposition{Categoria} % Supervisor name position/Category (comment out to hide this field)
% \supervisoraffiliation[]{Faculdade de Ciências} % Supervisor university/faculty % Supervisor university/faculty
% %\supervisoraffiliation[]{Instituto de Engenharia de Sistemas e Computadores, Tecnologia e Ciência}


% % Cosupervisor info ----- Comment out if not needed
% \cosupervisor[mailto:example@fc.up.pt]{Prof. João Pedro Esteves de Araújo} % Cosupervisor name
% %\cosupervisorposition{Categoria} % Cosupervisor name position/Category (comment out to hide this field)
% \cosupervisoraffiliation[]{Faculdade de Ciências} % Supervisor university/faculty



% +++++++++++++++++++++++++++++++++++++++++++++++++++++++++++++++++++++++++++++++++
% 								Inserir nome do autor
% +++++++++++++++++++++++++++++++++++++++++++++++++++++++++++++++++++++++++++++++++
\author{Francisco Alves Miranda Valente Pereira}


% +++++++++++++++++++++++++++++++++++++++++++++++++++++++++++++++++++++++++++++++++
% 						Não sei bem o que é que este comando faz
% +++++++++++++++++++++++++++++++++++++++++++++++++++++++++++++++++++++++++++++++++
\makeindex	


% +++++++++++++++++++++++++++++++++++++++++++++++++++++++++++++++++++++++++++++++++
% 								Iniciar Documento
% +++++++++++++++++++++++++++++++++++++++++++++++++++++++++++++++++++++++++++++++++
\begin{document}
	
	
% +++++++++++++++++++++++++++++++++++++++++++++++++++++++++++++++++++++++++++++++++
% 				Este comando numera as páginas iniciais até ao primeiro
% 							capítulo em numeração romana
% +++++++++++++++++++++++++++++++++++++++++++++++++++++++++++++++++++++++++++++++++
\frontmatter


% +++++++++++++++++++++++++++++++++++++++++++++++++++++++++++++++++++++++++++++++++
% 					Front matter contempla as seguintes secções :
%
%		- Capa 						: front-matter/templeate.pdf pág. 1
%		- Página Presidente do Jurí : front-matter/templeate.pdf pág. 2
%		- Segunda capa   			: front-matter/coverface.tex
%		- Acknowledgment 			: front-matter/acknowledgments.tex
% 		- Resumo 		 			: front-matter/resumo.tex 
%		- Abstract 		 			: front-matter/abstract.tex
%		- Motivation ( opcional ? ) : front-matter/motivation.tex
%		- Table of Contents 		
%		- List of Figures
%		- List of Tables
% 		- List of Abbreviations
% +++++++++++++++++++++++++++++++++++++++++++++++++++++++++++++++++++++++++++++++++
% ++++++++++++++++++++++++++++++++++++++++++++++++++++++++++++++++++++++++++++++++++++++++++++++++++++
% 										Adicionar capa guardada 
%				Ficheiro guardado como templeate-cover.pdf no diretório front-matter
% ++++++++++++++++++++++++++++++++++++++++++++++++++++++++++++++++++++++++++++++++++++++++++++++++++++
\pagestyle{empty}
\includepdf[pages={1},pagecommand={},scale=1]{front-matter/template_fcup.pdf}

\pagestyle{empty}
\includepdf[pages={2},pagecommand={},scale=1]{front-matter/template_fcup.pdf}
\clearemptydoublepage



%\pagestyle{empty}
%\includepdf[pages={3},pagecommand={},scale=1,offset=0 -70 offy]{front-matter/templeate-cover-dec.pdf}
%\clearemptydoublepage



% ++++++++++++++++++++++++++++++++++++++++++++++++++++++++++++++++++++++++++++++++++++++++++++++++++++
% 									Adicionar página de rosto
% ++++++++++++++++++++++++++++++++++++++++++++++++++++++++++++++++++++++++++++++++++++++++++++++++++++
%%%%%%%%%%%%%%%%%%%%%%%%%%%%%%%%%%%%%%%%%%%%%%%%%%%%%%%%%%%%%%%%%%
%%%%%%%%%%%%%%%%%%%%% PÁGINA DE ROSTO    %%%%%%%%%%%%%%%%%%%%%%%%
%%%%%%%%%%%%%%%%%%%%%%%%%%%%%%%%%%%%%%%%%%%%%%%%%%%%%%%%%%%%%%%%%
%
%\newpage{
%        \thispagestyle{empty}%
%        
%%define the thickness of the contour on PhD Word        
%\contourlength{3pt}   
%
%%MSc word
%%\rput[B]{270}(112mm,-180mm){{\fontsize{215pt}{10cm}\selectfont \color{fcupcolor}M\contour{fcupcolor}{\color{white}{Sc}}}}
%\includegraphics[scale=0.97]{front-matter/msctitle.png}
%%%%%  %Figure on top of the page   
%%%%% \begin{figure}[hbt]
%%%%%     \vskip-15mm
%%%%% 	\hskip70mm
%%%%% 		\includegraphics[width=260pt]{front-matter/PAC_4detektor}
%%%%% \end{figure}
%%%%%     
%
%%%bold title     
%%\rput[tl]{0}(-22mm,-15mm){\parbox{120mm}{{\fontsize{45pt}{1em}\selectfont \textbf{\raggedleft{\mbox{Atomic scale properties of Pyroxenes: a combined experimental and Density Functional Theory study }}}}}}
%
%
%
%\rput[tl]{0}(-22mm,-1mm){
%\begin{minipage}[t]{200mm}
%  {\fontsize{35pt}{1em}\selectfont\textbf{\raggedleft{Atomic scale properties}}}\\
%  \bigskip{}\\
%  {\fontsize{35pt}{1em}\selectfont\textbf{\raggedleft{of Pyroxenes:}}}\\
%    \bigskip{}\\
%  {\fontsize{25pt}{1em}\selectfont\textbf{\raggedleft{A combined experimental}}}
%   \bigskip{}  \\
%  {\fontsize{25pt}{1em}\selectfont\textbf{\raggedleft{and}}}
%  \bigskip{}  \\
%  {\fontsize{25pt}{1em}\selectfont\textbf{\raggedleft{Density Functional Theory study}}}
% \end{minipage}
%}
%
%
%
%
%
%
%
%
%
%
%
%
%%author and stuff text
%   \rput[tl](-22mm,-110mm){\begin{tabular}[t]{@{}l@{}l@{}l@{}l@{}l@{}l@{}l@{}l@{}l@{}}{\fontsize{18pt}{18pt}\selectfont \color{blue}{António Neves Cesário}}\\
%   {\fontsize{12pt}{18pt}\selectfont Master's in Physics}\\
%   {\fontsize{10pt}{16pt}\selectfont \color{blue}{Department of Physics and Astronomy}}\\
%   {\fontsize{10pt}{16pt}\selectfont Faculty of Sciences}\\
%   {\fontsize{10pt}{16pt}\selectfont University of Porto}\\
%   {\fontsize{10pt}{10pt}\selectfont \the\year}\\
% \\
%   {\fontsize{10pt}{10pt}\selectfont \textbf{Supervisor}}\\
%   {\fontsize{10pt}{10pt}\selectfont \color{blue}{Dr. Armandina Maria Lima Lopes}, FCUP}\\
%  \\
%   {\fontsize{10pt}{10pt}\selectfont \textbf{Co-Supervisor}}\\
%   {\fontsize{10pt}{10pt}\selectfont \color{blue}{Prof. Jo\~{a}o Pedro Esteves de Araújo}, FCUP}\\
%   \end{tabular}}
%}
% 
%   
%%\newevenside
%   \clearemptydoublepage
%    \newpage
%
%
%
%%%%%%%%%%%%%%%%%%%%%%%%%%%%%%%%%%%%%%%%%%%%%%%%%%%%%%%%%%%%%%%%%
%%%%%%%%%%%%%%%%%%	EXAMINER PAGE  %%%%%%%%%%%%%%%%%%%%%%%%%%%%%
%%%%%%%%%%%%%%%%%%%%%%%%%%%%%%%%%%%%%%%%%%%%%%%%%%%%%%%%%%%%%%%%%
%
%\newpage{
%        \thispagestyle{empty}%
%     \hfill
%\begin{minipage}[b][250mm][b]{30mm}
%	\setstretch{1.0}
%	\ifdefined\extraaffilnolink
%		\includegraphics[width=52mm]{front-matter/fcup_clipped.png}
%		\\[2mm]
%	\fi
%	\ifdefined\otheraffilnolink
%		\includegraphics[width=52mm]{\otherlogo}
%		\\[2mm]
%	\fi
%	\includegraphics[width=52mm]{front-matter/fc_logo.png}
%
%	\setstretch{1.2}
%	{ \noindent\footnotesize Todas as correções determinadas \\
%		pelo júri, e só essas, foram efetuadas. \\
%		\\
%		O Presidente do Júri, \\
%		\\ 
%		\\
%		\\
%		Porto, \underline{\qquad\quad}/\underline{\quad\qquad}/\underline{\qquad\qquad} \\
%	}
%
%	\includegraphics[scale=0.97]{front-matter/msctitle.png}
%\end{minipage}
%
%}
%
%
%
%%\newevenside
%   \clearemptydoublepage
%    \newpage
%
%
%%%%%%%%%%%%%%%%%%
%%%%%%%%%%%%%%%%%%%

\newpage{


        \thispagestyle{empty}%
        \null\vskip1in%
        \begin{center}
                {\Large \bf Francisco Valente Pereira}\\
                \vskip0.5in%
                {\huge \bf \expandafter{\doublespacing Desenvolvimento de módulo de Análise de Dados para identificação de situações de fraude e corrupção em processos de contratação pública }}
        \end{center}
        
    	%$\text{Ca}_{3}(\text{Mn},\text{Ti})_{2}\text{O}_{7}$
        
        \vfill
        	\begin{figure}[h!]
        	\centering
  			\includegraphics[width=4cm]{front-matter/fc_logo}
			\end{figure}
			\vfill
%        \epsfxsize=4cm
%        \epsffile{Frontmatter/fc_logo.ps}}
        
        \begin{center}
                {\small \it \textbf{Orientador:} Prof. Dr. Sílvio Gama\\
                \textbf{Coorientador:} Prof. Dr. Margarida Brito}\\
        \end{center}
        
        %\vfill
        %\begin{center}
        %        {\small \it Thesis submitted to the Faculty of Sciences of the\\
        %        University of Porto in partial fulfilment of the requirements for\\
        %        the degree of Master in Physics}\\
        %\end{center}
        
        \vfill
        \begin{center}
                \expandafter{Departmento de Matemática\\
                Faculdade de Ciências da Universidade do Porto}\\
                2024
        \end{center}
        \vskip.5in
   \clearemptydoublepage
    %\newpage
    }


% %%%%%PAGINA COM LOGOS 
% %%%%%%%%%%%%%%%%%%%%%%%%%%%%
% %%%%%%%%%%%%%%%%%%%%%%%%%%%%

% \newpage{
%         \thispagestyle{empty}%


% \Large{Institutions involved in this thesis:}

% \vskip1cm
%         	\begin{figure}[h!]
%         	\centering
%   			\includegraphics[width=1\linewidth]{front-matter/institutionsWork}
% 			\end{figure}
	
			
% \vskip2cm




% \Large{Funding:}

% \vskip1cm
%         	\begin{figure}[h!]
%         	\centering
%   			\includegraphics[width=0.4\linewidth]{front-matter/logo_fct}
% 			\end{figure}


%         \vfill



%   \clearemptydoublepage
%     %\newpage
%     }




% ++++++++++++++++++++++++++++++++++++++++++++++++++++++++++++++++++++++++++++++++++++++++++++++++++++
% 										Adicionar citação
% ++++++++++++++++++++++++++++++++++++++++++++++++++++++++++++++++++++++++++++++++++++++++++++++++++++
% %%%%%% PÁGINA DE dedicatoria    %%%%%%%%%%%%%%%%%%%%%%%%
\newpage{%
        \thispagestyle{empty}%
        \null\vskip0.2in%
        \vfill
        \vfill
                {
                \vskip.8in
                \large \bf \expandafter{
             \begin{table}[!ht]
\begin{minipage}[b]{0.17\linewidth}
%\vspace{-20pt}
%\includegraphics[scale=0.65]{./FIGBOOK1/P4/P14_FIG02}
%\vspace{-20pt}
\end{minipage}
\hspace{0.20\linewidth}
\begin{minipage}[b]{0.75\textwidth}
%\raggedright
\begin{quote}
{\fontsize{9pt}{16pt}\selectfont 
    \textit{``When you think you know something: that is a most perfect barrier against learning.''}}
\end{quote}
\center
\textit{\textbf{Frank Herbert},\\ \textit{``God Emperor of Dune''}}
\end{minipage}
\end{table}
}
        \vskip.5in
            \clearemptydoublepage
}




% ++++++++++++++++++++++++++++++++++++++++++++++++++++++++++++++++++++++++++++++++++++++++++++++++++++
% 									Adicionar agradecimentos
% ++++++++++++++++++++++++++++++++++++++++++++++++++++++++++++++++++++++++++++++++++++++++++++++++++++
\markboth{Acknowledgements}{Acknowledgements}
\chapter*{Agradecimentos}
\label{Chp:Acknowledgements}
\addcontentsline{toc}{chapter}{Agradecimentos}
\vspace{-3em}



O presente relatório é o culminar de uma jornada científica, intelectual e emocional deveras desafiante. A completitude do meu percurso académico é, e irá ser, um marco singular da minha vida, que nunca esquecerei, tampouco daqueles com quem tive o prazer de o partilhar. \\ 



Agradeço ao meus orientadores, Professor Doutor Sílvio Gama e Professora Doutora Margarida Brito, a confiança e a disponibilidade para me orientarem ao longo deste projeto, todas as contribuições e sugestões dadas, tal como a motivação prestada.\\



Manifesto o meu agradecimento aos meus colegas na FORCERA, Rafael Arrais, João Pereira e Tiago Nunes, pela confiança e oportunidada de estágio proporcionada. Fiquei grato por toda a simpatia, compreensão, afabilidade, ensinamentos e troca de ideias e sugestões durante este trajeto. Agradeço, também, ao Ricardo pela ajuda e auxílio prestado.\\



Não posso deixar de agradecer a todos os meus amigos, tanto de Monção, como do Porto, que me acompanharam ao longo destes anos e me proporcionaram os melhores momentos que já vivi. \\



Por fim, gostaria de expressar a minha enorme gratidão para com os meus pais, a minha irmã, a minha sobrinha, o meu cunhado e todos os meus animais de estimação, os quais guardo, e sempre guardarei, com muito carinho no meu coração. 








% ++++++++++++++++++++++++++++++++++++++++++++++++++++++++++++++++++++++++++++++++++++++++++++++++++++
% 										Adicionar resumo
% ++++++++++++++++++++++++++++++++++++++++++++++++++++++++++++++++++++++++++++++++++++++++++++++++++++
\markboth{Resumo}{Resumo}
\documentclass{book}

\usepackage[portuguese]{babel}
\usepackage{hyperref}
\usepackage{outlines}
\setlength\parindent{0pt}

\hypersetup{
	colorlinks=true,
	linkcolor=blue,
	filecolor=magenta,      
	urlcolor=cyan,
	pdftitle={Overleaf Example},
	pdfpagemode=FullScreen,
}

\urlstyle{same}


\title{Notas Forcera}


\usepackage[top = 2.5cm, bottom = 2.5cm, left = 2.5cm, right = 2.5cm]{geometry} 

% Unfortunately, LaTeX has a hard time interpreting German Umlaute. The following two lines and packages should help. If it doesn't work for you please let me know.
\usepackage[T1]{fontenc}
\usepackage[utf8]{inputenc}

% The following two packages - multirow and booktabs - are needed to create nice looking tables.
\usepackage{multirow} % Multirow is for tables with multiple rows within one cell.
\usepackage{booktabs} % For even nicer tables.

% As we usually want to include some plots (.pdf files) we need a package for that.
\usepackage{graphicx} 

% The default setting of LaTeX is to indent new paragraphs. This is useful for articles. But not really nice for homework problem sets. The following command sets the indent to 0.
\usepackage{setspace}
\setlength{\parindent}{0in}

% Package to place figures where you want them.
\usepackage{float}

% The fancyhdr package let's us create nice headers.
\usepackage{fancyhdr}

\usepackage{amssymb}
\usepackage{amsmath}

\usepackage{pdfpages}
% ----------------------------------------------
% 			  Header (and Footer)
% ----------------------------------------------

% To make our document nice we want a header and number the pages in the footer.

\pagestyle{fancy} % With this command we can customize the header style.

\fancyhf{} % This makes sure we do not have other information in our header or footer.

%\lhead{\footnotesize Séries Temporais e Previsão : Quizz 1}% \lhead puts text in the top left corner. \footnotesize sets our font to a smaller size.

%\rhead works just like \lhead (you can also use \chead)
%\rhead{\footnotesize Francisco Valente Pereira} %<---- Fill in your lastnames.

% Similar commands work for the footer (\lfoot, \cfoot and \rfoot).
% We want to put our page number in the center.
\rfoot{\footnotesize \thepage} 

\begin{document}
	
	\section*{Como funciona a contratação pública em Portugal}
	\hrule
	\vspace{1cm}
	
	Para compreender a contratação pública em Portugal podemos ler o Código dos Contratos Públicos (CCP). 
	
	\begin{itemize}
		
		\item \href{https://www.base.gov.pt/Base4/pt/perguntas-frequentes/}{FAQ do site base-gov}
		
		\item \href{https://www.base.gov.pt/Base4/pt/documentacao/caracteristicas-dos-procedimentos/}{Características dos diferentes procedimentos}
		
		\item \href{https://www.pgdlisboa.pt/leis/lei_mostra_articulado.php?nid=2063&tabela=leis&so_miolo=}{Procuradoria Geral Distrital de Lisboa}
		
		\item \href{https://www.base.gov.pt/base4/media/ferbgqli/ccp-consolidado-impic-ap%C3%B3s-lei-30-2021.pdf}{Código dos Contratos Públicos pelo IMPIC possivelmente desatualizado}
		
		\href{https://www.base.gov.pt/Base4/pt/documentacao/caracteristicas-dos-procedimentos/}{Fluxogramas}
		
	\end{itemize}
	
	\hrule
	\vspace{1cm}
	
	O ato de adjudicar consiste em conferir o direito de algo a alguém, entregar algo ao maior licitante ou atribuir algo a alguém por concurso ou por ajuste. Este é um termo essencial na área de contratação pública, sendo esta constituída pelas entidades adjudicantes e entidades adjudicatárias. As entidades adjudicantes definidas no CCP são, essencialmente, as seguintes : \textit{Estado, Regiões Autónomas, Autarquias locais, Institutos públicos, Entidades Administrativas Independentes, Banco de Portugal, Fundações Públicas, Associações Pública}s\\
	
	A contratação pública consiste na celebração de contratos públicos entre entidades adjudicantes e entidades adjudicadas, sendo esta composta por atos e formalidades relativos à formação, conclusão e produção de uma plena eficácia jurídica de um contrato público. A eficácia jurídica - ao contrário da eficácia social - é um conceito teórico, segundo o qual uma norma definida de acordo com a lei se torna eficaz em termos jurídicos. \\
	
	Existem regras que devem ser cumpridas ao longo de todas as fases do processo de contratação pública. A primeira fase é a \textbf{fase preparatória} em que é feita a decisão de realizar um contrato e inclui uma fase preparatória do procedimento e uma fase instrutória que terminará no ato de ajudicação. A segunda fase é a \textbf{fase conclusiva} em que é concluído e celebrado o contrato. Existe também uma \textbf{fase complementar} que pode ser necessária na eventualidade do contrato público depender de atos posterioes à sua celebração tais como a aprovação, visto e publicidade. \\
	
	Existem diferentes tipos de procedimentos de contratação pública : \textit{ajuste direto - regime geral e simplificado-, consulta prévia, concurso público - normal e urgente, concurso limitado por prévia qualificação, procedimento de negociação, diálogo concorrencial, parceria para a inovação, disponibilização de bens móveis, serviços sociais e outros serviços específicos, concurso de conceção simplificado e concurso de ideias simplificado}.\\
	
	
	\newpage
	\subsection*{Ajuste Direto em Regime Geral}
	
	No \textbf{ajuste direto} a entidade ajudicante convida diretamente uma entidade adjudicatária, à sua escolha, a apresentar uma proposta. Contudo, este procedimento só é válido se cumprir um dos dois critérios.  Critério do Valor -  o valor do contrato for inferior a 20.000€ para aquisição ou alocação de bens móveis, inferior a 30.000€ para empreitadas de obras públicas e inferior a 50.000€ para outro tipo de contratos. Critérios materiais - não existe valor máximo imposto para o contrato mas o orgão competente para a deicsão de contratar tem de justificar de forma clara e objetiva que a situação em concreto reúne todos os pressupostos previstos em alguma das alíneas dos artigos 24º a 27º. 
	
	\begin{figure}[H]
		\centering
		\includegraphics[width=0.8\textwidth]{ajuste_direto_ccp.png}
		\caption{}
		\label{}
	\end{figure}
	
		
	\newpage
	
	\subsubsection{Ajuste Direto Simplificado}
	
	Um ajuste direto simplificado dispensa formalidades procedimentais. O contrato é celebrado / consumado quando o orgão competente para a decisão de contratar aprova a fatura / documento apresentado pela entidade convidada. Este procedimento só pode ser adotado para a formação de contratos de aquisição ou alocação de bens móveis ou de aquisição de serviços cujo preço contratual não seja superior a 5000€, ou no caso de empreitadas de obras públicas não seja superior a 10.000€.  O prazo de execução do contrato não pode ser superior a 3 anos a contar da data de ajdudicação, não pode ser prorrogado / prolongado e o preço contratual não pode ser objeto de qualquer revisão. 
	
	
	
	\begin{figure}[H]
		\centering
		\includegraphics[width=0.8\textwidth]{ajuste_direto_simplificado_ccp.png}
		\caption{}
		\label{}
	\end{figure}
	
	
	\newpage
	\subsection*{Consulta Prévia}
	
	Neste tipo de procedimento a entidade adjudicante convida diretamente, pelo menos, 3 entidades à sua escolha a apresentar uma proposta. Este pode ser aplicado para aquisição de bens móveis ou serviços com valor inferior a 75.000€, empreitadas de obras públicas com valor inferior a 150.000€  e outro tipo de contratos com valor inferior a 100.000€. 
	
	\begin{figure}[H]
		\centering
		\includegraphics[width=0.8\textwidth]{consulta_previa_ccp.png}
		\caption{}
		\label{}
	\end{figure}
	
	
	\newpage
	\subsection*{Concurso Público}
	
	Neste procedimento, o concurso é dado a conhecer através do Diário da República (e no Jornal Oficial da União Europeia quando o valor do contrato a celebrar for superior aos limiares comunitários).\\
	Não existe nenhuma fase prévia de qualificação dos concorrentes relativamente à capacidade técnica e/ou financeira. Este procedimento pode ser adotado sempre que a entidade adjudicante assim o decidir. 
	
	\begin{figure}[H]
		\centering
		\includegraphics[width=0.8\textwidth]{concurso-publico_ccp.png}
		\caption{}
		\label{}
	\end{figure}
	
	\newpage
	\subsection*{Concurso Público Urgente}
	
	Neste caso, o concurso é publicado no Diário da República e o prazo de apresentação de propostas pode ir de 24h até 72h consoante o tipo de empreitada. 
	
	\begin{figure}[H]
		\centering
		\includegraphics[width=0.8\textwidth]{cp_urgente_ccp.png}
		\caption{}
		\label{}
	\end{figure}
	
	
	\newpage
	\subsection*{Concurso Limitado por Prévia Qualificação}

	Este procedimento é realizado quando o valor do contrato a celebrar for superior aos limiares Europeus. É publicado no Diário da República e no Jornal Oficial da União Europeia. Existem 2 fases neste procedimento, sendo a primeira caracterizada pela apresentação das candidaturas e qualificação dos candidatos e a segunda pela apresnetação e análise das propostas e adjudicação. \\
	
	
	\subsection*{Procedimento de Negociação}
	É semelhante ao Concurso Limitado por Prévia Qualificação. Contudo, na segunda fase, após os concorrentes terem sido qualificadas, existe a possibilidade de melhorar a proposta numa fase de negociação. \\
	
	
	\subsection*{Diálogo Concorrencial}
	Este procedimento é utilizado para situações em que a entidade adjudicante identificou a sua necessidade mas não sabe como satisfazer. Antes da fase de apresentação de proposta, existe uma fase de apresentação de soluções e diálogo já com as entidades qualificadas. \\
	
	
	\subsection*{Parceria para Inovação}
	Destina-se à realização de atividades de investigação e desenvolvimento de bens, serviços ou obras inovadoras. Tem como objetivo a aquisição destes bens desde que se cumpram os níveis de desempenho de preços máximos previamente combinados. Acontece quando um entidade adjudicante pretende adquirir um bem/serviço/obra pública com determinadas características que não se encontram no mercado. 
	
	
	
	
	
\end{document}




% ++++++++++++++++++++++++++++++++++++++++++++++++++++++++++++++++++++++++++++++++++++++++++++++++++++
% 									 	Adicionar abstrato
% ++++++++++++++++++++++++++++++++++++++++++++++++++++++++++++++++++++++++++++++++++++++++++++++++++++
\markboth{Abstract}{Abstract}
\chapter*{Abstract}
\label{chap:abstract}
\addcontentsline{toc}{chapter}{Abstract}

%In this work, Ca$_{3}$Mn$_{2-x}$Ti$_{x}$O$_{7}$ $n=2$ Ruddlesden-Popper (RP) systems were synthesized and characterized in detail at the macroscopic and local scales. In particular, the $x=0.25$ composition (CMTO) was examined with X-Ray Powder Diffraction (XRPD), Raman Spectroscopy, and Perturbed Angular Correlation (PAC). Density Functional Theory (DFT) calculations in the $x=0$ undoped system (CMO) were also carried.
%Such systems belong to the $n=2$ family of Ruddlesden-Popper layered perovskites, which have attracted much interest due to the presence of Hybrid Improper Ferroelectricity. 

%The samples were synthesised via a common solid-state reaction method and the phase purity checked by XRPD measurements at 300K.

%With Raman ($93-653$K) and PAC ($74-1224$K) spectroscopies, we found a first order structural phase transition from what appears to be the ferroelectric $A2_{1}am$ (s.g. 36) phase to an intermediate and paraelectric $Acaa$ (s.g. 68) symmetry within a $300-550$K temperature range, where both phases coexist. Thus, in that case, when compared to CMO, we report an increase in the percentage of the ferroelectric phase at room temperature due to the Ti$^{4+}$ concentration. However, our results point to the possibility of the ground state not corresponding to the $A2_{1}am$ symmetry, but to one with inequivalent A-sites at the rock-salt of the RP system. Such statement will be examined in future works with DFT considerations. PAC spectroscopy also revealed a second order structural phase transition from the intermediate orthorhombic $Acaa$ (s.g. 68) to the tetragonal $I4/mmm$ (s.g. 139) at 1150K, as was previously reported for CMO.

%The Ca$_{3}$Mn$_{2}$O$_{7}-A2_{1}am$ system was studied with DFT. We determined a Hubbard-$U$ correction by first principles $(U=6\text{ eV})$ to properly describe the Mn $3d-$orbitals, and then compared the structural and electronic properties with $U=0$ and with a value commonly found in the literature $(U=3.9 \text{ eV})$. The cell parameters and bulk moduli were calculated for the mentioned $U$ values, as well each system's electronic properties, such as the projected density-of-states (PDOS), Bader charges, and the Electric Field Gradient at multiple lattice sites.




% ++++++++++++++++++++++++++++++++++++++++++++++++++++++++++++++++++++++++++++++++++++++++++++++++++++
% 									   Adicionar motivações
% ++++++++++++++++++++++++++++++++++++++++++++++++++++++++++++++++++++++++++++++++++++++++++++++++++++
\markboth{Motivation}{Motivation}
\chapter*{Motivation}
\label{ch:motivation}
\addcontentsline{toc}{chapter}{Motivation}

%Hybrid Improper Ferroelectric (HIF) systems constitute an encouraging set of materials in the search for room temperature ferroelectrics and multiferroics\cite{benedekHybridImproperFerroelectricity2011}, the latter being remarkably rare and sought after\cite{hillWhyAreThere2000}, since the combination of ferroelectricity and ferromagnetism in the same phase may lead to incredible possibilities and practical breakthroughs, such as the long awaited electric field controlled magnetic data storages\cite{spaldinRenaissanceMagnetoelectricMultiferroics2005}.

%Systems that present HIF offer a wide range of applications as magneto-electrics or ferroelectric random-access memories (FRAMs)\cite{panPerovskiteMaterialsSynthesis2016}, with recent studies aiming to fine tune their properties, such as polarizability \cite{sennNegativeThermalExpansion2015}. Ruddlesden-Popper (RP) perovskites, in particular $n=2$ Ca$_{3}(\text{Mn,Ti})_{2}$O$_{7}$ systems, have been very promising in this regard, due to their naturally layered structure and the presence of Mn ions, which allow for an unconventional condensation of a multiferroic phase below 110K\cite{benedekWhyAreThere2013,gaoInterrelationDomainStructures2017}. This is still far from room temperature, nothing like promised above. However, the origin of these system's functional properties and, most notably, the structural phase transitions they take from the high symmetry to the ground state, have been the subject of much controversy\cite{sennNegativeThermalExpansion2015,glamazdaSoftTiltRotational2018,rocha-rodriguesCa3Mn2O7StructuralPath2020}. Hence, a good understanding of the mechanisms behind structural phase transitions in these and other systems presenting HIF may pave the way for important discoveries in the quest for room temperature multiferroics. 

%That is precisely the context in which we developed this work. In addition, we intend to explore and systematize a deeper connection between the interpretation of PAC experiments with first principles calculations of Electric Field Gradients (EFGs) via DFT\cite{petrilliElectricfieldgradientCalculationsUsing1998}, aiming to ally two very powerful techniques in the characterization of solid state systems.



% ++++++++++++++++++++++++++++++++++++++++++++++++++++++++++++++++++++++++++++++++++++++++++++++++++++
% 									Adicionar tabela de conteúdos
% ++++++++++++++++++++++++++++++++++++++++++++++++++++++++++++++++++++++++++++++++++++++++++++++++++++
\renewcommand{\contentsname}{Table of Contents}
\tableofcontents
\addcontentsline{toc}{chapter}{Table of Contents}



% ++++++++++++++++++++++++++++++++++++++++++++++++++++++++++++++++++++++++++++++++++++++++++++++++++++
% 									Adicionar lista de figuras
% ++++++++++++++++++++++++++++++++++++++++++++++++++++++++++++++++++++++++++++++++++++++++++++++++++++
\clearemptydoublepage
%\renewcommand\listfigurename{List of Figures}
\listoffigures
\addcontentsline{toc}{chapter}{List of Figures}



% ++++++++++++++++++++++++++++++++++++++++++++++++++++++++++++++++++++++++++++++++++++++++++++++++++++
% 									Adicionar lista de tabelas
% ++++++++++++++++++++++++++++++++++++++++++++++++++++++++++++++++++++++++++++++++++++++++++++++++++++

\clearemptydoublepage
% \renewcommand\listtablename{List of Tables}
\listoftables
\addcontentsline{toc}{chapter}{List of Tables}
\clearemptydoublepage



% ++++++++++++++++++++++++++++++++++++++++++++++++++++++++++++++++++++++++++++++++++++++++++++++++++++
% 								  Adicionar lista de abreviaturas
% ++++++++++++++++++++++++++++++++++++++++++++++++++++++++++++++++++++++++++++++++++++++++++++++++++++

\markboth{List of Abreviations}{List of Abreviations}
\chapter*{Lista de Abreviaturas}
\label{chap:abbreviations}
\addcontentsline{toc}{chapter}{Lista de Abreviaturas}

\begin{multicols}{2}[]
	\begin{scriptsize}
        	\begin{acronym}
        		    \acro{cpv}[CPV]{Common Procurement Vocabulary}
        		    \acro{fcup}[FCUP]{Faculdade Ciências da Universidade do Porto}
                    \acro{ocds}[OCDS]{Open Contracting Data Standard}
                    		\end{acronym}
	\end{scriptsize}
\end{multicols}
\clearemptydoublepage



% +++++++++++++++++++++++++++++++++++++++++++++++++++++++++++++++++++++++++++++++++
% 						Não sei bem o que é este também
% +++++++++++++++++++++++++++++++++++++++++++++++++++++++++++++++++++++++++++++++++
\pagestyle{fancy}

% +++++++++++++++++++++++++++++++++++++++++++++++++++++++++++++++++++++++++++++++++
%						Numerar páginas no sistema decimal 
% +++++++++++++++++++++++++++++++++++++++++++++++++++++++++++++++++++++++++++++++++
\mainmatter




% +++++++++++++++++++++++++++++++++++++++++++++++++++++++++++++++++++++++++++++++++
% 							  Dar input aos capítulos
%
%	Para cada capítulo é criado um diretório dentro do diretório chapters. Dentro 
%	desse diretório pode existir um ou mais ficheiros .tex. Se o capítulo for 
%   muito grande é convenienta dividi-lo em vários ficheiros .tex
% +++++++++++++++++++++++++++++++++++++++++++++++++++++++++++++++++++++++++++++++++

% +++++++++++++++++++++++++++++++++++++++++++++++++++++++++++++++++++++++++++++++++
% 	Exemplo : 
%
%		\chapter{Capítulo 1}\label{sec:chapter1}
%		\input{chapters/ch1.tex}
%
%		\chapter{Capítulo 2}\label{sec:chapter2}
%		\input{chapters/ch2-parte1.tex}
%		\input{chapters/ch2-parte2.tex}
%		\input{chapters/ch2-parte3.tex}
%
%		\chapter{Conclusions and Future Work}
%		\section{Conclusions}
	In this work, we synthesized Ca$_{3}$Mn$_{2-x}$Ti$_{x}$O$_{7}$ $n=2$ Ruddlesden-Popper samples with various compositions $(x=0.1,\,0.25,\,0.3)$ via a common solid-state reaction method and studied its structural phase transitions and properties through different methods: X-Ray Powder Diffraction (XRPD), Raman Spectroscopy, Perturbed Angular Correlation (PAC), and Density Functional Theory (DFT) simulations, with special focus on the $x=0.25$ (C$_{2}$MTO) system.
	
	After the last annealing of the synthesis process, Rietveld refinements on the XRPD data were performed to check the purity of the formed $n=2$ phases. We observed an approximate $12\%$ presence of Ca$_{2}$Mn$_{0.875}$Ti$_{0.125}$O$_{4}$ ($n=1$ C$_{1}$MTO) in C$_{2}$MTO, however with $a-$ and $b-$lattice\-parameters considerably lower (by $\approx 0.6\AA$) of what can be found in the literature for the undoped Ca$_{2}$MnO$_{4}$ system.
	
	Raman and PAC spectroscopies, respectively in the $93-653$K and $74-1224$K temperature ranges, revealed a first order structural phase transition from what appears to be a the ferroelectric $A2_{1}am$ (s.g. 36) phase to an intermediate and paraelectric $Acaa$ (s.g. 68) symmetry within a $300-550$K temperature range, where both structural phases coexist. Thus, when comparing CMTO to the undoped Ca$_{3}$Mn$_{2}$O$_{7}$ (C$_{2}$MO), we report an increase in the percentage of the ferroelectric phase at room temperature by introducing the $x=0.25$ Ti$^{4+}$ doping. However, our results point to the possibility of the ground state not corresponding to the $A2_{1}am$ symmetry, but to one with inequivalent A-sites at the rock-salt of the RP system. PAC spectroscopy also revealed a second order structural phase transition from the intermediate orthorhombic $Acaa$ (s.g. 68) to the tetragonal $I4/mmm$ (s.g. 139) at 1150K.
	
	We came across an unexpected result at temperatures below 300K, where we clearly detected the probing of a second local environment, albeit without a clear physical origin. Interestingly, PAC measurements did not found evidence of the C$_{1}$MTO phase observed from XRPD refinements, and we believe this suggests that the $A2_{1}am$ space group might not correspond to the ground state symmetry for every composition of Ca$_{3}$Mn$_{2-x}$Ti$_{x}$O$_{7}$ mixed B-site systems. Phonon calculations on these doped systems in the $A2_{1}am$ symmetry could reveal if there are any lattice instabilities, shedding a light on whether the $A2_{1}am$ still corresponds to the ground state when Ti is added. 
	
	The C$_{2}$MO system was studied with DFT in it's ground state $A2_{1}am$ symmetry. We determined a Hubbard-$U$ correction by first principles $(U=6\text{ eV})$ to properly describe the Mn$-3d$ states, and then performed a series of structural and electronic properties calculations with our \textit{ab-initio} value and with $U=0$ and $U=3.9 \text{ eV}$, a value commonly found in the literature. We determined the cell parameters and bulk moduli for the mentioned $U$ values as well as multiple electronic properties, such as the projected density-of-states (PDOS), Bader charges, and the Electric Field Gradient at multiple sites. When we manage to define and adjust the most adequate parameters to compute the EFGs for this system, we intend to probe the Ti$^{4+}$ doped Ca$_{3}$Mn$_{2-x}$Ti$_{x}$O$_{7}$ structures, and analyse how the EFGs vary as a function of doping concentration.

\section{Future Work}
	In the future, we intend to deepen the discussion at several stages of this system's characterization. Regarding the observation of a C$_{1}$MTO phase impurity in XRPD experiments, and the underestimated lattice parameters, we intend to pursue a new refinement considering also the intermediate $Acaa$ phase. To complement this measurements, XAS and XPS spectroscopies would help us to confirm the oxidation states of Mn and Ti in the sample, to account for some deviations in the XRPD patterns. Additionally, neutron diffraction would confirm the structural and magnetic structural properties of these samples.
	
	More complex Raman Spectroscopy setups could be pursued to study the evolution of the softer modes that drive the $A2_1am-Acaa$ structural phase transition, and DFT calculations could assign the peak positions for each mode in the doped C$_{2}$MTO structure.
	
	The interpretation PAC results below 300K is still open to discussion, and we'll be pursing other methods to clarify them. Doped CMTO DFT calculations of the EFG parameters is a work in progress and the study of other doped systems is also scheduled, to later compare with undoped systems. In fact, $n=1$ Ca$_{2}$(Mn,Ti)O$_{4}$ RP compounds have already been measured by PAC and will be analysed in the future. Future phonon calculations on these B-site mixed systems may become a key point in understanding these systems.
%
% +++++++++++++++++++++++++++++++++++++++++++++++++++++++++++++++++++++++++++++++++
\chapter{Introdução}
\section{Estágio}

INSERIR TEXTO FORCERA

\section{Contratação Pública em Portugal}

A Contratação Pública em Portugal pode ser classificada de duas formas : aberta e fechada. De modo a garantir a eficácia jurídica, é fundamental conhecer o Código dos Contratos Públicos (CCP). As regras presentes no CCP dizem respeito aos contratos celebrados entre uma entidade adjudicante pública e uma entidade adjudicatária.

%sendo esta composta por atos e formalidades relativos à formação, conclusão e produção de uma plena eficácia jurídica de um contrato público. A eficácia jurídica - ao contrário da eficácia social - é um conceito teórico, segundo o qual uma norma definida de acordo com a lei se torna eficaz em termos jurídicos. \\


O ato de adjudicar consiste em conferir o direito de algo a alguém, entregar algo ao maior licitante ou atribuir algo a alguém por concurso ou por ajuste. 
%Este é um termo essencial na área de contratação pública, sendo esta constituída pelas entidades adjudicantes e entidades adjudicatárias. 
As entidades adjudicantes definidas no CCP são as seguintes : {Estado, Regiões Autónomas, Autarquias locais, Institutos públicos, Entidades Administrativas Independentes, Banco de Portugal, Fundações Públicas, Associações Públicas. ( COMPLETAR )\\


Existem regras que devem ser cumpridas ao longo de todas as fases do processo de contratação pública. A primeira fase é a \textbf{fase preparatória} em que é feita a decisão de realizar um contrato e inclui uma fase preparatória do procedimento e uma fase instrutória que terminará no ato de ajudicação. A segunda fase é a \textbf{fase conclusiva} em que é concluído e celebrado o contrato. Existe também uma \textbf{fase complementar} que pode ser necessária na eventualidade do contrato público depender de atos posterioes à sua celebração tais como a aprovação, visto e publicidade. \\

Existem diferentes tipos de procedimentos de contratação pública : \textit{ajuste direto - regime geral e simplificado-, consulta prévia, concurso público - normal e urgente, concurso limitado por prévia qualificação, procedimento de negociação, diálogo concorrencial, parceria para a inovação, disponibilização de bens móveis, serviços sociais e outros serviços específicos, concurso de conceção simplificado e concurso de ideias simplificado}.\\



%\chapter{Fraude e Corrupção na Contratação Pública}
%\input{chapters/fraude.tex}
%
%
%
%\chapter{Contratação Pública em Portugal}
%%A Contratação Pública em Portugal pode ser classificada de duas formas: aberta e fechada. As regras presentes no Código dos Contratos Públicos (CCP) dizem respeito aos contratos públicos celebrados entre uma entidade adjudicante pública e uma entidade adjudicatária.

%sendo esta composta por atos e formalidades relativos à formação, conclusão e produção de uma plena eficácia jurídica de um contrato público. A eficácia jurídica - ao contrário da eficácia social - é um conceito teórico, segundo o qual uma norma definida de acordo com a lei se torna eficaz em termos jurídicos. \\

%O ato de adjudicar consiste em conferir o direito de algo a alguém, conceder algo ao maior licitante ou atribuir algo a alguém por concurso ou por ajuste. 
%Este é um termo essencial na área de contratação pública, sendo esta constituída pelas entidades adjudicantes e entidades adjudicatárias. 

%O CCP é aplicado a entidades adjudicantes públicas, tais como o Estado, Regiões Autónomas, Autarquias locais, Institutos públicos, Entidades Administrativas Independentes, Banco de Portugal, Fundações Públicas, Associações Públicas, Associações de que façam parte uma ou várias pessoas coletivas referidas anteriormente e que sejam maioritariamente financiadas por estas. Além destas, são consideradas entidades adjudicantes organismos de direito público, pessoas coletivas e associações \footnote{nos termos do artigo 2.º n.º 2, alíneas a), b) e d)}. São consideradas, também, entidades adjudicantes organismos com atuação nos setores especiais da água, energia, tranposrtes e serviços postais \footnote{artigo 7.º n.º 1.º}. Existe, também, a possibilidade de aplicar o CCP a entidades não adjudicantes que pretendem celebrar determinados contratos de empreitadas de obras públicas ou de serviços associados a obras \footnote{artigo 275.º}.


% Associações de que façam parte uma ou várias pessoas coletivas referidas anteriormente, desde que sejam maioritariamente financiadas por estas, estejam sujeitas ao seu controle de gestão ou tenham um órgão de administração, de direção ou de fiscalização cuja maioria dos titulares seja, direta ou indiretamente, designada pelas mesmas

% São ainda entidades adjudicantes organismos de direito público, pessoas coletivas e associações, independentemente da sua natureza pública ou privada, nos termos do artigo 2.º n.º 2, alíneas a), b) e d).

% Para além das entidades adjudicantes referidas no artigo 2º, são também entidades adjudicantes as referidas no artigo 7.º n.º 1.º, concretamente as pessoas coletivas que realizam atividades nos seguintes sectores especiais da água, energia, transportes e serviços postais.

% O CCP aplica-se ainda a entidades que não sendo adjudicantes, se encontrem nas situações previstas no artigo 275.º, ou seja, entidades que pretendam celebrar determinados contratos de empreitadas de obras públicas ou de serviços associados a obras, desde que estes contratos sejam subsidiados diretamente em mais de 50\% do respetivo preço contratual por entidades adjudicantes, sempre que o preço contratual for igual ou superior aos limiares comunitários.


%Existem duas fases principais no processo de contratação pública. 
%A primeira fase é a \textbf{fase preparatória} em que é feita a decisão de realizar um contrato e inclui uma fase preparatória do procedimento e uma fase instrutória que terminará no ato de ajudicação. A segunda fase é a \textbf{fase conclusiva} em que é concluído e celebrado o contrato. Existe também uma \textbf{fase complementar} que pode ser necessária na eventualidade do contrato público depender de atos posterioes à sua celebração tais como a aprovação, visto e publicidade. \\

\section{Código dos Contratos Públicos}

O Código dos Contratos Públicos (CCP) é o documento que estabelece um alinhamento com as diretivas comunitárias,  estabelecidas pelo Parlamento Europeu \cite{ue_dire}, definindo um conjunto homogéneo de normas relativas aos processos pré-contratuais e uma nova sistematização e uniformização de regimes substantivos dos contratos administrativos \cite{guia_poise}. Este documento encontra-se dividido em duas grandes categorias: \textbf{disciplina aplicável à contratação pública} e \textbf{regime substantivo dos contratos públicos}, correspondentes às Partes II e III do documento, respetivamente. O CCP pauta-se por um conjunto de objetivos gerais tal como ilustrado na Figura \ref{fig:ccpgoals}.

\begin{figure}[H]
	\centering
	\includegraphics[width=0.5\textwidth]{imagens/ccp_objetivos.png}
	\caption{Objetivos gerais do Código dos Contratos Públicos.}
	\label{fig:ccpgoals}
\end{figure}


A simplificação da tramitação procedimental pré-contratual é feita através da implementação de novas tecnologias de informação/meios eletrónicos, incentivando a desburocratização, a desmaterialização, criando sistemas alternativos à utilização de papel e a simplificação da tramitação contratual.


Ao longo do processo contratual são definidos vários requisitos a cumprir:

\begin{my_enumerate}
	\item Qualificação dos candidatos;
	\item Métodos de avaliação;
	\item Valorização social e ambiental.
\end{my_enumerate}

Relativamente ao \textbf{nível de qualificação dos candidatos}, para determinadas tipologias de contrato, é exigido que os mesmos demonstrem possuir tanto capacidade técnica como financeira, a fim de completar o objeto contratual. 
É imperativo que os \textbf{métodos de avaliação de propostas}, componente crucial na formação e celebração de contratos públicos, sejam devidamente enunciados e publicitados, por forma a que as entidades concorrentes desenvolvam uma estratégia/proposta eficiente e a entidade adjudicante selecione, à luz desses mesmos critérios, a proposta mais vantajosa a nível económico, na ótica do interesse prosseguido. 
Além do mais, o CCP procura, de forma cabal, garantir que a enunciação e publicitação dos critérios de adjudicação e respetivos coeficientes de ponderação, se faça em conformidade com os princípos da igualdade, da concorrência, da imparcialidade, da proporcionalidade, da transparência, da publicidade e da boa fé.
É, também, necessário que o objeto do contrato a celebrar \textbf{reflita e valorize preocupações de cariz social e ambiental}\cite{ccp}\cite{guia_poise}. 

O incumprimento da legislação  que regula os procedimentos contratuais, pode implicar correções financeiras, cujas taxas são definidas em função da natureza e gravidade das irregularidades detetadas e, consequentemente, a uma perda de financiamento. As irregularidades que são detetadas com maior frequência encontram-se disponíveis para consulta na Tabela de Correções Financeiras COCOF \cite{corrections}\cite{cocoftab}.





\subsection{Entidades}

O artigo 2.º do CCP categoriza as entidades adjudicantes em dois organismos: organismos pertencentes ao setor público administrativo tradicional e organismos de direito público.


\begin{table}[h!]
	
	\centering
	\setlength{\tabcolsep}{15pt}
	\setlength\cellspacetoplimit{0.5cm} 
	\setlength\cellspacebottomlimit{0.5cm} 
	\renewcommand{\arraystretch}{1.5}

	\resizebox{\textwidth}{!}{
        \begin{tabular}{p{0.1\textwidth} p{0.1\textwidth} p{0.1\textwidth} p{0.25\textwidth} m{0.5\textwidth}} % Use p{} column type for vertical padding

			\hline
			\multicolumn{5}{Sc}{\textbf{Entidades Adjudicantes}} \\
			\hline
			\rowcolor[HTML]{EFEFEF} 
			\multicolumn{4}{Sc|}{\cellcolor[HTML]{EFEFEF}\textbf{Organismos pertencentes ao setor público administrativo tradicional}} & \textbf{Organismos de direito público} \\
			\hline
			Estado & Regiões Autónomas & Autarquias locais & Institutos públicos & \multirow{2}{=}{\centering Quaisquer pessoas coletivas que, independentemente da sua natureza pública ou privada, reúnam os requisitos presentes no n.º2 do artigo 2.º do CCP} \\ 
			Banco de Portugal & Fundações Públicas & Associações públicas & Entidades administrativas independentes & \\
			\hline
		\end{tabular}
	}
	
	\caption{Categorização das Entidades Adjudicantes.}
	\label{table:1}
\end{table}



Acresce às anteriormente menciondas outro tipo de entidades adjudicantes abrangidas pelo CCP pertencentes aos setores especiais da água, energia, transportes e serviços postais. 

A denominação  de \textit{entidade adjudicante} é válida, apenas, durante a fase pré-contratual. Após a celebração do contrato passa a denominar-se \textit{contraente público}. 



\subsection{Procedimentos para formação de contratos}
%\subsubsection{Equadramento Legal}

\subsubsection{Tipos de Procedimento e Contrato}
Os procedimentos pré-contratuais definidos no n.º1 do artigo 16.º do CCP, a serem adotados pelas entidades adjudicantes, encontram-se discriminados na Tabela \ref{table:2}.

\begin{table}[ht]
	\centering
	\renewcommand{\arraystretch}{1.15}
	\setlength{\tabcolsep}{15pt}
	%\resizebox{\textwidth}{!}{%
		\begin{tabular}{lc}
			\toprule
			\multicolumn{1}{c}{\textbf{Tipo de Procedimento}} & \textbf{Artigo do CCP} \\ 
			\midrule
			\rowcolor[HTML]{EFEFEF} 
			Ajuste Direto                                     & 112.º a 129.º          \\ 
			Consulta Prévia                                   & 112.º a 127.º          \\
			\rowcolor[HTML]{EFEFEF} 
			Concurso Público                                  & 130.º a 161.º          \\
			Concurso Limitado por Prévia Qualificação         & 162.º a 192.º          \\
			\rowcolor[HTML]{EFEFEF} 
			Diálogo Concorrencial                             & 204.º a 218.º          \\
			Procedimento de Negociação                        & 193.º a 203.º          \\
			\rowcolor[HTML]{EFEFEF} 
			Parceria para a Inovação                          & 218.º-A a 218.º-D      \\
			Acordos-quadro                                    & 251.º a 259.º          \\                             
			\bottomrule
		\end{tabular}
	%}
	\caption{Tipos de procedimento a adotar na fase pré-contratual.}
	\label{table:2}
\end{table}

Impõe-se, desde logo, uma descrição sumária de cada um dos procedimentos.

\begin{my_enumerate}
	
	%\item O \textbf{ajuste direto} consiste no convite direto, por parte da entidade adjudicante, a um operador económico à sua escolha, a fim de apresentar uma proposta para um determinado objeto contratual\cite{ajustedir}. Existem dois regimes para esta tipologia de procedimento: Ajuste Direto em Regime Geral e Ajuste Direto Simplificado. As diferenças entre estes dois regimes, que podem ser observadas na Tabela \ref{tab:4}, prendem-se com os valores contratuais máximos permitidos por lei.
	
	\item O \textbf{ajuste direto} corresponde ao procedimento de contratação pública em que a entidade adjudicante convida diretamente uma entidade à sua escolha para que esta apresente uma proposta\cite{ajustedir}. 
	
	Existem dois regimes para esta tipologia de procedimento: Ajuste Direto em Regime Geral e Ajuste Direto Simplificado. As diferenças entre estes dois regimes, consideradas na Tabela \ref{tab:4}, prendem-se com os valores contratuais máximos permitidos por lei.
	
	%\item Na \textbf{consulta prévia}, a entidade adjudicante convida diretamente, pelo menos, três operadores económicos à sua escolha a apresentar uma proposta. Os aspetos referentes ao contrato a celebrar podem ser negociados diretamente entre a entidade adjudicante e os operadores convidados \cite{consultaprev}. 
	
	\item A \textbf{consulta prévia} corresponde ao procedimento de contratação pública em que a entidade adjudicante convida diretamente, pelo menos, três entidades à sua escolha a apresentar proposta, podendo com elas negociar os aspetos da execução do contrato a celebrar\cite{consultaprev}. 
	
	
	%\item O \textbf{concurso público} pode ser adotado sempre que uma determinada entidade adjudicante o decidir. Não existe nenhuma fase prévia de qualificação dos concorrentes relativamente à capacidade técnico-financeira \cite{concursopub}. 
	
	\item O \textbf{concurso público} corresponde ao procedimento de contratação pública que é objeto de um anúncio num jornal oficial (Diário da República e/ou Jornal Oficial da União Europeia) no qual, qualquer entidade, que preencha os requisitos de participação, pode apresentar uma proposta\cite{concursopub}. 
	
	%\item O \textbf{concurso limitado por prévia qualificação} é adotado quando o valor do contrato a celebrar for superior aos limiares Europeus. Neste caso, o concurso é dado a conhecer, não só através do Diário da República, como nos itens anteriores, mas também no Jornal Oficial da União Europeia\cite{previaqual}. O \textbf{procedimento de negociação} partilha algumas características com este procedimento. 
	
	\item O \textbf{concurso limitado por prévia qualificação} corresponde ao procedimento de contratação pública que, sendo objeto de um anúncio num jornal oficial (Diário da República e/ou Jornal Oficial da União Europeia), se desdobra em 2 (duas) fases essenciais (qualificação e adjudicação), através das quais se constata se os candidatos preenchem os requisitos mínimos de capacidade definidos pela entidade adjudicante, sendo que os candidatos admitidos poderão, na segunda fase (adjudicação), apresentar uma proposta\cite{previaqual}.
	
	\item O \textbf{diálogo concorrencial} é utilizado nas situações em que a entidade adjudicante identifica uma necessidade e não sabe como satisfazer \cite{dialogoconc}. 
	
	\item A \textbf{parceria para a inovação} destina-se à realização de atividades de investigação e desenvolvimente de bens, serviços ou obras inovadoras. Tem como objetivo a aquisição destes bens desde que se cumpram os níveis de desempenho de preços máximos
	previamente combinados. Acontece quando um entidade adjudicante pretende adquirir um bem/serviço/obra
	pública com determinadas características que não se encontram no mercado. 
	
	%\item O \textbf{acordo quadro} é a celebração de um contrato entre uma ou várias entidades adjudicantes e uma ou mais entidades \cite{acordoquadro}. 
	
	\item O \textbf{acordo quadro} é a celebração de um contrato entre uma ou várias entidades adjudicantes e uma ou mais entidades \cite{acordoquadro}. 
	
	
\end{my_enumerate}

%Em ambos os casos existem limitações relativamente aos operadores económicos que podem ser convidados. (artigos 19.o, al. d), e 20.o, n.o 1, al. d), 19.o, al. c), e 20.o, n.o 1, al. c)). Não obstante, as limitações referidas não se aplicam em contratos ao abrigo de critérios materiais. 
%A celebração de consultas prévias e ajustes diretos deve ser publicitada no Portal dos Contratos Públicos pela entidade ajdudicante no prazo máximo de 20 dias a contar da data de celebração do contrato, a fim de comprovar a eficácia do respetivo contrato. 



%\textit{O CCP revê em alta ( o que é que isto significa? ) os limites relativos ao valor do contrato em função do procedimento pré-contratual adoptado. Para efeitos da determinação do valor do contrato, foi estabelecido que a escolha do procedimento condiciona o valor do contrato a celebrar. Por sua vez, o valor do contrato a celebrar pode ser entendido como o valor máximo do benefício económico obtido pelo adjudicatário com a execução de todas as prestações que constituem o objeto contratual.} \\



O n.º2 do artigo 16.º do CCP estabelece a tipologia do contrato a realizar, independentemente da sua natureza ou designação. As tipologias, \textit{empreitada de obras públicas} e \textit{concessão de obras públicas}\footnote{O n.º 2 do artigo 343.º do CCP define como obra pública qualquer trabalho de construção, reconstrução, ampliação, alteração ou adaptação, conservação, restauro, reparação, reabilitação, beneficiação e demolição de bens imóveis executados por conta de um contraente público.}, foram agregadas numa única categoria, doravante denominada \textit{Obras}. 
No que concerne às restantes tipologias, foram incluídas numa única categoria, doravante denominada \textit{Bens e Serviços}.

O artigo 135.º do supracitado código define o prazo mínimo, em dias, que as entidades concorrentes dispõem para apresentação de propostas relativas a cada umas das subcategorias de contratos. 

\begin{table}[h]
	\centering
	\resizebox{\textwidth}{!}{%
	\begin{tabular}{
			>{\columncolor[HTML]{FFFFFF}}l 
			>{\columncolor[HTML]{FFFFFF}}c 
			>{\columncolor[HTML]{FFFFFF}}c }
		\toprule
		\multicolumn{2}{c}{\cellcolor[HTML]{FFFFFF}\textbf{Tipo de Contrato}}                           & \textbf{\begin{tabular}[c]{@{}c@{}}Prazo mínimo para \\ apresentação de propostas\end{tabular}} \\ \midrule
		Empreitada de obras públicas        & \cellcolor[HTML]{FFFFFF}                                  & \cellcolor[HTML]{FFFFFF}                                                                 \\
		Concessão de obras públicas         & \multirow{-2}{*}{\cellcolor[HTML]{FFFFFF}Obras}           & \multirow{-2}{*}{\cellcolor[HTML]{FFFFFF}14 dias}                                        \\ \midrule
		Concessão de serviços públicos      & \cellcolor[HTML]{FFFFFF}                                  & \cellcolor[HTML]{FFFFFF}                                                                 \\
		Locação ou aquisição de bens móveis & \cellcolor[HTML]{FFFFFF}                                  & \cellcolor[HTML]{FFFFFF}                                                                 \\
		Aquisição de serviços               & \cellcolor[HTML]{FFFFFF}                                  & \cellcolor[HTML]{FFFFFF}                                                                 \\
		Sociedade                           & \multirow{-4}{*}{\cellcolor[HTML]{FFFFFF}Bens e Serviços} & \multirow{-4}{*}{\cellcolor[HTML]{FFFFFF}6 dias}                                         \\ 
		\bottomrule
	\end{tabular}%
	}
	\caption{Tipologia de contratos e respetivos prazos mínimos para apresentação de propostas por parte de entidades interessadas. }
	\label{table:3}
\end{table}

%Define-se, no  n.º 1 do artigo 343.º do CCP, \textbf{empreitada de obras públicas} como o contrato com objetivo de desenvolver e/ou executar uma obra pública inserida no regime de ingresso e permanência na actividade de construção. Por sua vez, no n.º 1 do artigo 407.º, é establecido que na \textbf{concessão de obras públicas} o co-contratante, obrigado, por lei, ao desenvolvimento e/ou execução do objeto contratual, tem direito a proceder, durante um determinado período de tempo, à exploração (da obra??) e, eventualmente, ao pagamento de um preço.
%Em suma, podem-se combinar os dois tipos de contratos anteriormente enunciados numa categoria definida como Obras. O n.º2 do artigo 343.º define como obra pública qualquer trabalho de construção, reconstrução, ampliação, alteração ou adaptação, conservação, restauro, reparação, reabilitação, beneficiação e demolição de bens imóveis executados por conta de um contraente público.

 




\subsubsection{Escolha do procedimento}

O artigo 36.º do CCP define que a escolha do tipo de procedimento, aquando da decisão de contratar, deve ser devidamente fundamentada. Antes da abertura de um procedimento de formação de contrato público, a entidade adjudicante pode efetuar consultas informais ao mercado a fim de poderem vir a ser utilizadas no planeamento da contratação. No caso dessa consulta ser efetuada a uma empresa que, posteriormente, se candidate ao concurso em questão, deve ser comunicada essa informação aos restantes concorrentes e incluí-la nas peças do procedimento\cite{guia_poise}.

Além disso, no momento da escolha do tipo de procedimento, devem ser considerados um de dois critérios: \textbf{critério do valor} e \textbf{critério material}. 
Ao optar pelo \textbf{critério do valor}, existem valores contratuais máximos consoante o tipo de procedimento, como se pode constatar na Tabela \ref{tab:4}. Define-se como valor do contrato, o valor máximo do benefício económico obtido pela entidade contratada após a completitude de todas as prestações que respeitam ao objeto contratual.

Se for elegido o \textbf{critério do valor}, nos termos do artigo 23.º do CCP, é permitida a celebração de contratos de qualquer valor. Para tal, é necessário que o órgão competente para a decisão de contratar, fundamente, de forma clara e objetiva, que os procedimentos adotados cumprem todos os requisitos previstos nos artigos 24.º a 30.º do CCP.




\subsubsection{Situações Excepcionais}


Aquando do desenvolvimento de contratos públicos existem algumas situações excepcionais, destacando-se a formação de \textbf{contratos mistos} e a \textbf{adjudicação por lotes}.
Os \textbf{contratos mistos} consistem num objeto contratual que contempla dois ou mais tipos de contrato diferentes\cite{mistos}, como por exemplo o \textit{fornecimento de bens móveis} e a \textit{prestação de serviços}.
A \textbf{adjudicação por lotes}, definida no artigo 46.º-A do CCP, consiste na divisão de um contrato financeiramente avultado, em vários contratos de valor inferior, sendo o número máximo de lotes estipulado pela entidade adjudicante. Desta forma, é permitida a participação de pequenas e médias empresas que não teriam capacidade organizacional e técnico-financeira adequada para a realização total do contrato. Na formação de contratos públicos de \textit{Bens e Serviços}, com um valor contratual superior a 135.000,00 €, a decisão de não contratação por lotes carece de obrigação de fundamentação, devendo respeitar os pontos inscritos no nr.º 2 do artigo 46.º-A do CCP. Na formação de contratos públicos de \textit{Obras}, aplica-se o procedimento anteriormente mencionado nas situações em que o valor contratual é superior a 500.000,00 €. 


\begin{table}[h!]
	\renewcommand{\arraystretch}{1.6}
	\setlength{\tabcolsep}{15pt}
	\resizebox{\textwidth}{0.7\height}{%
	\begin{tabular}{ccccc}
		\hline
		\rowcolor[HTML]{C0C0C0} 
		\multicolumn{2}{c}{\cellcolor[HTML]{C0C0C0}\textbf{Tipo de Procedimento}}                                               & \textbf{Preço Base}                                                 & \textbf{Objeto} & \textbf{Base Legal (CCP)} \\ \hline
		\rowcolor[HTML]{FFFFFF} 
		\cellcolor[HTML]{FFFFFF}                                & \cellcolor[HTML]{FFFFFF}                                      & $\leq$ €10.000,00                                                   & Obras           & artigo 128.º, n.º1        \\
		\rowcolor[HTML]{FFFFFF} 
		\cellcolor[HTML]{FFFFFF}                                & \multirow{-2}{*}{\cellcolor[HTML]{FFFFFF}Regime Simplificado} & $\leq$ €5.000,00                                                    & Bens e Serviços & artigo 128.º, n.º1        \\ \cline{2-5} 
		\rowcolor[HTML]{FFFFFF} 
		\cellcolor[HTML]{FFFFFF}                                & \cellcolor[HTML]{FFFFFF}                                      & $\leq$ €30.000,00                                                   & Obras           & artigo 19.º, al d)        \\
		\rowcolor[HTML]{FFFFFF} 
		\cellcolor[HTML]{FFFFFF}                                & \cellcolor[HTML]{FFFFFF}                                      & $\leq$ €20.000,00                                                   & Bens e Serviços & artigo 20.º, n.º1, al c)  \\
		\rowcolor[HTML]{FFFFFF} 
		\multirow{-5}{*}{\cellcolor[HTML]{FFFFFF}Ajuste Direto} & \multirow{-3}{*}{\cellcolor[HTML]{FFFFFF}Regime Geral}        & $\leq$ €50.000                                                      & Outros          & artigo 21.º, n.º1, al c)  \\ \hline
		\rowcolor[HTML]{EFEFEF} 
		\multicolumn{2}{c}{\cellcolor[HTML]{EFEFEF}}                                                                            & $\leq$ €150.000,00                                                  & Obras           & artigo 19.º, al c)        \\
		\rowcolor[HTML]{EFEFEF} 
		\multicolumn{2}{c}{\cellcolor[HTML]{EFEFEF}}                                                                            & $\leq$  €75.000,00                                                  & Bens e Serviços & artigo 20.º, n.º1, al c)  \\
		\rowcolor[HTML]{EFEFEF} 
		\multicolumn{2}{c}{\multirow{-3}{*}{\cellcolor[HTML]{EFEFEF}Consulta Prévia}}                                           & $\leq$ €100.000                                                     & Outros          & artigo 21.º, n.º1, al c)  \\ \hline
		\rowcolor[HTML]{FFFFFF} 
		\multicolumn{2}{c}{\cellcolor[HTML]{FFFFFF}}                                                                            & \cellcolor[HTML]{FFFFFF}                                            & Obras           & artigo 19.º, al b)        \\
		\rowcolor[HTML]{FFFFFF} 
		\multicolumn{2}{c}{\cellcolor[HTML]{FFFFFF}}                                                                            & \multirow{-2}{*}{\cellcolor[HTML]{FFFFFF}Até aos Limiares Europeus*} & Bens e Serviços & artigo 20.º, n.º1, al b)  \\
		\rowcolor[HTML]{FFFFFF} 
		\multicolumn{2}{c}{\multirow{-3}{*}{\cellcolor[HTML]{FFFFFF}Concurso Público}}                                          & Qualquer valor                                                      & Outros          & artigo 21.º, n.º1, al a)  \\ \hline
	\end{tabular}%
	}
	\caption{Valores contratuais máximos permitidos por lei consoante a tipologia de procedimento e de contrato.}
	\label{tab:4}
\end{table}

Os Limiares Europeus são valores estabelecidos pela União Europeia que determinam a partir de que montante é obrigatório aplicar as normas europeias em matéria de contratação pública, variando consoante o tipo de contrato. Estes limiares são atualizados periodicamente e, para o ano de 2024, os valores a praticar são de $5.350.000,00$ € para \textit{Obras}. Para \textit{Bens e Serviços} este valor cai para os $140.000,00$ €, caso a entidade contratante seja uma autoridade governamental ou $431.000,00$ € caso a entidade contratante seja uma entidade do setor público.




\subsection{Tramitação Procedimental}

Na Figura \ref{fig:pecas} é possível visualizar, de uma forma simplista, as peças que são transversais a todos os tipos de procedimento contratual. 
São estas peças que definem as formalidades e requisitos que devem ser cumpridos na fase de elaboração e apresentação de propostas pelas entidades concorrentes. 
%\footnote{No caso do Ajuste Direto e Consulta Prévia não são tidas em conta as duas primeiras etapas. No Diálogo Concorrencial existem outras etapas intermédias.}
\begin{figure}[H]
	\centering
	\includegraphics[width=0.55\textwidth]{imagens/pecasprocedimento.png}
	\caption{Ilustração das peças do procedimento de um concurso público.}
	\label{fig:pecas}
\end{figure}


O anúncio materializa-se em documento oficial, com base no qual a entidade adjudicante dá a conhecer ao mercado o início do procedimento de contratação pública. Este é obrigatoriamente publicado no Diário da República e, sob determinadas circunstâncias, no Jornal Oficial da União Europeia. \\
O programa do procedimento consiste no regulamento que define os termos que devem ser cumpridos desde a fase de formação do contrato até ao seu término\cite{programaproc}. \\
Por último, o caderno de encargos é a peça do procedimento onde estão definidas as cláusulas do contrato a celebrar \cite{caderno}. 
Na Figura \ref{fig:fasescp} é possível observar, com outro nível de detalhe, as fases constituintes da formação de um concurso público. 

\begin{figure}[H]
	\centering
	\includegraphics[width=.8\textwidth]{imagens/fasesconcpub.png}
	\caption{Fases de um concurso público.}
	\label{fig:fasescp}
\end{figure}




%
%
%
%\chapter{Base de Dados}
%\section{Portal BASE}

O Portal BASE  é um \textit{website} que tem como missão a divulgação de informação relativa a contratos públicos celebrados em Portugal, ao abrigo do CCP.
Este espaço virtual, com a sua primeira versão lançada no ano de 2008, assume-se como a central de informação de contratação pública, onde são publicados os elementos referentes à formação e execução de contratos. 
Desta forma, é possível acompanhar e monitorizar os contratos, tornando o processo transparente e acessível a qualquer cidadão. 


%\begin{figure}[H]
%	\centering
%	\includegraphics[scale=.5]{imagens/base.jpg}
%	\caption{Logotipo do Portal BASE}
%	\label{fig:base}
%\end{figure}



\subsection{Informação disponibilizada no Portal BASE}

No Portal BASE é possível encontrar informação relativa:

\begin{my_enumerate}
	\item Aos anúncios publicados no Diário da República relativos a procedimentos de formação de contratos públicos.
	\item Ao acesso às peças do procedimento.
	\item À formação dos contratos públicos sujeitos à parte II do CCP e à execução dos contratos administrativos sujeitos à parte III do CCP.
	\item À disponibilização e alienação de bens móveis.
	\item Às decisões definitivas de aplicação da sanção de proibição de participação previstas nos artigos 460.º e 464.º-A do CCP, durante o período da respetiva proibição.
	\item Às modificações objetivas de contratos que representem um valor acumulado superior a 10\% do preço contratual, as quais ficam disponibilizadas até seis meses após a extinção do contrato, nos termos do n.º 1 do artigo 315.º do CCP.
\end{my_enumerate}

Além do anteriormente mencionado, encontra-se disponível na Tabela \ref{tab:base} documentação suplementar relacionada com contratação pública.

%\begin{itemize}
%	\setlength{\itemsep}{0.2pt}
%	\setlength{\parskip}{-2pt}
%	\setlength{\parsep}{0pt}
%	
%	\item Legislação, regulamentação e jurisprudência nacional e comunitária
%	\item Guias de boas práticas e orientações técnicas
%	\item Informação estatística, na forma de relatórios anuais e sínteses mensais
%	\item Comunicados, notícias e eventos 
%
%\end{itemize}

\clearpage

\begin{table}[h!]
	\centering
	\resizebox{\textwidth}{!}{%
	\begin{tabular}{L L L L}
		\toprule
		Legislação, regulamentação e jurisprudência nacional e comunitária & Comunicados, notícias e eventos & Informação estatística, relatórios anuais e sínteses mensais & Guias de boas práticas e orientações técnicas \\
		\bottomrule
	\end{tabular}
	}
	\caption{Documentação suplementar disponível no Portal BASE}
	\label{tab:base}
\end{table}

A Figura \ref{fig:site1} reproduz a página inicial do \textit{website} do Portal BASE. Todas as entradas mencionadas na Tabela \ref{tab:base} podem ser facilmente consultadas na barra inicial delimitada a vermelho. É possível consultar, não só contratos, tal como se pode observar no rectângulo vermelho do lado esquerdo, assim como Anúncios, Entidades, Modificações Contratuais, Bens Móveis ou Impugnações. O campo \textbf{Pesquisa Avançada} permite selecionar vários parâmetros, nomeadamente, tipo de procedimento e contrato, intervalo de preço contratual, categoria de contrato, local de execução do contrato, data de celebração, entidade adjudicante e adjudicatária, entre outros. 

\begin{figure}[H]
	\centering
	\includegraphics[width=\textwidth]{imagens/portalbase_init_v2.png}
	\caption{\textit{Screenshot} da página inicial do Portal BASE.}
	\label{fig:site1}
\end{figure}

\clearpage

\begin{figure}[H]
	\centering
	\includegraphics[width=\textwidth]{imagens/portalbase.png}
	\caption{\textit{Screenshot} do campo de pesquisa avançada do Portal BASE.}
	\label{fig:site2}
\end{figure}

Na Figura \ref{fig:site3} é possível verificar os últimos quatro contratos adicionados ao Portal BASE, no momento de captura de imagem do ecrã, dia 24 de abril de 2024. Para aceder às especificidades de cada um dos contratos é necessário pressionar o sinal \img{imagens/plus.png}, que se encontra delimitado a vermelho. A título de exemplo, identificam-se na Figura \ref{fig:site4} os detalhes contratuais do concurso público referente à contrução da nova linha de metro \textit{Rubi}, na cidade do Porto.

\begin{figure}[H]
	\centering
	\includegraphics[width=\textwidth]{imagens/portalbase_pesquisa.png}
	\caption{\textit{Screenshot} dos últimos quatro contratos adicionados ao Portal BASE.}
	\label{fig:site3}
\end{figure}

\clearpage
\begin{figure}[H]
	\centering
	\includegraphics[width=\textwidth]{imagens/metro.png}
	\caption{\textit{Screenshot} dos detalhes do contrato referentes à contrução da Linha Rubi, do Metro do Porto.}
	\label{fig:site4}
\end{figure}

\clearpage
\subsection{Entidades envolvidas}

Existem diversas entidades que suportam o processo de contratação pública em Portugal. 

%\begin{figure}[h]
%	\centering
%	\begin{minipage}{.25\textwidth}
%		\centering
%		\includegraphics[width=\linewidth]{imagens/ESPAP.png}
%	\end{minipage}%
%	\begin{minipage}{.25\textwidth}
%		\centering
%		\includegraphics[scale = 0.1]{imagens/impic.jpg}
%	\end{minipage}%
%	\begin{minipage}{.25\textwidth}
%		\centering
%		\includegraphics[width=\linewidth]{imagens/incm.png}
%	\end{minipage}%
%	\begin{minipage}{.25\textwidth}
%		\centering
%		\includegraphics[width=.7\linewidth]{imagens/gns.png}
%	\end{minipage}
%	\caption{Entidades envolvidas no processo contratação pública}
%\end{figure}


\begin{my_enumerate}
	\item  \textbf{Instituto dos Mercados Públicos, do Imobiliário e da Construção, I.P. (IMPIC)}: É a entidade responsável por gerir o Portal BASE, monitorizar e fiscalizar as plataformas eletrónicas de contratação pública, regular os contratos públicos e assume-se como elo de ligação com a Comissão Europeia, para efeitos do disposto no nº 5, do artigo.º 83, da Diretiva nº 2014/24/EU. É responsável pelo desenvolvimento de manuais de boas práticas sobre contratos públicos de aquisição de obras, de bens e de prestação de serviços. Compete-lhe, ainda, a analise de queixas e denúncias de cidadãos e empresas.
	
	\item \textbf{Entidade de Serviços Partilhados da Administração Pública, I.P. (eSPap)}: É a entidade que desenvolve e presta serviços no âmbito da Administração Pública. Além disso, concebe, gere e avalia o sistema nacional de compras e assegura a gestão do PArque de Veículos do Estado (PVE), apoiando a definição de políticas estratégicas nas áreas das tecnologias de informação e comunicação (TIC) do Ministério das Finanças, garantindo o planeamento, conceção, execução e avaliação das iniciativas de informatização tecnológica dos respetivos serviços e organismos.
	
	\item \textbf{Gabinete Nacional de Segurança (GNS)}: É o organismo que garante a segurança da informação classificada de âmbito nacional e das organizações internacionais de que Portugal faz parte. É responsável por credenciar as plataformas eletrónicas de contratação pública, dos auditores de segurança, de pessoas e empresas para o acesso e manuseamento de informação classificada e entidades que atuem no âmbito do Sistema de Certificação Eletrónica do Estado - Infra-Estrutura de Chaves Públicas (SCEE).
	
	\item \textbf{Imprensa Nacional – Casa da Moeda (INCM)}: É a entidade responsável pelas publicações no Diário da República Eletrónico e no Jornal Oficial da União Europeia. Todos os anúncios dos procedimentos pré-contratuais (Concurso Público, Concurso Limitado por Prévia Qualificação, Procedimento de Negociação, Diálogo Concorrencial e Parceria para a Inovação) são publicados no Diário da República Eletrónico e, simultaneamente, publicitados no Portal BASE (excetuam-se os casos de ajuste direto e consulta prévia).
	
	\item \textbf{Entidades Adjudicantes}: São estas que conduzem e decidem o procedimento de formação de contrato e são responsáveis por introduzir, no Portal, informação sobre os contratos públicos celebrados. 
	
	\item \textbf{Adjudicatário}: Titular da proposta vencedora. Tem de comprovar que respeita os requisitos exigidos para poder celebrar o contrato. A informação é submetida no Portal BASE via plataforma eletrónica.
	
	\item \textbf{Plataforma Eletrónica}: É a infraestrutura tecnológica constituída por um conjunto de aplicações, meios e serviços informáticos onde, de forma totalmente eletrónica e desmaterializada, decorre a tramitação dos procedimentos para a formação de um contrato público. 
	
\end{my_enumerate}







\section{Descrição da Base de Dados}
\label{ch:variables}


O conjunto de dados contratuais disponibilizado e utilizado ao longo deste projeto encontra-se armazenado numa base de dados, em PostgreSQL. Salvo raras exceções, são adicionados ao Portal BASE, com caráter diário, os mais recentes contratos celebrados. Além disso, diariamente, são adicionados à base de dados todos os novos contratos do dia anterior, havendo, por isso, um desfasamento de um dia de forma a garantir que todos eles são coletados. 


\begin{figure}[H]
	\centering
	\includegraphics[width=0.7\textwidth]{imagens/portal_coleta.png}
	\caption{Processo de coleta de contratos e construção da base de dados.}
	\label{fig:processocoleta}
\end{figure}


Até ao primeiro dia do mês de maio, do presente ano civil, o conjunto de dados totalizava 1023443 contratos públicos, celebrados desde o dia 13 de maio de 2003. No trabalho que ora se apresenta, apenas foram considerados os contratos celebrados entre o dia 1 de janeiro de 2018 e o dia 1 de maio de 2024. Por assim ser, o número de contratos relativos ao ano de 2024 encontra-se incompleto. Pelo que, nas representações gráficas presentes nas secções que se seguem, a etiqueta relativa ao ano de 2024 será assinalada como \textbf{2024*}, a fim de sinalizar o facto de o conjunto não estar completo para o presente ano civil. 

Do total das 61 colunas da base de dados onde se encontram guardados todos os contratos públicos, as que reveleram maior interesse foram as seguintes: 


\begin{my_itemize}
	

\item \textbf{id} (integer): É o número que permite identificar um contrato específico na base de dados. A cada contrato foi atribuído um identificador único.


\item \textbf{n\_anuncio} (text): O número de anúncio é o número que permite identificar o anúncio no Diário da República Digital, referente a um determinado procedimento de contratação pública.


\item \textbf{anuncio\_preco\_base} (float): O preço base é definido como o preço máximo que a entidade adjudicante está disposta a pagar pela execução de todas as prestações que constituem o objeto do contrato a celebrar.


\item \textbf{anuncio\_proposalDeadline} (date): Este campo diz respeito ao número de dias estipulado para submeter uma proposta a um determinado concurso, por parte de uma entidade concorrente. 


\item \textbf{tipo\_procedimento} (text): Este parâmetro permite identificar o tipo de procedimento, de acordo com os enunciados na tabela \ref{table:2}, de um  determinado contrato público.


\item \textbf{objeto\_contrato} (text): O objeto de contrato consiste numa descrição detalhada do tipo de objeto celebrado. 


\item \textbf{data\_publicacao} (date): Este parâmetro diz respeito à data de publicação do contrato no Portal BASE.


\item \textbf{data\_celebracao} (date): Este parâmetro diz respeito à data de celebração do contrato. 

\item \textbf{preco\_contratual} (float): O  preço contratual é o preço a pagar pela entidade adjudicante à entidade vencedora após celebração do objeto contratual. 

\item \textbf{entidade\_adjudicante} (text): Este campo contém o nome, número de identificação fiscal e URL que remete para a página \textit{website} do Portal BASE com todos os contratos celebrados de uma determinada entidade adjudicante. 

\begin{figure}[H]
	\centering
	\includegraphics[width=.9\textwidth]{imagens/adjudicante.png}
	\caption{Exemplo de uma entidade adjudicante de um contrato presente na base de dados.}
	\label{fig:adjudicante}
\end{figure}

\item \textbf{fundamentacao} (text): A fundamentação diz respeito ao artigo do CCP utilizado para justificar a adoção do procedimento escolhido. 

\item \textbf{entidades\_contratadas} (text): À semelhança do campo \textbf{entidade\_adjudicante}, este campo contém o nome, número de identificação fiscal e URL que remete para a página \textit{web} do Portal BASE com todos os contratos celebrados de uma determinada entidade adjudicatária. 

\item \textbf{entidades\_concorrentes} (text): Neste campo é possível encontrar o nome, número de identificação fiscal e URL que remete para o \textit{website} do Portal BASE com todos os contratos celebrados para todas as entidades que concorrem a um determinado contrato público. Quando existe mais do que uma entidade concorrente, a separação entre entidades é feita através dos caracteres \(|||\).



\begin{figure}[H]
	\centering
	\includegraphics[width=.9\textwidth]{imagens/concorrentes.png}
	\caption{Exemplo de entidades concorrentes de um contrato presente na base de dados.}
	\label{fig:concorrentes}
\end{figure}


\item \textbf{url\_anuncio} (text): Contém um \textit{link} que redireciona para a página \textit{web} que contém os detalhes do anúncio.

\item \textbf{cpv} (text): Contém o CPV do contrato. CPV é a sigla de \textit{Common Procurement Vocabulary}. O CPV é um código de oito dígitos, usado no processo de contratação, que permite especificar e categorizar de forma hierárquica serviços e produtos. 

\begin{my_itemize}
	\item[$\circ$] \label{sec:cepeves}  Os 2 primeiros dígitos permitem identificar a \textbf{divisão} do serviço/produto. No total, existem 45 divisões.
	\item[$\circ$]  Os 3 primeiros dígitos permitem identificar o \textbf{grupo}. No total, existem 272 grupos.
	\item[$\circ$]  Os 4 primeiros dígitos permitem identificar a \textbf{classe}. No total, existem, 1002 classes.
	\item[$\circ$]  Os 5 primeiros dígitos permitem identificar a \textbf{categoria}. No total, existem 2379 categorias.
	\item[$\circ$]  Os 6 primeiros dígitos permitem identificar a \textbf{subcategoria}. No total, existem 5756 subcategorias.
\end{my_itemize}

A título de exemplo, atente-se no caso seguinte caso:

\begin{figure}[H]
	\centering
	\includegraphics[width=0.7\textwidth]{imagens/cpv.png}
	\caption{Ilustração do CPV de um contrato referente a veículos de combate a incêndios.}
	\label{fig:cpv}
\end{figure}


\begin{my_itemize}
	\item[$\circ$]  \textbf{Divisão:} 34 - Equipamentos de transporte e produtos auxiliares ao transporte
	\item[$\circ$]  \textbf{Grupo:} 341 - Veículos motorizados
	\item[$\circ$]  \textbf{Classe:} 3414 - Veículos motorizados pesados
	\item[$\circ$]  \textbf{Categoria:} 34144 - Veículos motorizados para fins especiais
	\item[$\circ$]  \textbf{Subcategoria:} 34144210 - Veículos de combate a incêndios
\end{my_itemize}



\item \textbf{contractType} (text): Este campo diz respeito ao tipo de contrato celebrado, tal como se encontra apresentado Tabela \ref{table:3}. 

\item \textbf{executionPlace} (text): Indica o local de execução do contrato. 

\item \textbf{totalEffectivePrice} (float): Na eventualidade de existirem alterações do preço contratual após a celebração do contrato, este campo é preenchido. Existem, também, casos em que o preço contratual diz respeito à unidade em questão (p. ex. quilómetro, hora, dia). Nessa situação, o preço total efetivo é o preço total após prestação do serviço \cite{jardinagem}.  

\end{my_itemize}

De entre o universo de contratos celebrados, pode-se perceber na Tabela \ref{tab:contratos} como estes se distribuem consoante os diferentes tipos de procedimentos. Como se pode verificar, os tipos de procedimentos com maior relevância são: o Ajuste Direto em Regime Geral, a Consulta Prévia, o Concurso Público e contratos ao abrigo de acordo-quadro, com especial evidência para o Ajuste Direto em Regime Geral que prefaz 50\% dos contratos. 


\begin{table}[H]
	\centering
	\renewcommand{\arraystretch}{1.15}
	\setlength{\tabcolsep}{15pt}
	\resizebox{\textwidth}{!}} \\ \hline
			Ajuste Direto Regime Geral                                                     & 526860                                                                   & 51.4                                                            \\ \hline
			\rowcolor[HTML]{EFEFEF} 
			Consulta Prévia                                                                & 228400                                                                   & 22.3                                                            \\ \hline
			Concurso público                                                               & 128422                                                                   & 12.5                                                            \\ \hline
			\rowcolor[HTML]{EFEFEF} 
			Ao abrigo de acordo-quadro (art.º 259.º)                                       & 111723                                                                   & 10.9                                                            \\ \hline
			Ao abrigo de acordo-quadro (art.º 258.º)                                       & 24406                                                                    & 2.4                                                             \\ \hline
			Concurso limitado por prévia qualificação                                      & 1816                                                                     & $< 1$                                                            \\ \hline
			\rowcolor[HTML]{EFEFEF} 
			Consulta Prévia Simplificada                                                   & 958                                                                      & $< 1$                                                            \\ \hline
			Contratação excluída II                                                        & 485                                                                      & $< 1$                                                            \\ \hline
			Setores especiais – isenção parte II                                           & 222                                                                      & $< 1$                                                          \\ \hline
			\rowcolor[HTML]{EFEFEF} 
			Procedimento de negociação                                                     & 45                                                                       & $< 1$                                                          \\ \hline
			Concurso público simplificado                                                  & 42                                                                       & $< 1$                                                          \\ \hline
			\rowcolor[HTML]{EFEFEF} 
			Consulta prévia ao abrigo do artigo 7º da Lei n.º 30/2021, de 21.05            & 25                                                                       & $< 1$                                                          \\ \hline
			Ajuste Direto Regime Geral ao abrigo do artigo 7º da Lei n.º 30/2021, de 21.05 & 21                                                                       & $< 1$                                                          \\ \hline
			\rowcolor[HTML]{EFEFEF} 
			Serviços sociais e outros serviços específicos                                 & 9                                                                        & $< 1$                                                          \\ \hline
			Concurso de conceção simplificado                                              & 4                                                                        & $< 1$                                                          \\ \hline
		\end{tabular}%
	}
	\caption{Número de contratos públicos e respetiva fração para os vários tipos de procedimentos.}
	\label{tab:contratos}	
\end{table}



%\rowcolor[HTML]{EFEFEF} 
%Parceria para a inovação                                                       & 2                                                                        & $< 1$                                                          \\ \hline
%Concurso de ideias simplificado                                                & 1                                                                        & $< 1$                                                          \\ \hline
%\rowcolor[HTML]{EFEFEF} 
%Não especificado & 1                                                                        & $< 1$                                                           \\ \hline



A leitura das Figuras \ref{fig:numcontrs} e \ref{fig:precoscontrs} permite inferir uma tendência crescente do número de contratos celebrados entre 2018 e 2023, estando este comportamento em linha com o crescimento do preço contratual total por ano. 

\begin{figure}[H]
	\centering
	\begin{minipage}{.49\linewidth}
		\includegraphics[width=\linewidth]{imagens/contratos_ano.png}
		\caption{Número de contratos celebrados, entre 2018 e 2024, para todas as tipologias.}
		\label{fig:numcontrs}
	\end{minipage}
	\hfill
	\begin{minipage}{.49\linewidth}
		\includegraphics[width=\linewidth]{imagens/precocontr_ano.png}
		\caption{Preço contratual total, entre 2018 e 2024, para todas as tipologias}
		\label{fig:precoscontrs}
	\end{minipage}
\end{figure}


%\begin{wrapfigure}{R}{0.45\textwidth}
%	\centering
%	\includegraphics[width=0.45\textwidth]{imagens/pib.png}
%	\caption{PIB português entre 2018 e 2024}
%\end{wrapfigure}




\begin{table}[H]
	\centering
	\renewcommand{\arraystretch}{1.35}
	\setlength{\tabcolsep}{20pt}
	\resizebox{\textwidth}{!}{%
		\begin{tabular}{l|l}
			\textbf{33} & {\color[HTML]{000000} Equipamento médico, medicamentos, e produtos para cuidados pessoais}     \\
			\textbf{45} & Construção                                                                                     \\
			\textbf{79} & Serviços a empresas: direto, comercialização, consultoria, recrutamento, impressão e segurança \\
			\textbf{50} & Serviços de reparação e manuntenção                                                           
		\end{tabular}%
	}
	\caption{Descrição das principais divisões de CPV.}
	\label{tab:maincpvs}
\end{table}

Na Figura \ref{fig:distritos}, constata-se que existe uma prevalência de contratos celebrados nos distritos de Lisboa e do Porto, com 21\% e 14\%, respetivamente. Pelo contrário, nos distritos do interior esse valor aproxima-se dos 14\%. Existe, também, um elevado de número de contratos cujo distrito é identificado como \textit{Outro}. Estes casos, dizem respeito a erros de preenchimentos aquando da submissão dos contratos no Portal BASE, pois no campo \textbf{executionPlace} verificava-se um dos três cenários: o campo encontrava-se vazio, era inserido \textit{Distrito não determinado} ou era inserido \textit{Portugal Continental}. A partir da Figura \ref{fig:cpvs} observa-se que existe uma predominância de contratos públicos celebrados referentes à aquisição de equipamento médico, medicamentos e produtos para cuidados pessoais. Estes são todos os contratos cujos primeiros dois dígitos do CPV são 33. As quatro principais divisões do CPV, com maior número de contratos públicos celebrados, encontram-se na Tabela \ref{tab:maincpvs}. 

%\begin{table}[H]
%	\centering
%	\renewcommand{\arraystretch}{1.15}
%	\setlength{\tabcolsep}{15pt}
%	\resizebox{\textwidth}{!}{%
%		\begin{tabular}{ccccccccc}
%			\hline
%			\rowcolor[HTML]{C0C0C0} 
%			\textbf{Ano} & \textbf{Count} & \textbf{Média} & \textbf{D.Padrão} & \textbf{Min} & \textbf{Q1} & \textbf{Q2} & \textbf{Q3} & \textbf{Máx} \\ \hline
%			2024         & 90658          & 77953.18       & 1669717           & 0            & 3000        & 10800       & 29960.68    & 321888000    \\ \hline
%			\rowcolor[HTML]{EFEFEF} 
%			2023         & 191800         & 80514.25       & 1257076           & 0            & 4462.195    & 12005.18    & 33094.81    & 379500000    \\ \hline
%			2022         & 187193         & 66523.14       & 2106541           & 0            & 2250        & 9600.37     & 26260       & 881534700    \\ \hline
%			\rowcolor[HTML]{EFEFEF} 
%			2021         & 192112         & 71350.19       & 1723635           & 0            & 1564        & 8786.44     & 25560       & 397191000    \\ \hline
%			2020         & 164008         & 65710.67       & 804483.3          & 0            & 2280        & 9775.44     & 27763.44    & 130286000    \\ \hline
%			\rowcolor[HTML]{EFEFEF} 
%			2019         & 144769         & 61262.23       & 655422.8          & 0            & 3499        & 10682       & 28935       & 130463800    \\ \hline
%			2018         & 122089         & 58729.34       & 382421.9          & 0            & 4100        & 11458.2     & 30496.77    & 38750000     \\ \hline
%		\end{tabular}%
%	}
%	\caption{}
%	\label{tab:my-table}
%\end{table}



\clearpage
\vfill
\begin{figure}[H]
	\centering
	\includegraphics[width=\textwidth]{imagens/treemap_distritos.png}
	\caption{Distribuição do número de contratos públicos, por distrito, em Portugal.}
	\label{fig:distritos}
\end{figure}
\vfill 
\begin{figure}[H]
	\centering
	\includegraphics[width=\textwidth]{imagens/treemap_contratos.png}
	\caption{Distribuição do número de contratos públicos, por divisão do CPV.}
	\label{fig:cpvs}
\end{figure}
\vfill
\clearpage


\begin{wrapfigure}{R}{0.49\textwidth}
	\centering
	\includegraphics[width=0.49\textwidth]{imagens/precoscontr_stat.png}
	\caption{Boxplot dos preços contratuais para toda a tipologia de contratos desde 2018 até 2024}
	\label{fig:precotodos}
\end{wrapfigure}


Outra das variáveis com especial relevância e que foi considerada ao longo do projeto é o preço contratual. A Figura \ref{fig:precotodos} representa os \textit{boxplots} referentes aos preços contratuais para todos os contratos da base de dados, independentemente da tipologia de procedimento, por ano e durante o período de tempo considerado (de 2018 a 2024). É importante sublinhar que, tendo em conta que existem 2672 contratos cujo preço contratual é nulo, estes gráficos são meramente ilustrativos, correndo assim um desfasamento entre a informação apresentada e os valores reais. Pode-se observar que os valores dos 1º, 2º e 3º quartis tomam, sensivelmente, valores próximos para todos os anos. A distância interquartil é, aproximadamente, a mesma para os anos considerados e a distribuição dos valores é claramente assimétrica à direita. Porém, tendo em conta que neste cálculo são consideradas todas as tipologias de contratos e todas as divisões de CPV, esta informação não é conclusiva. Evidentemente, os valores adjudicados aos contratos referentes à construção civil (45) são mais expressivos que os referentes a serviços recreativos, culturais e desportivos (92). Por sua vez, tal como foi apresentado no capítulo anterior, existem valores contratuais máximos permitidos mediante a tipologia de contrato. Desta forma, é natural que exista um elevado número de \textit{outliers}, não representados na Figura \ref{fig:precotodos}. \\
Ao longo deste estágio foram apenas consideradas duas tipologias de contratos: Ajustes Diretos em Regime Geral e Concursos Públicos. Assim, é oportuno incluir alguma descrição estatística destes dois casos. Foram construídos  \textit{boxplots} como representações gráficas relativas ao número de contratos e respetivo preço contratual total, por ano. Os resultados encontram-se nas Figuras \ref{fig:precocps}, \ref{fig:precocps1} e \ref{fig:precocps2}, para concursos públicos e nas Figuras \ref{fig:precoad}, \ref{fig:precoad1} e \ref{fig:precoad2}, para ajustes diretos em regime geral. 

Deste modo, possível observar que os valores contratuais dos concursos públicos, tal como observado na Figura \ref{fig:precocps}, são substancialmente superiores aos dos ajustes diretos, tal como ilustrado na Figura \ref{fig:precoad}. Apesar do número de contratos celebrados por ano para concursos públicos, tal como indicado na Figura \ref{fig:precocps1}, ser sempre inferior ao de ajustes diretos - Figura \ref{fig:precoad1} -, o valor adjudicado, por ano, foi sempre superior ao dos ajustes diretos. Deste modo, pode concluir-se que os valores contratuais de concursos públicos tem, necessariamente, de ser superiores. Em ambos os casos, as distribuições dos valores contratuais são assimétricas. 

No caso dos concursos públicos, 23\% dos contratos celebrados referem-se à aquisição de equipamento médico e 17\% a obras de contrução civil (ver Figura \ref{fig:cpcpv}), sendo que cerca de 25\% são celebrados no distrito de Lisboa, 10\% no Porto e 11\% em distritos indeterminados (ver Figura \ref{fig:cploc}). Relativamente a ajustes diretos, existe uma predominância de contratos referentes à aquisição de equipamento médico, totalizando 28\% dos contratos celebrados (ver Figura \ref{fig:adcpv}), com maior destaque para os distritos de Lisboa, com 23\% e Porto, com 17\%  (ver Figura \ref{fig:adloc}).

\vfill

\begin{figure}[H]
	\centering
	\begin{minipage}{0.31\linewidth}
		%\centering  % redundant
		\includegraphics[width=\textwidth]{imagens/concpub_stat.png}
		\caption{Boxplot dos preços contratuais para concursos públicos entre 2018 até 2024.}
		\label{fig:precocps}
	\end{minipage}
	\hfill
	\begin{minipage}{.31\linewidth}
		\includegraphics[width=\linewidth]{imagens/cpub_nrcontr.png}
		\caption{Número de concursos públicos celebrados, entre 2018 e 2024.}
		\label{fig:precocps1}
	\end{minipage}
	\hfill
	\begin{minipage}{.31\linewidth}
		\includegraphics[width=\linewidth]{imagens/cpub_price.png}
		\caption{Preço contratual total, entre 2018 e 2024, para concursos públicos.}
		\label{fig:precocps2}
	\end{minipage}
\end{figure}

\vfill

\begin{figure}[H]
	\centering
	\begin{minipage}{.31\linewidth}
		\includegraphics[width=\linewidth]{imagens/adir_stat.png}
		\caption{Boxplot dos preços contratuais de ajustes diretos celebrados, entre 2018 e 2024.}
		\label{fig:precoad}
	\end{minipage}
	\hfill
	\begin{minipage}{0.31\linewidth}
		%\centering  % redundant
		\includegraphics[width=\linewidth]{imagens/adir_nrcontr.png}
		\caption{Número de ajustes diretos celebrados, entre 2018 e 2024.}
		\label{fig:precoad1}
	\end{minipage}
	\hfill
	\begin{minipage}{.31\linewidth}
		\includegraphics[width=\linewidth]{imagens/adir_price.png}
		\caption{Preço contratual total, entre 2018 e 2024, para ajustes diretos.}
		\label{fig:precoad2}
	\end{minipage}
\end{figure}


\clearpage
\begin{figure}[H]
	\centering
	
	\begin{minipage}[t]{0.49\textwidth}
		\centering
		\includegraphics[width=\textwidth]{imagens/treemap_cpub.png}
		\caption{Distribuição do número de concursos públicos, por divisão de CPV.}
		\label{fig:cpcpv}
	\end{minipage}
	\hfill
	\begin{minipage}[t]{0.49\textwidth}
		\centering
		\includegraphics[width=\textwidth]{imagens/treemap_cpub_distritos.png}
		\caption{Distribuição do número de concursos públicos, por distrito.}
		\label{fig:cploc}
	\end{minipage}
	
	\vspace{1em}
	
	\begin{minipage}[t]{0.49\textwidth}
		\centering
		\includegraphics[width=\textwidth]{imagens/treemap_contratos_adir.png}
		\caption{Distribuição do número de ajustes diretos em regime geral, por divisão de CPV.}
		\label{fig:adcpv}
	\end{minipage}
	\hfill
	\begin{minipage}[t]{0.49\textwidth}
		\centering
		\includegraphics[width=\textwidth]{imagens/treemap_distritos_adir.png}
		\caption{Distribuição do número de ajustes diretos em regime geral, por distrito.}
		\label{fig:adloc}
	\end{minipage}
	
\end{figure}
\clearpage


%\subsection{Exemplo ilustrativo: Contratos referentes à área da saúde}
%
%\begin{figure}[H]
%	\begin{minipage}{0.33\linewidth}
%		%\centering  % redundant
%		\includegraphics[width=\textwidth]{imagens/cp_anos_33_v1.png}
%		\caption{Boxplot dos preços contratuais para concursos públicos referentes à área da saúde desde 2018 até 2024}
%	\end{minipage}%
%	\hfill% not: "\hspace{0.5cm}"
%	\begin{minipage}{0.33\linewidth}
%		%\centering  % redundant
%		\includegraphics[width=\textwidth]{imagens/cp_anos_33_v2.png}
%		\caption{Número de concursos públicos celebrados referentes à área da saúde desde 2018 até 2024}
%	\end{minipage}%
%	\hfill% not: "\hspace{0.5cm}"
%	\begin{minipage}{0.33\linewidth}
%		%\centering  % redundant
%		\includegraphics[width=\textwidth]{imagens/cp_anos_33_v3.png}
%		\caption{Montante total adjudicada a contratos públicos celebrados referentes à área da saúde desde 2018 até 2024}
%	\end{minipage}
%\end{figure}
%
%
%
%\begin{figure}[H]
%	\begin{minipage}{0.33\linewidth}
%		%\centering  % redundant
%		\includegraphics[width=\textwidth]{imagens/ajdir_anos33_v1.png}
%		\caption{Boxplot dos preços contratuais para ajustes diretos em regime geral referentes à área da saúde desde 2018 até 2024}
%	\end{minipage}%
%	\hfill% not: "\hspace{0.5cm}"
%	\begin{minipage}{0.33\linewidth}
%		%\centering  % redundant
%		\includegraphics[width=\textwidth]{imagens/ajdir_anos33_v2.png}
%		\caption{Número de ajustes diretos em regime geral celebrados referentes à área da saúde desde 2018 até 2024}
%	\end{minipage}%
%	\hfill% not: "\hspace{0.5cm}"
%	\begin{minipage}{0.33\linewidth}
%		%\centering  % redundant
%		\includegraphics[width=\textwidth]{imagens/ajdir_anos33_v3.png}
%		\caption{Montante total adjudicada a ajustes diretos em regime geral celebrados referentes à área da saúde desde 2018 até 2024}
%	\end{minipage}
%\end{figure}

%
%
%
%\chapter{Noções Matemáticas}
%\section{Indicadores Estatísticos Descritivos}

Falar nesta secçaõ sobre:

\begin{itemize}
	\item Boxplot / diagrama de extremos e quartis
	\item qunatis e as várias definições
	\item outliers
	\item função inversa generalizada
	\item missing values
	\item medidas de risco
\end{itemize}


\begin{figure}[H]
	\centering
	\includegraphics[width=0.7\textwidth]{imagens/skewed.png}
	\caption{}
	\label{}
\end{figure}

FAZER REPRESENTACAO GRÁFICA DESTE TIPO PARA JUSTIFICAR O PARÁGRAFO QUE ANTECEDE O PLOT DA COMPARACAO DA MEDIA, MEDIANA E DP DO NEC POR CPV DA FLAG R019
%
%
%
%\chapter{Construção das \textit{Red Flags}}
%%\section{Exploração do Dataset}
%
%Numa primeira fase foram criadas funções para explorar o conjunto de dados. \\
%
%A função \textbf{col\_names} retorna-nos uma dataframe com todos os nomes das colunas. A função \textbf{n\_contracts} retorna-nos o número de linhas, ou seja, o número de contratos totais, da base de dados. \\
%
%Foi desenvolvida, também, uma função \textbf{h} que permite ver uma dataframe de forma mais apresentável.\\
%
%\textit{É preciso criar uma função com o código que já tenho para categorizar os contratos de acordo com o tipo de procedimento. Idem para o tipo de contrato.} \\

 

%\begin{table}[H]
%	\centering
%	\includegraphics[width= \textwidth]{tabelas/tabela_rf.pdf}
%	\caption{}
%	\label{}
%\end{table}



 

%\section{Funções Auxiliares}
%
%Para a construção das \textit{flags} são necessárias outras funções mais simples. Certas funções irão ter um único propósito e os outputs de certas funções irão ser usados como inputs de outras funções. \\
%
%Para identificar um contrato necessitamos do seu id. Assim, criou-se uma função \textbf{all\_ids} que retorna os ids de todos os contratos da base de dados. Contudo, como não vamos trabalhar com todos os tipos de contratos e procedimentos, é necessário filtrar estes ids.
%Para tal, criaram-se as funções \textbf{ajustes\_dir}, \textbf{consulta\_prev} e \textbf{concurso\_pub} que retornam os ids de todos os ajustes diretos, consultas prévias e concursos públicos, respetivamente. \\
%
%De seguida, filtraram-se os contratos por tipo de procedimento e por CPV. Para isso, criou-se a função \textit{cpv\_direto} que retorna todos os ids de ajustes diretos para serviços de consultoria em IT ( todos os CPV's começados por 72). De seguida, foi feito o mesmo para contratos públicos na função \textit{cpv\_cpub}. De forma a generalizar, criou-se a função \textit{cpv} que retorna todos os ids de contratos de um tipo de procedimento e para um determinado CPV. Tanto o tipo de procedimento como os primeiros 2 algarismos do CPV são parâmetros de entrada da função. \\
%
%
%Para começar, criaram-se duas funções que permitem ver qual é o contrato associado a um determinado id. A função \textit{contrato} tem como input um único id de um contrato ( todas estas funções dizem respeito à tabela \textit{contratos} da DB) e retorna uma dataframe com uma única linha referente ao contrato associado a esse mesmo id. A função \textit{contratos} faz exatamente a mesma coisa mas para um conjunto de id's. \\
%
%Quando as flags estiverem construídas e quisermos ver o anúncio de um contrato suspeito no site do basegov, podemos fazê-lo usando a função \textit{url} que, novamente, tem como parâmetro de entrada o id de um contrato. Esta função pode ter bastante utilidade nos casos que chamam bastante a atenção, como por exemplo, quando se viu um ajuste direto de 3 milhões de euros. \\

No website da OCDS encontra-se disponível para descarregar uma folha de cálculo em Excel com 73 red flags \cite{spreadsheet1} \cite{spreadsheet}, tal como se encontra representado na Tabela \ref{table:flags}.

\begin{table}[H]
	\centering
	\includegraphics[width=\textwidth]{imagens/tabela_flags.png}
	\caption{Exemplo da descrição de uma flag}
	\label{table:flags}
\end{table}


A fim de construir as funções que permitam selecionar contratos anómalos, foi feita uma análise, de caráter subjetivo, relativamente ao nível de importância e de facilidade de implementação. Para cada uma das 73 flags, foi atribuído um valor de 1 a 3, tanto para o nível de facilidade de implementação como para o nível de importância. Para atribuir um valor à facilidade de implementação foi necessário ter em conta se a base de dados possuía todas as variáveis necessárias, o que nem sempre se verificou. Relativamente ao nível de importância, o critério foi mais subjetivo e serviu como uma ferramenta de filtragem. Após esta classificação, foi realizada uma média ponderada destas classificações, com uma peso atribuído de 0.6 ao primeiro parâmetro e de 0.4 ao segundo parâmetro. Dessa forma, foi possível filtrar todas as flags e excluir aquelas que não eram passíveis de implementar. 

Além destes indicadores, foram desenvolvidos outros 3, após discussão, que se acreditava terem valor para o resultado final. Estes indicadores foram construídos tendo em conta o CCP, as variáveis presentes na base de dados e flags definidas pela OCDS. 

Nas secções que se seguem, todos os indicadores establecidos pela OCDS são identificados pela letra R e um número, enquanto que aqueles que os que foram desenvolvidos em equipa são identificados por RF e um número.

Os contratos coletados encontram-se todos guardados numa tabela chamada \textit{contratos\_basegov} numa base de dados em PostgreSQL. De modo a construir os indicadores que se encontram nos capítulos seguintes, foi necessário criar quatro tabelas adicionais auxiliares:

\begin{my_itemize}
	\item concursos\_publicos: tabela para onde foram copiadas as principais variáveis referentes a concursos públicos
	
	\item ajustes\_diretos: tabela para onde foram copiadas as principais variáveis referentes a ajustes diretos em regime geral
	
	\item cpv\_stats: tabela que contém os indicadores estatísticos relativamente ao número de entidade de concorrentes por divisão de CPV. Para cada divisão de CPV, dentro de todo o conjunto de concursos públicos, foi calculado o número total de contratos celebrados (count), o número total de entidades que se candidataram a todos os concursos (nec\_t), o número médio de entidades concorrentes por concurso, desvio padrão, mínimo, máximo, Q1, Q2 e Q3. 
	
	\item preco\_stats: Esta tabela segue uma filosofia similar à apresentada anteriormente. Contudo, em vez de se fazer uma análise por divisão de CPV, esta é feita a partir do Grupo do CPV (os três primeiros dígitos) a fim de ter uma maior granularidade. 
\end{my_itemize}


Na Figura \ref{fig:basededados} encontra-se um esquema da relação entre as diferentes tabelas. Para efeitos de concisão do diagrama apenas são apresentadas algumas das principais colunas. As colunas que se encontram a vermelho, explicadas nas secções que se segue, dizem respeito a colunas auxiliares para a construção dos indicadores

\begin{figure}[H]
	\centering
	\includegraphics[width=0.9\textwidth]{imagens/basedadostabelas.png}
	\caption{Organização das tabelas da base de dados.}
	\label{fig:basededados}
\end{figure}





\section{RF1: Verificação dos Preços Contratuais para Ajustes Diretos em Regime Geral}


Como foi apresentado anteriormente na Tabela \ref{tab:4}, os ajustes diretos, consoante o tipo de obra/serviço, tem diferentes limites máximos impostos por lei relativamente ao valor de adjudicação. A fim de poder identificar os contratos que não estão em conformidade com o CCP, procedeu-se da seguinte forma. Primeiramente, criou-se uma nova tabela na base de dados, a partir da tabela original, para onde foram copiados todos os ajustes diretos em regime geral e onde foi criada um nova coluna, chamada \textbf{criterio}. De seguida, para cada um dos contratos, foi identificado o artigo utilizado para fundamentar a adoção deste tipo de procedimento, presente na coluna da fundamentação. Se o artigo pertencer entre o 17º e o 22º, é atribuído ao contrato a string \textbf{valor} na nova coluna criada. Caso contrário, é atribuído a string \textbf{material}. 

 \begin{table}[H]
 	\centering
 	\begin{tabular}{|c|c|c|c|}
 		\hline
 		\textbf{Critério}                  & \textbf{Artigos}           & \textbf{Tipo de Contrato}           & \textbf{Valor} \\ \hline
 		\multirow{3}{*}{Critério do Valor} & \multirow{3}{*}{17º a 22º} & Aquisição de bens móveis e serviços & 20000€         \\ \cline{3-4} 
 		&                            & Empreitadas de obras públicas       & 30000 €        \\ \cline{3-4} 
 		&                            & Outro tipo de contratos             & 50000 €        \\ \hline
 		Critério Material                  & 24º a 27º                  & Qualquer                            & Indefinido     \\ \hline
 	\end{tabular}
 	\caption{Valores máximos permitidos para Ajustes Diretos consoante o critério, artigo e tipo de contrato.}
 \end{table}

De seguida, foi necessário classificar os contratos relativamente à tipologia. Tal como se encontra representado na Tabela \ref{table:3}, os contratos podem ser classificados em duas grandes categorias: Bens e Serviços ou Empreitadas. Contudo, na coluna referente ao tipo de contrato não é feita esta discriminação. Assim, foi adicionada uma nova coluna a esta tabela, \textbf{tipo\_contrato}, que irá ter apenas uma das duas categorias anteriormente presentes para cada contrato. Para classificar cada contrato numa destas duas categorias fez-se uma busca da palavra \textit{obra} na string presente na coluna da tipologia de contrato. Se esta palavra estivesse contida na string, o contrato era classificado como Empreitada. Caso contrário, Bens e Serviços. 


\begin{figure}[H]
	\centering
	\includegraphics[width=0.9\textwidth]{imagens/rf1.png}
	\caption{Processo de classificação de ajustes diretos em regime geral}
	\label{}
\end{figure}




%\begin{lstlisting}[
%	language=SQL,
%	showspaces=false,
%	showstringspaces=false,
%	basicstyle=\ttfamily,
%	numbers=left,
%	numberstyle=\tiny,
%	commentstyle=\color{gray}, frame = single,	autogobble=true,
%	postbreak=\mbox{\textcolor{red}{$\hookrightarrow$}\space},
%	]
%	ALTER TABLE ajustesdiretos
%	ADD COLUMN artigo text;
%	
%	UPDATE ajustesdiretos
%	SET artigo = TRIM(SUBSTRING(fundamentacao 
%			FROM 1 FOR POSITION('º' IN fundamentacao)));
%\end{lstlisting}

%\begin{lstlisting}[
%	language=SQL,
%	showspaces=false,
%	showstringspaces=false,
%	basicstyle=\ttfamily,
%	numbers=left,
%	numberstyle=\tiny,
%	commentstyle=\color{gray}, frame = single,	autogobble=true,
%	postbreak=\mbox{\textcolor{red}{$\hookrightarrow$}\space},
%	]
%	ALTER TABLE ajustesdiretos
%	ADD COLUMN criterio text;
%	
%	UPDATE ajustesdiretos
%	SET criterio = CASE
%	WHEN artigo = 'Artigo 17.º' OR artigo = 'Artigo 18.º' 
%	OR artigo = 'Artigo 19.º' OR artigo = 'Artigo 20.º' 
%	OR artigo = 'Artigo 21.º' OR artigo = 'Artigo 22.º' 
%	THEN 'valor'
%	
%	ELSE 'material' 
%	END;
%\end{lstlisting}


Finda estas classificações quanto à tipologia e fundamentação, é possível identificar todos os contratos que não respeitem os valores apresentados na Tabela \ref{tab:4} a partir da seguinte \textit{query}: 

\begin{lstlisting}[
	language=SQL,
	showspaces=false,
	showstringspaces=false,
	basicstyle=\ttfamily,
	numbers=left,
	numberstyle=\tiny,
	commentstyle=\color{gray}, frame = single,	autogobble=true,
	breaklines=true,
	postbreak=\mbox{\textcolor{red}{$\hookrightarrow$}\space},
	]
	SELECT id
	FROM ajustesdiretos
	WHERE (criterio = 'valor' AND tipocontrato = 'Bens e Servicos' AND preco > 20000) OR (criterio = 'valor' AND tipocontrato = 'Empreitadas' AND preco > 30000);
\end{lstlisting}

Da totalidade de Ajustes Diretos em Regime Geral presentes na base de dados, 2.8\% (12828) destes foram adjudicados com um valor superior ao permitido por lei. Destes, 2.4\% (14812) dizem respeito a contratos de Bens e Serviços e 0.4\% (1984) a contratos de Empreitadas.


\begin{figure}[H]
	\centering
	\begin{minipage}{.45\linewidth}
		\includegraphics[width=\linewidth]{imagens/rf1/dist.png}
		\caption{Distribuição do número de contratos em inconformidade com o CCP}
		
	\end{minipage}
	\hfill
	\begin{minipage}{.5\linewidth}
		\includegraphics[width=\linewidth]{imagens/rf1/boxplot.png}
		\caption{Boxplot dos valores contratuais para os contratos em inconformidade com a lei}
		
	\end{minipage}
\end{figure}


Na Tabela \ref{tab:rf1} é possível observar quais são as entidades com o maior número de contratos celebrados em inconformidade com o CCP. 

\begin{table}[H]
	\centering
	\renewcommand{\arraystretch}{1.15}
	\setlength{\tabcolsep}{15pt}
	\resizebox{\textwidth}{!}{%
		\begin{tabular}{lc}
			\hline
			\multicolumn{1}{c}{\textbf{Entidade Adjudicante}}                                           & \textbf{Número de Contratos} \\ \hline
			Infraestruturas de Portugal, S. A.                                                          & 377                          \\
			Hospital de Santo Espírito da Ilha Terceira, E. P. E. R.                                    & 329                          \\
			Município de Ponta Delgada                                                                  & 204                          \\
			Instituto Português de Oncologia do Porto Francisco Gentil, E. P. E.                        & 203                          \\ \hline
			\multicolumn{1}{c}{\textbf{Entidade Vencedora}}                                             & \textbf{Número de Contratos} \\ \hline
			Meo - Serviços de Comunicações e Multimédia S.A                                             & 165                          \\
			Urbhorta - Construção, Gestão e Exploração de Projectos de Desenvolvimento Empresarial, EEM & 53                           \\
			Sotermáquinas-Sociedade Terceirense de Máquinas e Acessórios S.A                            & 53                          
		\end{tabular}%
	}
	\caption{Entidades, adjudicantes e vencedoras, com maior número de ajustes diretos celebrados em inconformidade com o CCP}
	\label{tab:rf1}
\end{table}

% ADCIONAR À SECÇÃO DOS ANEXOS
%\begin{figure}[H]
%	\centering
%	\includegraphics[width=\textwidth]{imagens/rf1/adjudicantes.png}
%	\caption{Entidades Adjudicantes com maior número de ajustes diretos celebrados em inconformidade com o CCP}
%	\label{}
%\end{figure}
%
%
%
%\begin{figure}[H]
%	\centering
%	\includegraphics[width=\textwidth]{imagens/rf1/vencedora.png}
%	\caption{Entidades Adjudicantes com maior número de ajustes diretos celebrados em inconformidade com o CCP}
%	\label{}
%\end{figure}



\section{R003 : Análise do Prazo de Apresentação de Propostas}

Um factor importante ao longo da publicitação de um concurso público é o prazo disponível para que diferentes entidades interessadas possam efetuar propostas. Uma das flags definidas pela OCDS incide precisamente sobre esta questão.

\Lemma{}
{Short or inadequate notice to bidders to submit expressions of interest or bids. Bid period is less than expected (usually $\leq$2 days). The indicator should be calculated grouping by procurement method used, since the minimum bidding period might vary depending on the method.  It is important to check in local regulations if there is a minimun period.}


De acordo com o artigo 135.º do CCP - ver Tabela \ref{table:3} - o prazo mínimo para apresentação de propostas num concurso público para Bens e Serviços não pode ser inferior a 6 dias e, no caso de se tratar de um procedimento de formação de um contrato de empreitada de obras públicas, a 14 dias. Assim, para a construção desta flag, foi criada uma coluna adicional na tabela \textit{concursos\_publicos} chamada \textbf{tipo\_contrato}, tal como no indicador RF1. De seguida, foi construída uma \textit{query} para determinar todos os contratos referentes a concursos públicos que não respeitam estas condições apresentadas na Tabela mencionada. \\


\begin{lstlisting}[
	language=SQL,
	showspaces=false,
	showstringspaces=false,
	basicstyle=\ttfamily,
	numbers=left,
	numberstyle=\tiny,
	commentstyle=\color{gray}, frame = single,
	autogobble=true,
	breaklines=true,
	postbreak=\mbox{\textcolor{red}{$\hookrightarrow$}\space},
	]
	SELECT contratos_basegov."id" 
	FROM contratos_basegov
	JOIN concursos_publicos ON contratos_basegov."id" = concursos_publicos."id"
	WHERE (concursos_publicos."tipo_contrato" = 'Empreitadas' AND contratos_basegov."anuncio_proposalDeadline" < 14 AND contratos_basegov."anuncio_proposalDeadline" > 0) 
	OR
	(concursos_publicos."tipo_contrato" = 'Bens e Servicos' AND contratos_basegov."anuncio_proposalDeadline" < 6 AND
	contratos_basegov."anuncio_proposalDeadline" > 0);
	
\end{lstlisting}


Foi incluída uma condição adicional na \textit{query} para selecionar os contratos com prazo de apresentação de propostas superior a 0 dias. Esta condição foi imposta pois verificou-se que existiam alguns contratos com prazos negativos. Tendo em conta a dimensão dos valores - sempre superiores a 10, em módulo - inferiu-se que seriam erros de preenchimento aquando da inserção do contrato no Portal BASE e, como tal, não foram considerados. 

\begin{figure}[H]
	\centering
	\begin{minipage}{.48\linewidth}
		\includegraphics[width=\linewidth]{imagens/r003/boxplot_prazos.png}
		\caption{Distribuição do prazo de apresentação das propostas em dias para as duas tipologias de contratos.}
		
	\end{minipage}
	\hfill
	\begin{minipage}{.48\linewidth}
		\includegraphics[width=\linewidth]{imagens/r003/main_cpvs.png}
		\caption{Principais divisões de CPV com maior número de contratos em trangressão.}
		
	\end{minipage}
\end{figure}

É na divisão da Construção Civil onde se verifica a existência de um maior incumprimento do prazo mínimo de dias para apresentação de propostas, atingindo uma percentagem de 70\% dos contratos inconformes. Perante a totalidade de concursos públicos celebrados referentes a construção civil, esta percentagem é de 13\%.


\subsection{R017 : Comparação do Preço Contratual com Preço Médio por CPV}


\Lemma{}
{
	The difference between an item value and its expected value is above a threshold. 
}

Para construção deste indicador foi calculado o preço médio por CPV. O conjunto de contratos foi agrupado pelos primeiros três dígitos - grupo - do CPV de forma a obter um maior nível de granularidade. Para cada um dos 302 grupos foi calculado o preço contratual médio, desvio padrão, mínimo, máximo e primeiro, segundo e terceiro quartis. O resultado obtido assemelha-se à seguinte tabela : 


\begin{table}[h!]
	
	\setlength\tabcolsep{1pt}
	\begin{tabular*}{\linewidth}{@{\extracolsep{\fill}} |c|c|c|c|c|c|c|c|c|c|}
		\hline
		\textbf{cpv3} & \textbf{preco\_total} & \textbf{count} & \textbf{mean} & \textbf{std} & \textbf{min} & \textbf{q1} & \textbf{q2} & \textbf{q3} & \textbf{max} \\ \hline
		641           & 6700942.8           & 57             & 117560.4    & 179755.1   & 1575         & 10639.2     & 60000       & 165892.2    & 885500       \\ \hline
	\end{tabular*}
	\caption{Primeira linha da tabela auxiliar construída}
	
\end{table}


Contudo, a partir da construção desta tabela, constatou-se que, certamente devido a erros de preenchimento na plataforma BaseGOV, existem inúmeros valores de preço contratuais com valores negativos ou com valor nulo. Desta forma, o valor médio calculado não corresponde ao verdadeiro valor médio e, como tal, a comparação que se pretendia efetuar na próxima etapa deste indicador entre o preço contratual e o preço médio contratual por grupo é inapropriada. 

\subsection{R018 : Análise de Contratos com uma entidade concorrente}

Um dos parâmetros importantes a ter em conta ao analisar um contrato público prende-se com o número de entidades concorrentes. A partir deste ponto, podem ser analisadas três situações distintas ao longo do processo contratual: 
quando apenas se candidatou uma entidade,quando se candidatou um \textit{elevado} número de entidades,quando se candidatou um \textit{baixo} número de entidades. Comecemos pela primeira, definida da seguinte forma pela OCDS:

\Lemma{}
{Single bid received : Tender featured a single bidder only }


Dado que uma das colunas da tabela diz respeito às entidades que se candidataram a um determinado concurso, pode ser feita uma análise do número de entidades concorrentes. Esta coluna é preenchida de uma forma análoga ao representado na Tabela \ref{tab:entsconc}. 


\begin{table}[H]
	\centering
	\begin{tabular}{|c|c|}
		\hline
		\textbf{ID}   & \textbf{EntidadesConcorrentes}                                                                                                      \\ \hline
		$\text{ID}_1$ & $\text{Entidade}\_i$ ( $\text{NIF}\_i$) $|||$ $\text{Entidade}\_j$ ( $\text{NIF}\_j$)                                               \\ \hline
		$\text{ID}_2$ & $\text{Entidade}\_k$ ( $\text{NIF}\_k$)                                                                                             \\ \hline
		$\dots$       & $\dots$                                                                                                                             \\ \hline
		$\text{ID}_n$ & $\text{Entidade}\_l$ ( $\text{NIF}\_l$) $|||$ $\text{Entidade}\_m$ ( $\text{NIF}\_m$) $|||$ $\text{Entidade}\_n$ ( $\text{NIF}\_n$) \\ \hline
	\end{tabular}
	\caption{Formato da coluna entidades\_concorrentes}
	\label{tab:entsconc}
\end{table}

Para cada entidade que concorre, é inserido o respetivo nome e número de identificação fiscal (NIF) e, na eventualidade do número de entidades concorrentes ser superior a 1, a separação entre as mesmas é feita recorrendo a um separador $|||$. Desta forma, foi necessário desenvolver uma \textit{query} que conta o número de elementos separados e insira o resultado desta contagem numa nova coluna criada, chamada \textit{nr\_entidadesconcorrentes}.


\begin{lstlisting}[
	language=SQL,
	showspaces=false,
	basicstyle=\ttfamily,
	numbers=left,
	numberstyle=\tiny,
	commentstyle=\color{gray}, frame = single,
	breaklines=true,
	autogobble =true,
	postbreak=\mbox{\textcolor{red}{$\hookrightarrow$}\space},
	]
	UPDATE concursospublicos
	SET nr_entidadesconcorrentes = ARRAY_LENGTH(STRING_TO_ARRAY( entidades_concorrentes, '|||'), 1) + 1;	
\end{lstlisting}

Através desta nova coluna, é possível selecionar todos os contratos que apenas contemplem uma entidade concorrente. 

\begin{lstlisting}[
	language=SQL,
	showspaces=false,
	basicstyle=\ttfamily,
	numbers=left,
	numberstyle=\tiny,
	commentstyle=\color{gray}, frame = single,
	breaklines=true,
	autogobble =true,
	postbreak=\mbox{\textcolor{red}{$\hookrightarrow$}\space},
	]
	SELECT contratos_basegov."id"
	FROM contratos_basegov
	JOIN concursos_publicos ON contratos_basegov."id" = concursos_publicos."id"
	WHERE concursos_publicos."nr_entidadesconcorrentes" = 1;
\end{lstlisting}


\begin{figure}[H]
	\centering
	\begin{minipage}{.48\linewidth}
		\includegraphics[width=\linewidth]{imagens/r018/main_cpvs.png}
		\caption{Divisões de CPV com maior número de contratos em inconformidade.}
	\end{minipage}
	\hfill
	\begin{minipage}{.49\linewidth}
		\includegraphics[width=\linewidth]{imagens/r018/prices.png}
		\caption{Valor adjudicado total para todos dos contratos em inconformidade para divisão de CPV.}
	\end{minipage}
\end{figure}




\begin{table}[H]
	\centering
	\renewcommand{\arraystretch}{1.15}
	\setlength{\tabcolsep}{15pt}
	\resizebox{\textwidth}{!} & \textbf{\begin{tabular}[c]{@{}c@{}}Número de \\ contratos total\end{tabular}} & \textbf{\%} & \textbf{\begin{tabular}[c]{@{}c@{}}Preço contratual\\ total\end{tabular}} \\ \hline
		33                   & 4444                                                                                & 27          & 29351                                                                         & 15.1        & 279.688.414,00€                                                           \\
		45                   & 3498                                                                                & 22          & 21525                                                                         & 16.3        & 2.103.391.451,00€                                                         \\
		15                   & 1625                                                                                & 10          & 5963                                                                          & 27.2        & 280.726.463,00€                                                           \\
		72                   & 1406                                                                                & 9           & 3739                                                                          & 37.6        & 65.760.473,00€                                                            \\
		34                   & 1162                                                                                & 7           & 4597                                                                          & 25.3        & 303.614.946,00€                                                           \\
		60                   & 1124                                                                                & 7           & 3512                                                                          & 32          & 351.923.997,00€                                                           \\
		50                   & 1069                                                                                & 6           & 3908                                                                          & 27.4        & 205.240.508,00€                                                           \\ \midrule
		\textbf{Total}       & 14328                                                                               & -           & 72595                                                                         & -           & -                                                                         \\ \hline
	\end{tabular}%
	}
	\caption{Descrição do número de contratos, total e inconformes, para as principais divisões de CPV.}
	\label{tab:rf18stats}
\end{table}







\subsection{R019 : Análise do Número de Entidades Concorrentes}

\Lemma{}
	{
	Low number of bidders for item and procuring entity. Number of bidders significantly less than average, based on prior similar contracts (for similar item or procuring entity) 
	}


Para a construção deste indicador, foi necessário construir uma tabela adicional. Para cada uma das 46 divisões de CPV, foi feita uma contagem de todos os contratos, calculado o número total de entidades concorrentes de entre todos os contratos e foram calculados os seguintes indicadores estatísticos : média, desvio-padrão, primeiro-quartil, segundo-quartil, terceiro-quartil, mínimo e máximo. Este procedimento foi feito através de um script \textit{python}, obtendo-se uma tabela do género : 

\begin{table}[H]
	\centering
	\begin{tabular}{|c|c|c|c|c|c|c|c|c|c|}
		\hline
		\textbf{cpv} & \textbf{nec\_t} & \textbf{count} & \textbf{mean} & \textbf{std} & \textbf{min} & \textbf{q1} & \textbf{q2} & \textbf{q3} & \textbf{max} \\ \hline
		98           & 1739            & 586            & 2.968         & 2.251        & 1            & 1           & 2           & 4           & 15           \\ \hline
	\end{tabular}
	\caption{}
\end{table}

sendo que \textbf{cpv} corresponde a cada umas das divisões de CPV, \textit{nec\_t} corresponde ao número total de entidades concorrentes em todos os concursos, \textit{count} é número total de contratos, \textbf{mean}, \textbf{std}, \textbf{min}, \textbf{q1}, \textbf{q2}, \textbf{q3} e \textbf{max} são, respetivamente, o valor médio, o desvio padrão, mínimo, primeiro quartil, segundo quartil, terceiro quartil e máximo para todos os contratos cujos primeiros dois digitos do CPV sejam iguais aos da primeira coluna. 

Com o resultado obtido, foi feita, numa primeira instância, uma representação gráfica dos valores da média, mediana e desvio-padrão, por CPV, a fim de ter uma ideia da dispersão - distância entre a média e o desvio padrão - e simetria da distribuição - distância entre média e mediana - dos preços contratuais.

\begin{figure}[!htbp]
	\centering
	\includegraphics[width=\textwidth]{imagens/mmdp.png}
	\caption{Comparação da Média, Mediana e Desvio Padrão do NEC em CP por CPV}
	\label{}
\end{figure}

Foi, também, feita uma representação gráfica do boxplot para cada umas divisões do CPV. 

%\begin{figure}[!htbp]
%	\centering
%	\includegraphics[width=.9\textwidth]{imagens/cpv1.png}
%	\caption{Boxplot referente ao número de entidades concorrentes em Concursos Públicos por CPV : I}
%	\label{}
%\end{figure}
%
%
%\begin{figure}[h!]
%	\centering
%	\includegraphics[width=.9\textwidth]{imagens/cpv2.png}
%	\caption{Boxplot referente ao número de entidades concorrentes em Concursos Públicos por CPV : II}
%	\label{}
%\end{figure}

\begin{figure}[H]
	\centering
	\includegraphics[width=.9\textwidth]{imagens/cpv3.png}
	\caption{Boxplot referente ao número de entidades concorrentes em Concursos Públicos por CPV }
	\label{}
\end{figure}


Assim, a fim de selecionarmos os IDs de todos os contratos que respeitem a definição enunciada anteriormente, foi construída a seguinte \textit{query} que retorna todos os contratos cujo número de entidades concorrentes seja inferior à metade do número de entidades concorrentes esperado.  

\begin{lstlisting}[
	language=SQL,
	showspaces=false,
	basicstyle=\ttfamily,
	numbers=left,
	numberstyle=\tiny,
	commentstyle=\color{gray}, frame = single,
	breaklines=true,
	autogobble =true,
	postbreak=\mbox{\textcolor{red}{$\hookrightarrow$}\space},
	]
	SELECT concursospublicos."id"
	FROM concursospublicos 
	JOIN cpv_stat ON concursospublicos."cpv2" = cpv_stat."cpv"
	WHERE concursospublicos."nr_entidadesconcorrentes" < 0.5 * cpv_stat."mean";
\end{lstlisting}

Os resultados obtidos, que podem ser observados na seguinte figura, mostram que existe uma forte predominância dos CPVs começados por 33 e 45, o que pode ser um indicador de que existe um elevado número de contratos com um número \textit{baixo} de entidades concorrentes. 

\begin{figure}[H]
	\centering
	\includegraphics[width=\textwidth]{imagens/r019.png}
	\caption{Grupos com maior número de contratos indiciados}
	\label{}
\end{figure}





\subsection{R031 : Análise da relação entre Preço Base e Preço contratual}

A primeira flag construída tem como objetivo comparar o preço base com o preço contratual.


Numa primeira instância, pegou-se num \textit{subset} de contratos da base de dados em PostgreSQL e guardou-se numa \textit{dataframe} em Python. 
Utilizando a função \textit{cpv}, obtiveram-se os id's dos contratos de um dos tipos de contrato desejados e para um determinado cpv. Numa fase inicial optou-se pelos ids concursos públicos para CPV's começados por 72. 


Para se poder comparar ambos os preços foram desenvolvidas algumas funções auxiliares.
Primeiramente, desenvolveu-se a função \textit{preco\_contratual1} que devolve, a partir do id de um anúncio, o valor do preço contratual desse mesmo anúncio.
A função \textit{preco\_contratual2} faz o mesma coisa mas para um tabela genérica pertencente à base de dados. A função \textit{preco\_contrato3} generaliza a primeira função, pois retorna um conjunto de preços contratuais referentes a um conjunto de ids de anúncios que leva como input. A função \textit{preco\_contratual4} generaliza a função anterior para qualquer tabela. Os precos contratuais são retornados no formato de array.


De seguida, foi feito o mesmo para o preço base. Foi utilizada a mesma abordagem e construíu-se a função \textit{preco\_base3} que retorna os valores dos preços base para um conjunto de id's. 


Assim, foi construída uma primeira versão para esta flag. 

\begin{verbatim}
	def redflag(pbase, pcontr, tol, ids, r, df):
\end{verbatim}

Definiram-se como parâmetros de entrada desta função o conjunto de preços bases - pbase - e preços contratuais - pcontr - associados a um determinado conjunto de ids - ids - que já conseguimos determinar usando as funções criadas anteriormente. É preciso fornecer uma tolerância - tol - que vai definir a diferença máxima permitida entre o preço base e o contratual e um racio máximo permitido entre o preco base e contratual. 


O objetivo desta flag é identificar todos os contratos cujo preço contratual pertença a um intervalo em torno do preço base, definido pelo parâmetro \textit{tol}.

\begin{figure}[H]
	\centering
	\includegraphics[width=0.7\textwidth]{imagens/pbasecontr.png}
	\caption{}
	\label{}
\end{figure}


%O preço base é conhecido à priori, por isso esta flag não tem muito valor por si só. Quando for acoplada com outras flags pode sugerir um comportamento suspeito. Também é preciso ter em conta como é que o preço base é calculado. Se o preço base for calculado por excesso e o preço contratual for próximo do preço base, pode haver corrupção tanto por parte da entidade adjudicante como da adjudicatária. 

Contudo, ao longo da construção desta flag verificou-se a existência de casos em que o preço base é significativamente superior ao preço contratual, chegando a atingir uma ordem de grandeza de $\approx 10^2$. Isto deve-se a dois fatores : o preço base definido é demasiado elevado comparativamente ao preço contratual ou o preço base diz respeito a casos em que a adjudicação de um determinado serviço/obra é feita por lotes.

\begin{table}[H]
	\centering
	\begin{tabular}{|c|c|c|c|c|}
		\hline
		\multicolumn{1}{|l|}{\textbf{Número de Anúncio}} & \multicolumn{1}{l|}{\textbf{ID}} & \multicolumn{1}{l|}{\textbf{Lote}} & \multicolumn{1}{l|}{\textbf{Preço Base}} & \multicolumn{1}{l|}{\textbf{Preço Contratual}}          \\ \hline
		& 10471090                         & 1                                  &                                          & \cellcolor[HTML]{F2FCFE}{\color[HTML]{555555} 77454.72} \\ \cline{2-3} \cline{5-5} 
		& 10471049                         & 2                                  &                                          & \cellcolor[HTML]{F2FCFE}{\color[HTML]{555555} 85365.84} \\ \cline{2-3} \cline{5-5} 
		\multirow{-3}{*}{11171/2023}                     & 10470972                         & 3                                  & \multirow{-3}{*}{373860.48}              & \cellcolor[HTML]{F2FCFE}{\color[HTML]{555555} 89886.72} \\ \hline
	\end{tabular}
	\caption{Exemplo de contrato adjudicado por lotes}
\end{table}


Tendo em conta que na base de dados não é disponibilizado o preço base para cada lote, foi utilizada a seguinte abordagem. É de salientar que no caso de adjudicação por lotes, o número de anúncio é igual para cada um dos lotes, mas cada lote tem um id diferente. 

Sempre que o rácio $\text{preçobase} / \text{preçocontratual}$, para um certo contrato, é superior a determinado limite, o respetivo id é guardado. A partir do id do contrato obtém-se o número do anúncio e, para todos os contratos com o mesmo número de anúncio, são somados os preços contratuais e o resultado final é comparado com o preço base. 

Por fim, o resultado final é um conjunto de id's na forma de tuplo que respeita as condições anteriormente mencionadas.




%\section{Análise dos Concursos Públicos}
%
%Aqui só foram usadas funções ja definidas. Criaram-se variáveis que guardam o valor do preço base e contratual dos concursos. Deu-se isso como input à funcao \textit{redflag}, além dos outros que são definidos por mim, e obtiveram-se as redflags. Representou-se um barplot dos precos contratuais em cima dos preços base para verificar a diferença entre os mesmos. 


%\section{Análise dos Ajustes Diretos em Regime Geral}
%
%Aqui, novamente, foram utilizadas as funções já criadas anteriormente para obter uma dataframe com os contratos referentes a ajustes diretos e os ids associados. Foi feito um sumário estatístico dos valores dos preços contratuais e um histograma e um boxplot. Mas como tem outliers, vê-se mal. 
%Ordenei os ajustes diretos por fundamentação. Não nenhum campo vazio. 
%Com isto tudo ja se podem calcular as flags usando a função \textit{redflag2}. 
%É interessante saber qual a proporção de atividades suspeitas por entidade adjudicante e entidade adjudicatária. Para isso, calculou-se primeiro o número de contratos suspeitos e respetiva percentagem. 
%Para conseguir analisar foi preciso separar as colunas das entidades adjudicantes e adjudicatárias - que estavam no formato Entidade(NIF)(URL) - em três colunas distintas para cada uma das duas entidades. Isto é necessário para filtrar os contratos pelo NIF porque diferentes entidades podem ter o mesmo NIF. 
%Depois disso, ordenou-se por ordem decrescente de ajustes diretos realizados os NIFS das empresas. Uma das listas ordenadas só para as entidades adjudicantes, outra para as entidades adjudicatárias. 
%Para esta analise foi necessário criar duas funções que retornem os ids dos ajustes diretos a partir do NIF da empresa. Queremos todos os ajustes diretos celebrados para um determinada entidade a partir do seu NIF. Isso foi feito a partir das funções \textit{e\_adjudicante} e \textit{e\_adjudicataria}. 
%De seguida, foi então calculado para cada NIF o número de ajustes diretos realizados usando uma das funções anteriores e os respetivos ID's. Para cada NIF, foi comparado cada um dos nos id's com a lsita de id's dado pela função \textit{redflags2} e feito o rácio para ver a percentagem de contratos suspeitos. 
%Agora seria interessante verificar se existem, dentro destas empresas, subgrupos entre entidades adjudicantes e adjudicatárias. 

\subsection{R051 : Alta concentração de mercado}

\Lemma{}
{A small number of companies concentrate a high share of contracts in a particular market.}



\subsection{RF2 : Comparação entre Preço Contratual e Preço Total Efetivo}

Existem situações em que o valor contratual celebrado é alterado após celebração do contrato, seja por não existir necessidade de usar tantos recursos como a situação oposta. Estes valores são guardados na coluna referefente ao Preço Total Efetivo. Nesses casos, é simples verificar quais são os contratos em que ocorre esta situação através de uma \textit{query}: 

\begin{lstlisting}[
	language=SQL,
	showspaces=false,
	basicstyle=\ttfamily,
	numbers=left,
	numberstyle=\tiny,
	commentstyle=\color{gray},	frame=single,
	breaklines=true,
	autogobble = true,
	postbreak=\mbox{\textcolor{red}{$\hookrightarrow$}\space},
	]
	SELECT contratos_basegov."id", 
		preco_contratual, 
		contratos_basegov."totalEffectivePrice", 
		contratos_basegov."totalEffectivePrice"/preco_contratual AS racio
	FROM contratos_basegov 
	WHERE contratos_basegov."totalEffectivePrice" > 0 AND 
		ABS(contratos_basegov."totalEffectivePrice" - preco_contratual) > 0 AND 
		preco_contratual > 0;
\end{lstlisting}


\begin{figure}[H]
	\centering
	\includegraphics[width=0.5\textwidth]{imagens/rf2/main_cpvs.png}
	\caption{Grupos com maior número de contratos indiciados}
	\label{}
\end{figure}



\subsection{RF3: Análise da data de publicação do anúncio}

Adicionalmente, foi feita uma análise do dia de publicação do anúncio em Diário da República. O objetivo deste indicador é verificar se existem anúncios publicados em dias não convencionais - feriados nacionais - na tentativa de os fazer passar despercebidos. Não existe nenhuma ocorrência deste indicador nos contratos presentes na base de dados durante o período de tempo estudado. 




\section{Processo de Automação}









%\chapter{Conclusions and Future Work}
%\section{Conclusions}
	In this work, we synthesized Ca$_{3}$Mn$_{2-x}$Ti$_{x}$O$_{7}$ $n=2$ Ruddlesden-Popper samples with various compositions $(x=0.1,\,0.25,\,0.3)$ via a common solid-state reaction method and studied its structural phase transitions and properties through different methods: X-Ray Powder Diffraction (XRPD), Raman Spectroscopy, Perturbed Angular Correlation (PAC), and Density Functional Theory (DFT) simulations, with special focus on the $x=0.25$ (C$_{2}$MTO) system.
	
	After the last annealing of the synthesis process, Rietveld refinements on the XRPD data were performed to check the purity of the formed $n=2$ phases. We observed an approximate $12\%$ presence of Ca$_{2}$Mn$_{0.875}$Ti$_{0.125}$O$_{4}$ ($n=1$ C$_{1}$MTO) in C$_{2}$MTO, however with $a-$ and $b-$lattice\-parameters considerably lower (by $\approx 0.6\AA$) of what can be found in the literature for the undoped Ca$_{2}$MnO$_{4}$ system.
	
	Raman and PAC spectroscopies, respectively in the $93-653$K and $74-1224$K temperature ranges, revealed a first order structural phase transition from what appears to be a the ferroelectric $A2_{1}am$ (s.g. 36) phase to an intermediate and paraelectric $Acaa$ (s.g. 68) symmetry within a $300-550$K temperature range, where both structural phases coexist. Thus, when comparing CMTO to the undoped Ca$_{3}$Mn$_{2}$O$_{7}$ (C$_{2}$MO), we report an increase in the percentage of the ferroelectric phase at room temperature by introducing the $x=0.25$ Ti$^{4+}$ doping. However, our results point to the possibility of the ground state not corresponding to the $A2_{1}am$ symmetry, but to one with inequivalent A-sites at the rock-salt of the RP system. PAC spectroscopy also revealed a second order structural phase transition from the intermediate orthorhombic $Acaa$ (s.g. 68) to the tetragonal $I4/mmm$ (s.g. 139) at 1150K.
	
	We came across an unexpected result at temperatures below 300K, where we clearly detected the probing of a second local environment, albeit without a clear physical origin. Interestingly, PAC measurements did not found evidence of the C$_{1}$MTO phase observed from XRPD refinements, and we believe this suggests that the $A2_{1}am$ space group might not correspond to the ground state symmetry for every composition of Ca$_{3}$Mn$_{2-x}$Ti$_{x}$O$_{7}$ mixed B-site systems. Phonon calculations on these doped systems in the $A2_{1}am$ symmetry could reveal if there are any lattice instabilities, shedding a light on whether the $A2_{1}am$ still corresponds to the ground state when Ti is added. 
	
	The C$_{2}$MO system was studied with DFT in it's ground state $A2_{1}am$ symmetry. We determined a Hubbard-$U$ correction by first principles $(U=6\text{ eV})$ to properly describe the Mn$-3d$ states, and then performed a series of structural and electronic properties calculations with our \textit{ab-initio} value and with $U=0$ and $U=3.9 \text{ eV}$, a value commonly found in the literature. We determined the cell parameters and bulk moduli for the mentioned $U$ values as well as multiple electronic properties, such as the projected density-of-states (PDOS), Bader charges, and the Electric Field Gradient at multiple sites. When we manage to define and adjust the most adequate parameters to compute the EFGs for this system, we intend to probe the Ti$^{4+}$ doped Ca$_{3}$Mn$_{2-x}$Ti$_{x}$O$_{7}$ structures, and analyse how the EFGs vary as a function of doping concentration.

\section{Future Work}
	In the future, we intend to deepen the discussion at several stages of this system's characterization. Regarding the observation of a C$_{1}$MTO phase impurity in XRPD experiments, and the underestimated lattice parameters, we intend to pursue a new refinement considering also the intermediate $Acaa$ phase. To complement this measurements, XAS and XPS spectroscopies would help us to confirm the oxidation states of Mn and Ti in the sample, to account for some deviations in the XRPD patterns. Additionally, neutron diffraction would confirm the structural and magnetic structural properties of these samples.
	
	More complex Raman Spectroscopy setups could be pursued to study the evolution of the softer modes that drive the $A2_1am-Acaa$ structural phase transition, and DFT calculations could assign the peak positions for each mode in the doped C$_{2}$MTO structure.
	
	The interpretation PAC results below 300K is still open to discussion, and we'll be pursing other methods to clarify them. Doped CMTO DFT calculations of the EFG parameters is a work in progress and the study of other doped systems is also scheduled, to later compare with undoped systems. In fact, $n=1$ Ca$_{2}$(Mn,Ti)O$_{4}$ RP compounds have already been measured by PAC and will be analysed in the future. Future phonon calculations on these B-site mixed systems may become a key point in understanding these systems.




% +++++++++++++++++++++++++++++++++++++++++++++++++++++++++++++++++++++++++++++++++
%									Bibliografia
% +++++++++++++++++++++++++++++++++++++++++++++++++++++++++++++++++++++++++++++++++
\clearemptydoublepage
\begin{footnotesize}
	\renewcommand{\bibname}{Bibliography}
	\addcontentsline{toc}{chapter}{Bibliography}
	\bibliographystyle{biolett}
	\bibliography{../thesis-bib}
	\bibliography{../thesis-bib-aux}
\end{footnotesize}






% +++++++++++++++++++++++++++++++++++++++++++++++++++++++++++++++++++++++++++++++++
%									  Apêndice
% +++++++++++++++++++++++++++++++++++++++++++++++++++++++++++++++++++++++++++++++++


% +++++++++++++++++++++++++++++++++++++++++++++++++++++++++++++++++++++++++++++++++
% 	resets chapter numbering, uses letters for chapter numbers and doesn't 
%	fiddle with page numbering;
% +++++++++++++++++++++++++++++++++++++++++++++++++++++++++++++++++++++++++++++++++
\appendix


\addcontentsline{toc}{chapter}{APPENDICES}
\chapter{}

%\begin{longtblr}[
%	caption = {Indicadores Estatísticos referentes ao número de entidades concorrentes em concursos públicos por CPV : R019},
%	label = {tab:test},
%	]{
%		colspec = {|c|c|c|c|c|c|c|c|c|c|},
%		rowhead = 1,
%		hlines,
%		%row{even} = {gray9},
%		%row{1} = {olive9},
%	} 
%		\textbf{CPV} & \textbf{NEC Total} & \textbf{Nº Concursos} & \textbf{Média}     & \textbf{$\sigma$}  & \textbf{Mínimo} & \textbf{Q1} & \textbf{Q2} & \textbf{Q3} & \textbf{Máximo} \\  
%		98           & 1739               & 586                   & 2.967              & 2.251 & 1               & 1           & 2           & 4           & 15              \\ 
%		92           & 2127               & 680                   & 3.127 			   & 2.501 & 1               & 1           & 3           & 4           & 18              \\ 
%		90           & 27135              & 4031                  & 6.731 			   & 4.878 & 1               & 3           & 6           & 9           & 27              \\ 
%		85           & 4497               & 1231                  & 3.653 			   & 2.810 & 1               & 1           & 3           & 5           & 18              \\ 
%		80           & 1812               & 495                   & 3.660 			   & 3.362 & 1               & 1           & 3           & 5           & 32              \\ 
%		79           & 20469              & 3678                  & 5.565 			   & 4.006 & 1               & 2           & 5           & 8           & 30              \\ 
%		77           & 13301              & 1528                  & 8.704 			   & 6.449 & 1               & 4           & 8           & 12          & 33              \\ 
%		76           & 64                 & 28                    & 2.285 			   & 1.560 & 1               & 1           & 1.5         & 3.25        & 6               \\ 
%		75           & 606                & 134                   & 4.522 			   & 4.179 & 1               & 1           & 3           & 6           & 21              \\ 
%		73           & 598                & 121                   & 4.942 			   & 6.532 & 1               & 1           & 3           & 6           & 32              \\ 
%		72           & 12662              & 3080                  & 4.111 			   & 4.807 & 1               & 1           & 2           & 5           & 32              \\ 
%		71           & 27396              & 3430                  & 7.987 			   & 7.420 & 1               & 3           & 6           & 11          & 54              \\ 
%		70           & 111                & 40                    & 2.775 			   & 3.050 & 1               & 1           & 1           & 2.25        & 14              \\ 
%		66           & 11105              & 2792                  & 3.977 			   & 2.132 & 1               & 2           & 4           & 6           & 11              \\ 
%		65           & 1476               & 339                   & 4.353 			   & 2.864 & 1               & 2           & 4           & 6           & 12              \\ 
%		64           & 2379               & 962                   & 2.472 			   & 1.588 & 1               & 2           & 2           & 3           & 29              \\ 
%		63           & 7115               & 831                   & 8.561 			   & 6.509 & 1               & 2           & 8           & 14          & 24              \\ 
%		60           & 12889              & 2930                  & 4.398 			   & 3.443 & 1               & 1           & 4           & 6           & 27              \\ 
%		55           & 7394               & 1365                  & 5.416 			   & 7.699 & 1               & 2           & 4           & 5           & 46              \\ 
%		51           & 475                & 193                   & 2.461 			   & 1.851 & 1               & 1           & 2           & 3           & 12              \\ 
%		50           & 17208              & 3187                  & 5.399 			   & 7.272 & 1               & 2           & 3           & 6           & 40              \\ 
%		48           & 5520               & 1717                  & 3.214 			   & 3.300 & 1               & 1           & 2           & 4           & 19              \\ 
%		45           & 103881             & 17913                 & 5.799 			   & 4.428 & 1               & 3           & 5           & 8           & 38              \\ 
%		44           & 8598               & 2123                  & 4.049 			   & 2.879 & 1               & 2           & 4           & 5.5         & 20              \\ 
%		43           & 988                & 239                   & 4.133 			   & 3.556 & 1               & 2           & 3           & 5           & 15              \\ 
%		42           & 4178               & 989                   & 4.224 			   & 3.524 & 1               & 2           & 3           & 5           & 22              \\ 
%		41           & 14                 & 3                     & 4.666 			   & 3.511 & 1               & 3           & 5           & 6.5         & 8               \\ 
%		39           & 18888              & 2673                  & 7.066 			   & 5.431 & 1               & 3           & 6           & 10          & 37              \\ 
%		38           & 9390               & 1254                  & 7.488 			   & 7.428 & 1               & 2           & 4           & 11          & 33              \\ 
%		37           & 777                & 195                   & 3.984 			   & 2.512 & 1               & 2           & 4           & 5           & 12              \\ 
%		35           & 6221               & 876                   & 7.101 			   & 7.613 & 1               & 3           & 5           & 9           & 47              \\ 
%		34           & 13432              & 3736                  & 3.595 			   & 2.682 & 1               & 2           & 3           & 5           & 19              \\ 
%		33           & 265993             & 23180                 & 11.47 			   & 9.472 & 1               & 4           & 9           & 16          & 65              \\ 
%		32           & 4316               & 1008                  & 4.281 			   & 3.432 & 1               & 2           & 3           & 6           & 21              \\ 
%		31           & 3632               & 748                   & 4.855 			   & 3.633 & 1               & 2           & 4           & 6           & 27              \\ 
%		30           & 30955              & 4393                  & 7.046 			   & 4.863 & 1               & 3           & 6           & 10          & 21              \\ 
%		24           & 3971               & 846                   & 4.693 			   & 5.198 & 1               & 2           & 3           & 5           & 39              \\ 
%		22           & 2196               & 544                   & 4.036 			   & 3.075 & 1               & 2           & 3           & 5           & 25              \\ 
%		19           & 1093               & 292                   & 3.743 			   & 2.160 & 1               & 2           & 4           & 5           & 13              \\ 
%		18           & 8730               & 982                   & 8.890 			   & 8.402 & 1               & 4           & 7           & 12          & 53              \\ 
%		16           & 788                & 190                   & 4.147 			   & 2.954 & 1               & 2           & 3           & 6           & 13              \\ 
%		15           & 26523              & 4749                  & 5.584 			   & 5.137 & 1               & 2           & 4           & 8           & 30              \\ 
%		14           & 752                & 245                   & 3.069 			   & 1.837 & 1               & 2           & 3           & 4           & 10              \\ 
%		09           & 9086               & 2392                  & 3.798 			   & 2.410 & 1               & 2           & 3           & 5           & 27              \\ 
%		03           & 2915               & 661                   & 4.409 			   & 4.044 & 1               & 2           & 3           & 6           & 30              \\ 
%		00           & 136                & 107                   & 1.271 			   & 1.120 & 1               & 1           & 1           & 1           & 6               \\ 
%\end{longtblr}


%\begin{longtable}{@{} *{11}{>{\centering\arraybackslash}X} @{}} % Adjust the number of columns
%	\caption{Indicadores Estatísticos referentes ao número de entidades concorrentes em concursos públicos por CPV e por Tipo de Contrato : R019}
%	\label{tab:test} 
%	\toprule
%	\textbf{CPV} & \textbf{Tipo de Contrato} & \textbf{NEC Total} & \textbf{Nº Concursos} & \textbf{Média} & \textbf{$\sigma$} & \textbf{Min} & \textbf{Q1} & \textbf{Q2} & \textbf{Q3} & \textbf{Máx} \\
%	\midrule
%	\endfirsthead
%	
%	\multicolumn{11}{c}%
%	{\tablename\ \thetable\ -- \textit{Continued from previous page}} \\
%	\toprule
%	\textbf{CPV} & \textbf{Tipo de Contrato} & \textbf{NEC Total} & \textbf{Nº Concursos} & \textbf{Média} & \textbf{$\sigma$} & \textbf{Min} & \textbf{Q1} & \textbf{Q2} & \textbf{Q3} & \textbf{Máx} \\
%	\midrule
%	\endhead
%	
%	\midrule \multicolumn{11}{r}{\textit{Continued on next page}} \\
%	\endfoot
%	
%	\bottomrule
%	\endlastfoot
%    
%	   
%			98  & Bens e Serviços                       & 1723      & 581          & 2.966  & 2.244    & 1   & 1    & 2   & 4    & 15  \\  
%			98  & Empreitadas                           & 4         & 2            & 2.000  & 1.414    & 1   & 1.5  & 2   & 2.5  & 3   \\  
%			98  & Outro                                 & 12        & 3            & 4.000  & 4.359    & 1   & 1.5  & 2   & 5.5  & 9   \\  
%			92  & Bens e Serviços                       & 2126      & 679          & 3.131  & 2.502    & 1   & 1    & 3   & 4    & 18  \\  
%			92  & Empreitadas                           & 1         & 1            & 1.000  & NaN      & 1   & 1    & 1   & 1    & 1   \\  
%			90  & Bens e Serviços                       & 27135     & 4031         & 6.732  & 4.878    & 1   & 3    & 6   & 9    & 27  \\  
%			85  & Bens e Serviços                       & 4496      & 1230         & 3.655  & 2.811    & 1   & 1    & 3   & 5    & 18  \\  
%			85  & Empreitadas                           & 1         & 1            & 1.000  & NaN      & 1   & 1    & 1   & 1    & 1   \\  
%			80  & Bens e Serviços                       & 1812      & 495          & 3.661  & 3.363    & 1   & 1    & 3   & 5    & 32  \\  
%			79  & Bens e Serviços                       & 20464     & 3677         & 5.565  & 4.007    & 1   & 2    & 5   & 8    & 30  \\  
%			79  & Outro                                 & 5         & 1            & 5.000  & NaN      & 5   & 5    & 5   & 5    & 5   \\  
%			77  & Bens e Serviços                       & 13300     & 1527         & 8.710  & 6.448    & 1   & 4    & 8   & 12   & 33  \\  
%			77  & Empreitadas                           & 1         & 1            & 1.000  & NaN      & 1   & 1    & 1   & 1    & 1   \\  
%			76  & Bens e Serviços                       & 64        & 28           & 2.286  & 1.560    & 1   & 1    & 1.5 & 3.25 & 6   \\  
%			75  & Bens e Serviços                       & 606       & 134          & 4.522  & 4.180    & 1   & 1    & 3   & 6    & 21  \\  
%			73  & Bens e Serviços                       & 598       & 121          & 4.942  & 6.532    & 1   & 1    & 3   & 6    & 32  \\  
%			72  & Bens e Serviços                       & 12662     & 3080         & 4.111  & 4.808    & 1   & 1    & 2   & 5    & 32  \\  
%			71  & Bens e Serviços                       & 27341     & 3418         & 7.999  & 7.428    & 1   & 3    & 6   & 11   & 54  \\  
%			71  & Empreitadas                           & 55        & 12           & 4.583  & 3.825    & 1   & 1.75 & 4   & 4.75 & 14  \\  
%			70  & Bens e Serviços                       & 111       & 40           & 2.775  & 3.051    & 1   & 1    & 1   & 2.25 & 14  \\  
%			66  & Bens e Serviços                       & 11105     & 2792         & 3.977  & 2.132    & 1   & 2    & 4   & 6    & 11  \\  
%			65  & Bens e Serviços                       & 1476      & 339          & 4.354  & 2.865    & 1   & 2    & 4   & 6    & 12  \\  
%			64  & Bens e Serviços                       & 2379      & 962          & 2.473  & 1.589    & 1   & 2    & 2   & 3    & 29  \\  
%			63  & Bens e Serviços                       & 7115      & 831          & 8.562  & 6.510    & 1   & 2    & 8   & 14   & 24  \\  
%			60  & Bens e Serviços                       & 12886     & 2929         & 4.399  & 3.444    & 1   & 1    & 4   & 6    & 27  \\  
%			60  & Outro                                 & 3         & 1            & 3.000  & NaN      & 3   & 3    & 3   & 3    & 3   \\  
%			55  & Bens e Serviços                       & 7376      & 1356         & 5.440  & 7.719    & 1   & 2    & 4   & 5    & 46  \\  
%			55  & Outro                                 & 18        & 9            & 2.000  & 1.658    & 1   & 1    & 1   & 2    & 6   \\  
%			51  & Bens e Serviços                       & 475       & 193          & 2.461  & 1.851    & 1   & 1    & 2   & 3    & 12  \\  
%			50  & Bens e Serviços                       & 17153     & 3172         & 5.408  & 7.286    & 1   & 2    & 3   & 6    & 40  \\  
%			50  & Empreitadas                           & 55        & 15           & 3.667  & 3.155    & 1   & 1    & 2   & 6.5  & 10  \\  
%			48  & Bens e Serviços                       & 5520      & 1717         & 3.215  & 3.301    & 1   & 1    & 2   & 4    & 19  \\  
%			45  & Empreitadas                           & 103881    & 17913        & 5.799  & 4.429    & 1   & 3    & 5   & 8    & 38  \\  
%			44  & Bens e Serviços                       & 8598      & 2123         & 4.050  & 2.880    & 1   & 2    & 4   & 5.5  & 20  \\  
%			43  & Bens e Serviços                       & 988       & 239          & 4.134  & 3.556    & 1   & 2    & 3   & 5    & 15  \\  
%			42  & Bens e Serviços                       & 4164      & 985          & 4.227  & 3.531    & 1   & 2    & 3   & 6    & 22  \\  
%			42  & Empreitadas                           & 14        & 4            & 3.500  & 1.000    & 2   & 3.5  & 4   & 4    & 4   \\  
%			41  & Bens e Serviços                       & 14        & 3            & 4.667  & 3.512    & 1   & 3    & 5   & 6.5  & 8   \\  
%			39  & Bens e Serviços                       & 18881     & 2669         & 7.074  & 5.432    & 1   & 3    & 6   & 10   & 37  \\  
%			39  & Empreitadas                           & 7         & 4            & 1.750  & 0.957    & 1   & 1    & 1.5 & 2.25 & 3   \\  
%			38  & Bens e Serviços                       & 9387      & 1251         & 7.504  & 7.431    & 1   & 2    & 4   & 11   & 33  \\  
%			38  & Empreitadas                           & 3         & 3            & 1.000  & 0.000    & 1   & 1    & 1   & 1    & 1   \\  
%			37  & Bens e Serviços                       & 777       & 195          & 3.985  & 2.513    & 1   & 2    & 4   & 5    & 12  \\  
%			35  & Bens e Serviços                       & 6209      & 875          & 7.096  & 7.616    & 1   & 3    & 5   & 9    & 47  \\  
%			35  & Empreitadas                           & 12        & 1            & 12.000 & NaN      & 12  & 12   & 12  & 12   & 12  \\  
%			34  & Bens e Serviços                       & 13336     & 3730         & 3.575  & 2.628    & 1   & 2    & 3   & 5    & 17  \\  
%			34  & Empreitadas                           & 96        & 6            & 16.000 & 6.419    & 3   & 17.5 & 19  & 19   & 19  \\ 
%			33  & Bens e Serviços                       & 265988    & 23179        & 11.475 & 9.473    & 1   & 4    & 9   & 16   & 65  \\  
%			33  & Empreitadas                           & 5         & 1            & 5.000  & NaN      & 5   & 5    & 5   & 5    & 5   \\  
%			32  & Bens e Serviços                       & 4292      & 1005         & 4.271  & 3.432    & 1   & 2    & 3   & 6    & 21  \\  
%			32  & Empreitadas                           & 24        & 3            & 8.000  & 0.000    & 8   & 8    & 8   & 8    & 8   \\  
%			31  & Bens e Serviços                       & 3632      & 748          & 4.856  & 3.633    & 1   & 2    & 4   & 6    & 27  \\  
%			30  & Bens e Serviços                       & 30955     & 4393         & 7.046  & 4.863    & 1   & 3    & 6   & 10   & 21  \\  
%			24  & Bens e Serviços                       & 3971      & 846          & 4.694  & 5.198    & 1   & 2    & 3   & 5    & 39  \\  
%			22  & Bens e Serviços                       & 2196      & 544          & 4.037  & 3.075    & 1   & 2    & 3   & 5    & 25  \\  
%			19  & Bens e Serviços                       & 1093      & 292          & 3.743  & 2.160    & 1   & 2    & 4   & 5    & 13  \\  
%			18  & Bens e Serviços                       & 8730      & 982          & 8.890  & 8.403    & 1   & 4    & 7   & 12   & 53  \\  
%			16  & Bens e Serviços                       & 788       & 190          & 4.147  & 2.955    & 1   & 2    & 3   & 6    & 13  \\  
%			15  & Bens e Serviços                       & 26523     & 4749         & 5.585  & 5.137    & 1   & 2    & 4   & 8    & 30  \\  
%			14  & Bens e Serviços                       & 752       & 245          & 3.069  & 1.837    & 1   & 2    & 3   & 4    & 10  \\  
%			09  & Bens e Serviços                       & 9077      & 2390         & 3.798  & 2.412    & 1   & 2    & 3   & 5    & 27  \\  
%			9   & Empreitadas                           & 9         & 2            & 4.500  & 0.707    & 4   & 4.25 & 4.5 & 4.75 & 5   \\  
%			03  & Bens e Serviços                       & 2915      & 661          & 4.410  & 4.044    & 1   & 2    & 3   & 6    & 30  \\  
%			00  & Outro                                 & 136       & 107          & 1.271  & 1.121    & 1   & 1    & 1   & 1    & 6   \\  
%\end{longtblr}




\begin{code}[caption={Simulação em Python para calcular percentagem de outliers numa distribuição normal.},captionpos=b, label=lst:normalcode,
	language=python,
	showspaces=false,
	showstringspaces=false,
	basicstyle=\ttfamily,
	numbers=left,
	numberstyle=\tiny,
	commentstyle=\color{gray}, 
	frame=single,
	autogobble=true,
	breaklines=true,
	postbreak=\mbox{\textcolor{red}{$\hookrightarrow$}\space},
	]
	np.random.seed(5)	
	upper_size = list()
	lower_size = list()
	
	for i in range(10000):
	
		# Dimensao da amostra
		n = 2000
		
		# Amostra Aleatoria (Distribuicao Normal: Media = 0 | Var = 1)
		amostra = np.random.normal(loc=0, scale=1, size=n)
		
		# Calcular primeiro e terceiro quartil
		q1 = np.percentile(amostra, 25)
		q3 = np.percentile(amostra, 75)
		
		# Calcular extremidade/whisker superior e inferior
		upper_whisker = q3 + 1.5 * (q3 - q1)
		lower_whisker = q1 - 1.5 * (q3 - q1)
		
		# Localizar pontos pertencentes alem-fronteira
		upper_points = np.where(amostra > upper_whisker)[0]
		lower_points = np.where(amostra < lower_whisker)[0]
		
		# Calcular a percentagem de pontos alem-fronteira
		upper_size.append(len(upper_points) / n * 100)
		lower_size.append(len(lower_points) / n * 100)
	
\end{code}


\begin{figure}[H]
	\centering
	\includegraphics[width=\textwidth]{imagens/stats/histos_barreiras.png}
	\caption{Distribuição da percentagem de outliers, para ambos os lados da distribuição, de 10000 amostras geradas a partir de uma distribuição normal.}
	\label{fig:normalouts}
\end{figure}







\chapter{Construção das \textit{Red Flags}}

\begin{code}[caption={\textit{Query} em PostgreSQL para detetar contratos cuja data de publicação do anúncio em Diário da República seja feito num domingo ou feriado nacional.},captionpos=b, label=lst:rf3code,
	language=SQL,
	showspaces=false,
	basicstyle=\ttfamily,
	numbers=left,
	numberstyle=\tiny,
	commentstyle=\color{gray},	frame=single,
	breaklines=true,
	autogobble = true,
	postbreak=\mbox{\textcolor{red}{$\hookrightarrow$}\space},
	]
	WITH sundays AS (
		SELECT generate_series::date AS sunday_date
		FROM generate_series('2018-01-01'::date, '2024-05-01'::date, '1 week'::interval)
		WHERE EXTRACT(DOW FROM generate_series) = 0)
	SELECT contratos_basegov."id"
	FROM contratos_basegov
	JOIN concursos_publicos ON contratos_basegov."id" = concursos_publicos."id"
	WHERE tipo_procedimento = 'Concurso público' 
		AND EXTRACT(YEAR FROM contratos_basegov."data_publicacao") >= 2018 
		AND contratos_basegov."data_celebracao" <= '2024-05-01'
		AND (TO_CHAR(contratos_basegov."anuncio_drPublicationDate", 'MM-DD') IN 
		('01-01','04-25','05-01','06-10','08-15','10-05','11-01','12-01','12-08','12-25')
		OR contratos_basegov."anuncio_drPublicationDate" IN (SELECT sunday_date FROM sundays));
	
\end{code}


\begin{figure}[H]
	\centering
	\includegraphics[width=\textwidth]{imagens/r019_hist1.png}
\end{figure}

\begin{figure}[H]
	\centering
	\includegraphics[width=\textwidth]{imagens/r019_hist2.png}
\end{figure}

\begin{figure}[H]
	\centering
	\includegraphics[width=\textwidth]{imagens/r019_hist3.png}
	\caption{Distribuição do número de entidades concorrentes por cada umas das 45 divisões de CPV.}
	\label{fig:distisnec}
\end{figure}

\begin{figure}[H]
	\centering
	\includegraphics[width=\textwidth]{imagens/cpv3.png}
	\caption{Boxplot do número de entidades concorrentes por cada umas das 45 divisões de CPV.}
	\label{fig:boxplotsnec}
\end{figure}


\begin{code}[caption={Código em Python referente à flag R031.},captionpos=b,label=lst:r31,
	language=python,
	showspaces=false,
	showstringspaces=false,
	basicstyle=\ttfamily,
	numbers=left,
	numberstyle=\tiny,
	commentstyle=\color{gray}, 
	frame=single,
	autogobble=true,
	breaklines=true,
	postbreak=\mbox{\textcolor{red}{$\hookrightarrow$}\space},
	]
	def flagr31(tol):
	"""
	Parametro de entrada:
	tol(float): número entre 0 e 1
	
	return: tuplo de IDs que respeitam as condicoes impostas
	"""
	
	cur = conn.cursor()
	cur.execute('''
	SELECT id
	FROM contratos_basegov
	WHERE tipo_procedimento = 'Concurso publico' 
	AND anuncio_preco_base IS NOT NULL 
	AND anuncio_preco_base > 0 
	AND preco_contratual < anuncio_preco_base 
	AND preco_contratual > anuncio_preco_base * (1 - %s) 
	AND n_anuncio IN (
	SELECT n_anuncio
	FROM contratos_basegov
	GROUP BY n_anuncio
	HAVING COUNT(DISTINCT n_anuncio) = 1
	);
	''', (tol,))
	
	return(cur.fetchall())
	
\end{code}



\begin{figure}[h]
	\begin{minipage}{.33\textwidth}
		\centering
		\includegraphics[width=1\linewidth]{imagens/r31/pb.png}
		\caption{Distribuição do número de concursos públicos, cujo valor do preço base no Portal BASE é nulo, por divisão de CPV.}
		\label{fig:r31nulos}
	\end{minipage}
	\begin{minipage}{.33\textwidth}
		\centering
		\includegraphics[width=1\linewidth]{imagens/r31/pcpb.png}
		\caption{Distribuição do número de concursos públicos, cujo valor do preço contratual é superior ao preço base, por divisão de CPV.}
		\label{fig:r31nulos2}
	\end{minipage}
	\begin{minipage}{.33\textwidth}
		\centering
		\includegraphics[width=1\linewidth]{imagens/r31/pcpbnear.png}
		\caption{Distribuição do número de concursos públicos, cujo valor do preço contratual é muito próximo do preço base, por divisão de CPV.}
		\label{fig:r31nulo3}
	\end{minipage}
\end{figure}



\blankpage
\newpage


\backmatter
% turns off chapter numbering and doesn't fiddle with page numbering.



\clearemptydoublepage
\end{document}
