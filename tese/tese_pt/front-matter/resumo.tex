
\chapter*{Resumo}
\label{Chp:Resumo}
\addcontentsline{toc}{chapter}{Resumo}


A necessidade de adicionar mais transparência e fiabilidade na deteção de eventuais casos de fraude e corrupção, em processos de contratação pública, motivou o desenvolvimento de um mecanismo de monitorização e avaliação de contratos públicos. 

A presente tese desenvolve e implementa um sistema automatizado de aplicação de \textit{red flags}, ou indicadores, definidas pela Open Contracting Partnership (OCP), em conformidade com a legislação aplicada pelo Código dos Contratos Públicos (CCP), com o propósito de identificar eventuais irregularidades em ações de contratação pública.

Adicionalmente, a automatização do processo é feita através da utilização do serviço \textit{cloud} Amazon Web Services (AWS) de modo a garantir que a análise dos contratos seja efetuada diariamente, acompanhando a entrada de novos contratos numa base de dados PostgreSQL.

Da pesquisa efetuada, concluiu-se que o sistema é capaz de identificar e sinalizar contratos públicos que não considerem na íntegra os critérios definidos pelo CCP pelo que se assume como uma ferramenta eficaz na gestão e supervisão de contratos públicos.

\vspace{2cm}

\textbf{Palavras-chave:} concurso público, ajuste direto, código dos contratos públicos (CCP), red flags

