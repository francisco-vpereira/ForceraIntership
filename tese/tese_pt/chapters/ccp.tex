\section{Contratação Pública em Portugal}


A Contratação Pública em Portugal pode ser classificada de duas formas: aberta e fechada. As regras presentes no Código dos Contratos Públicos (CCP) dizem respeito aos contratos públicos celebrados entre uma entidade adjudicante pública e uma entidade adjudicatária.

%sendo esta composta por atos e formalidades relativos à formação, conclusão e produção de uma plena eficácia jurídica de um contrato público. A eficácia jurídica - ao contrário da eficácia social - é um conceito teórico, segundo o qual uma norma definida de acordo com a lei se torna eficaz em termos jurídicos. \\

O ato de adjudicar consiste em conferir o direito de algo a alguém, conceder algo ao maior licitante ou atribuir algo a alguém por concurso ou por ajuste. 
%Este é um termo essencial na área de contratação pública, sendo esta constituída pelas entidades adjudicantes e entidades adjudicatárias. 

O CCP é aplicado a entidades adjudicantes públicas, tais como o Estado, Regiões Autónomas, Autarquias locais, Institutos públicos, Entidades Administrativas Independentes, Banco de Portugal, Fundações Públicas, Associações Públicas, Associações de que façam parte uma ou várias pessoas coletivas referidas anteriormente e que sejam maioritariamente financiadas por estas. Além destas, são consideradas entidades adjudicantes organismos de direito público, pessoas coletivas e associações \footnote{nos termos do artigo 2.º n.º 2, alíneas a), b) e d)}. São consideradas, também, entidades adjudicantes organismos com atuação nos setores especiais da água, energia, tranposrtes e serviços postais \footnote{artigo 7.º n.º 1.º}. Existe, também, a possibilidade de aplicar o CCP a entidades não adjudicantes que pretendem celebrar determinados contratos de empreitadas de obras públicas ou de serviços associados a obras \footnote{artigo 275.º}.


% Associações de que façam parte uma ou várias pessoas coletivas referidas anteriormente, desde que sejam maioritariamente financiadas por estas, estejam sujeitas ao seu controle de gestão ou tenham um órgão de administração, de direção ou de fiscalização cuja maioria dos titulares seja, direta ou indiretamente, designada pelas mesmas

% São ainda entidades adjudicantes organismos de direito público, pessoas coletivas e associações, independentemente da sua natureza pública ou privada, nos termos do artigo 2.º n.º 2, alíneas a), b) e d).

% Para além das entidades adjudicantes referidas no artigo 2º, são também entidades adjudicantes as referidas no artigo 7.º n.º 1.º, concretamente as pessoas coletivas que realizam atividades nos seguintes sectores especiais da água, energia, transportes e serviços postais.

% O CCP aplica-se ainda a entidades que não sendo adjudicantes, se encontrem nas situações previstas no artigo 275.º, ou seja, entidades que pretendam celebrar determinados contratos de empreitadas de obras públicas ou de serviços associados a obras, desde que estes contratos sejam subsidiados diretamente em mais de 50\% do respetivo preço contratual por entidades adjudicantes, sempre que o preço contratual for igual ou superior aos limiares comunitários.


Existem duas fases principais no processo de contratação pública. 
A primeira fase é a \textbf{fase preparatória} em que é feita a decisão de realizar um contrato e inclui uma fase preparatória do procedimento e uma fase instrutória que terminará no ato de ajudicação. A segunda fase é a \textbf{fase conclusiva} em que é concluído e celebrado o contrato. Existe também uma \textbf{fase complementar} que pode ser necessária na eventualidade do contrato público depender de atos posterioes à sua celebração tais como a aprovação, visto e publicidade. \\

\subsection{Tipos de Procedimento de Contratação Pública}

Aquando da formação dos contratos, as entidades adjudicantes devem adotar um dos seguintes tipos de procedimentos : 

% \usepackage{multirow}
% \usepackage{colortbl}


\begin{table}[h!]
	\centering
	\begin{tabular}{|l|l|l|} 
		\hline
		\rowcolor[HTML]{B1DEFF}  \multicolumn{1}{|c|}{\textbf{Tipo de Procedimento~ ~}} & \multicolumn{1}{c|}{\textbf{Subtipo de Procedimento}} & \multicolumn{1}{c|}{\textbf{Artigo do CCP}}  \\ 
		\hline
		\multirow{2}{*}{\textbf{Ajuste Direto}}                                             & Regime Geral                                          & 112.º a 127.º                                \\ 
		\cline{2-3}
		& Regime Simplificado                                   & 128.º a 129.º                                \\ 
		\hline
		\multicolumn{2}{|l|}{\textbf{Consulta Prévia}}                                                                                              & 112.º a 127.º                                \\ 
		\hline
		\multirow{2}{*}{\textbf{Concurso Público}}                                          & Regime Geral                                          & 130.º a 154.º                                \\ 
		\cline{2-3}
		& Simplificado                                          & 155.º a 161.º                                \\ 
		\hline
		\multirow{2}{*}{\textbf{Concurso Limitado por Pŕevia Qualificação~ ~}}              & Regime Geral                                          &                                              \\ 
		\cline{2-3}
		& Simplificado                                          &                                              \\ 
		\hline
		\multicolumn{2}{|l|}{\textbf{Procedimento de Negociação}}                                                                                   & 193.º a 203.º                                \\ 
		\hline
		\multicolumn{2}{|l|}{\textbf{Diálogo Concorrencial }}                                                                                       & 204.º a 218.º                                \\ 
		\hline
		\multicolumn{2}{|l|}{\textbf{Parceria para a Inovação }}                                                                                    & 218.º-A a 218.º-D                            \\ 
		\hline
		\multicolumn{1}{l}{}                                                                & \multicolumn{1}{l}{}                                  & \multicolumn{1}{l}{}                         \\
		\multicolumn{1}{l}{}                                                                & \multicolumn{1}{l}{}                                  & \multicolumn{1}{l}{}                        
	\end{tabular}
	\caption{Tipos de Procedimentos na Contratação Pública}
	\label{}
\end{table}


% Existem diferentes tipos de procedimentos de contratação pública : \textit{ajuste direto - regime geral e simplificado-, consulta prévia, concurso público - normal e urgente, concurso limitado por prévia qualificação, procedimento de negociação, diálogo concorrencial, parceria para a inovação, disponibilização de bens móveis, serviços sociais e outros serviços específicos, concurso de conceção simplificado e concurso de ideias simplificado}.\\

\subsection{Tipos de Contratos}

A natureza e designação do tipo de contrato que é possível realizar para cada um dos procedimentos anteriormente enumerados é : 


\begin{table}[h!]
	\centering
	\begin{tabular}{|l|}
		\hline
		\rowcolor[HTML]{B1DEFF} 
		\multicolumn{1}{|c|}{\cellcolor[HTML]{B1DEFF}\textbf{Tipo de Contrato}} \\ \hline
		Empreitadas de obras públicas                                           \\ \hline
		Concessão de obras públicas                                             \\ \hline
		Concessão de serviços públicos                                          \\ \hline
		Locação ou aquisição de bens móveis                                     \\ \hline
		Aquisição de serviços                                                   \\ \hline
		Sociedade                                                               \\ \hline
	\end{tabular}
	\caption{Tipo de Contrato}
	\label{}
\end{table}



\begin{table}[h!]
	\centering
	\begin{tabular}{|l|l|l|l}
		\cline{1-3}
		\rowcolor[HTML]{B1DEFF} 
		\multicolumn{1}{|c|}{\cellcolor[HTML]{B1DEFF}\textbf{Preço Base}} & \multicolumn{1}{c|}{\cellcolor[HTML]{B1DEFF}\textbf{Objeto}} & \multicolumn{1}{c|}{\cellcolor[HTML]{B1DEFF}\textbf{Base Legal CCP}} & \cellcolor[HTML]{FFFFFF} \\ \cline{1-3}
		\rowcolor[HTML]{FFFFFF} 
		$\leq 10000$ €                                                     & Empreitada de obras públicas                                 & Artigo 128.º, nº1                                                    &                          \\ \cline{1-3}
		\rowcolor[HTML]{FFFFFF} 
		$\leq 50000$ €                                                     & Bens e Serviços                                              & Artigo 128.º, nº1                                                    &                          \\ \cline{1-3}
	\end{tabular}
	\caption{Ajuste Direto Regime Simplificado}
\end{table}


\begin{table}[h!]
	\centering
	\begin{tabular}{|l|l|l|l}
		\cline{1-3}
		\rowcolor[HTML]{B1DEFF} 
		\multicolumn{1}{|c|}{\cellcolor[HTML]{B1DEFF}\textbf{Preço Base}} & \multicolumn{1}{c|}{\cellcolor[HTML]{B1DEFF}\textbf{Objeto}} & \multicolumn{1}{c|}{\cellcolor[HTML]{B1DEFF}\textbf{Base Legal CCP}} & \cellcolor[HTML]{FFFFFF} \\ \cline{1-3}
		\rowcolor[HTML]{FFFFFF} 
		$< 30000 $ €                                                   & Empreitada de obras públicas                                 & Artigo 128.º, nº1                                                    &                          \\ \cline{1-3}
		\rowcolor[HTML]{FFFFFF} 
		$< 20000$ €                                                    & Bens e Serviços                                              & Artigo 128.º, nº1                                                    &                          \\ \cline{1-3}
		\cellcolor[HTML]{FFFFFF} $< 50000$ €                           & Outro tipo de contratos                                      &                                                                      &                          \\ \cline{1-3}
	\end{tabular}
	\caption{Ajuste Direto Regime Geral}
\end{table}