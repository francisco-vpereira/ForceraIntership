O conjunto de dados utilizado consiste  numa matriz com dimensão $m \times n$, sendo $m = 187198 $ o número de contratos celebrados e $n = 61$ o número de colunas. De entre o universo de contratos celebrados, podemos atentar na seguinte tabela como se distribuem consoante os diferentes tipos de procedimento : 

% Please add the following required packages to your document preamble:
% \usepackage[table,xcdraw]{xcolor}
% Beamer presentation requires \usepackage{colortbl} instead of \usepackage[table,xcdraw]{xcolor}
\begin{table}[H]
	\begin{tabular}{|
			>{\columncolor[HTML]{FFFFFF}}l |
			>{\columncolor[HTML]{FFFFFF}}l |}
		\hline
		{\color[HTML]{000000} Ajuste Direto Regime Geral}                                                     & {\color[HTML]{000000} 89882} \\ \hline
		{\color[HTML]{000000} Consulta Prévia}                                                                & {\color[HTML]{000000} 40331} \\ \hline
		{\color[HTML]{000000} Concurso público}                                                               & {\color[HTML]{000000} 28438} \\ \hline
		{\color[HTML]{000000} Ao abrigo de acordo-quadro (art.º 259.º)}                                       & {\color[HTML]{000000} 20624} \\ \hline
		{\color[HTML]{000000} Ao abrigo de acordo-quadro (art.º 258.º)}                                       & {\color[HTML]{000000} 5205}  \\ \hline
		{\color[HTML]{000000} Ajuste direto simplificado}                                                     & {\color[HTML]{000000} 1011}  \\ \hline
		{\color[HTML]{000000} Ajuste direto simplificado ao abrigo da Lei n.º 30/2021, de 21.05}              & {\color[HTML]{000000} 919}   \\ \hline
		{\color[HTML]{000000} Consulta Prévia Simplificada}                                                   & {\color[HTML]{000000} 435}   \\ \hline
		{\color[HTML]{000000} Concurso limitado por prévia qualificação}                                      & {\color[HTML]{000000} 297}   \\ \hline
		{\color[HTML]{000000} Procedimento de negociação}                                                     & {\color[HTML]{000000} 25}    \\ \hline
		{\color[HTML]{000000} Concurso público simplificado}                                                  & {\color[HTML]{000000} 24}    \\ \hline
		{\color[HTML]{000000} Consulta prévia ao abrigo do artigo 7º da Lei n.º 30/2021, de 21.05}            & {\color[HTML]{000000} 16}    \\ \hline
		{\color[HTML]{000000} Ajuste Direto Regime Geral ao abrigo do artigo 7º da Lei n.º 30/2021, de 21.05} & {\color[HTML]{000000} 9}     \\ \hline
		{\color[HTML]{000000} Serviços sociais e outros serviços específicos}                                 & {\color[HTML]{000000} 8}     \\ \hline
		{\color[HTML]{000000} Concurso de conceção simplificado}                                              & {\color[HTML]{000000} 2}     \\ \hline
		{\color[HTML]{000000} Setores especiais – isenção parte II}                                           & {\color[HTML]{000000} 1}     \\ \hline
		{\color[HTML]{000000} Concurso de ideias simplificado}                                                & {\color[HTML]{000000} 1}     \\ \hline
	\end{tabular}
	\caption{Número de contratos celebrados para os diferentes tipos de procedimento}
\end{table}

De todas as colunas, as que reveleram maior utilidade foram as colunas referentes ao ID do anúncio, número de anúncio, preço base, preço contratual, data de publicação de concurso, entidades concorrentes, CPV, \textbf{PREENCHER COM RESTANTES}. 



