\section{Portal Base-gov}


\section{Descrição da Base de Dados}



O conjunto de dados utilizado consiste  numa matriz com dimensão $m \times n$, sendo $m = 925627 $ o número de contratos celebrados e $n = 61$ o número de colunas. De entre o universo de contratos celebrados, podemos atentar na seguinte tabela como se distribuem consoante os diferentes tipos de procedimento : 

\begin{table}[H]
	\begin{tabular}{|
			>{\columncolor[HTML]{FFFFFF}}l |
			>{\columncolor[HTML]{FFFFFF}}l |}
		\hline
		{\color[HTML]{000000} Ajuste Direto Regime Geral}                                                     & {\color[HTML]{000000} 461250} \\ \hline
		{\color[HTML]{000000} Consulta Prévia}                                                                & {\color[HTML]{000000} 193801} \\ \hline
		{\color[HTML]{000000} Concurso público}                                                               & {\color[HTML]{000000} 110796} \\ \hline
		{\color[HTML]{000000} Ao abrigo de acordo-quadro (art.º 259.º)}                                       & {\color[HTML]{000000} 96281} \\ \hline
		{\color[HTML]{000000} Ao abrigo de acordo-quadro (art.º 258.º)}                                       & {\color[HTML]{000000} 21483}  \\ \hline
		{\color[HTML]{000000} Ajuste direto simplificado}                                                     & {\color[HTML]{000000} 38255}  \\ \hline
		{\color[HTML]{000000} Ajuste direto simplificado ao abrigo da Lei n.º 30/2021, de 21.05}              & {\color[HTML]{000000} 1135}   \\ \hline
		{\color[HTML]{000000} Consulta Prévia Simplificada}                                                   & {\color[HTML]{000000} 840}   \\ \hline
		{\color[HTML]{000000} Concurso limitado por prévia qualificação}                                      & {\color[HTML]{000000} 40}   \\ \hline
		{\color[HTML]{000000} Procedimento de negociação}                                                     & {\color[HTML]{000000} 37}    \\ \hline
		{\color[HTML]{000000} Concurso público simplificado}                                                  & {\color[HTML]{000000} 35}    \\ \hline
		{\color[HTML]{000000} Consulta prévia ao abrigo do artigo 7º da Lei n.º 30/2021, de 21.05}            & {\color[HTML]{000000} 23}    \\ \hline
		{\color[HTML]{000000} Ajuste Direto Regime Geral ao abrigo do artigo 7º da Lei n.º 30/2021, de 21.05} & {\color[HTML]{000000} 16}     \\ \hline
		{\color[HTML]{000000} Serviços sociais e outros serviços específicos}                                 & {\color[HTML]{000000} 9}     \\ \hline
		{\color[HTML]{000000} Concurso de conceção simplificado}                                              & {\color[HTML]{000000} 4}     \\ \hline
		{\color[HTML]{000000} Setores especiais – isenção parte II}                                           & {\color[HTML]{000000} 1}     \\ \hline
		{\color[HTML]{000000} Concurso de ideias simplificado}                                                & {\color[HTML]{000000} 1}     \\ \hline
	\end{tabular}
	\caption{Número de contratos celebrados para os diferentes tipos de procedimento}
\end{table}


Os contratos presentes na base de dados foram celebrados entre os anos de 2003 e 2024. Porém, a maioria dos contratos pertence ao périodo de 2019 a 2024. 

\begin{table}[H]
	\centering
	\begin{tabular}{|c|c|}
		\toprule
		Ano & Contagem \\
		\midrule
		2021 & 176172 \\  \hline
		2023 & 175044 \\ \hline
		2022 & 171497 \\ \hline
		2020 & 148924 \\ \hline
		2019 & 145337 \\ \hline
		2018 & 47317 \\ \hline
		NA   & 39390 \\ \hline
		2017 & 14942 \\ \hline
		2024 & 4494 \\ \hline
		2016 & 1709 \\ \hline
		2015 & 556 \\ \hline
		2013 & 194 \\ \hline
		2014 & 178 \\ \hline
		2012 & 55 \\ \hline
		2011 & 32 \\ \hline
		2010 & 10 \\ \hline
		2009 & 8 \\ \hline
		2008 & 3 \\ \hline
		2007 & 3 \\ \hline
		2006 & 1 \\ \hline
		2003 & 1 \\ 
		\bottomrule
	\end{tabular}
	\caption{Contagem de Contratos por Ano}
\end{table}



\section{Caracterização das Colunas}

De todas as colunas, as que reveleram maior utilidade foram as seguintes : 


	
\subsection{Identificador}

O \textbf{id} é o número que nos permite identificar um contrato específico no conjunto total de contratos. 

\subsection{Número de Anúncio}

O anúncio é o documento através do qual a entidade adjudicante informa ou comunica ao exterior (o mercado) o início do procedimento de contratação pública em causa. O anúncio contém um conjunto de informações relativas ao respetivo procedimento de contratação pública.(https://diariodarepublica.pt/dr/lexionario/termo/anuncio-contratacao-publica) 
Através do \textbf{número de anúncio} ??

\subsection{Preço Base}

O preço base é o montante máximo que a entidade adjudicante se dispõe a pagar pela execução de todas as prestações que constituirão o objeto do contrato. Funciona, assim, como um limite máximo de «aceitabilidade» do preço das propostas que sejam apresentadas: qualquer proposta que apresente um preço contratual superior ao preço base deve ser excluída. (https://diariodarepublica.pt/dr/lexionario/termo/preco-base)

\subsection{Tempo de Execução}

\subsection{Tipo de Anúncio}

\subsection{Tipo de Contrato}

\subsection{Objeto do contrato}

\subsection{Data de Publicação do Anúncio}

\subsection{Data de Celebração do Contrato}

\subsection{Preço Contratual}

\subsection{Prazo de Execução}

\subsection{Entidade Adjudicante}

\subsection{Fundamentação}

\subsection{Entidades Contratadas}

\subsection{Entidades Concorrentes}

\subsection{URL Anúncio}

\subsection{URL Peças do Procedimento}

\subsection{URL Documentos}

\subsection{CPV}

\subsection{Tipo de Contrato ??}

\subsection{Descrição}

\subsection{Preço Total Efetivo}

\subsection{Local}

	


