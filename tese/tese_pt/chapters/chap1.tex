%\section{Estruta da tese e   Etapas da Resolução do Problema}
%
%1. Analisar conjunto de dados. Ver nr de contratos, ver nr de procedimentos e nr de tipo de contratos, etc
%
%2. Categorizar as flags de acordo com a facilidade de implementação e valor
%
%3. Focar, numa fase inicial, num conjunto de contratos. Selecionaram-se apenas ajustes diretos e contratos públicos referentes a CPV's que dizem respeito a %serviços de consultoria de IT - 72

%4. Contruiu-se a primeira flag que compara preço base com preço contratual 
%
%5. Construi-se função que verifica se os precos contratuais dentro dos ajustes diretos caem dentro de um intervalo em torno do preco base
%
%6. Nesta mesma função, inclui-se a presença de um parametro de racio para comparar com precobase/precocontratual. Nos casos em que este quociente é mt alto vai ser verificado se ha ou nao presenca de vários lotes no contrato
%
%7. Construcao de uma funcao que vai analisar ajustes diretos. Priemiro verifica quais os ids dos contratos q ultrapassam o valor maximo permitido por lei q é 20k€. Se ultrapssar dispara a flag. Se num ajuste direto nao houver fundamentação é disparada a flag tambem
%
%8. Os ajustes diretos foram ordenados por ordem crescente de nt de celebracao. Comparou-se o nr de ajustes diretos celebrados com o nt de ajustes suspeitos, calculou-se o racio entre os 2. 
%
%9. No caso dos concursos publicos : qnd o preco base é mt maior que o preco contratual verificar se existem varios id's associados a um mesmo nr de anuncio, somar os preços base e contratual e verificar se o rácio ainda é muito grande

%10. Atribuiu-se um valor de flag contínuo (entre 0 e 1) ao contratos públicos onde é disparada um flag

%11. Contruíu-se uma função que dispara um flag caso o valor do prazo de candidatura de um concurso público seja inferior a 6 dias




\section{Estágio na FORCERA}

%A FORCERA é uma empresa criada em 2021, sediada na região Centro e, embora o seu método de colaboração seja maioritariamente o trabalho remoto, disponibiliza escritório em Lisboa para os seus trabalhadores. A FORCERA enquadra o estatuto de microempresa, contabilizando à data menos de 10 trabalhadores efetivos. Ciente da crescente necessidade dos órgãos públicos em obter maior eficiência, visibilidade e inteligência sobre os seus processos organizacionais, atua ao nível na inovação, sustentabilidade e tecnologia com foco na Administração Pública. Em sintonia com os decisores e respetivas equipas, são identificados os desafios, delineados os objetivos a fim de facilitar o cumprimento das metas estabelecidas. Em paralelo, tem como objetivo construir o expertise tecnológico para o desenvolvimento de novas soluções digitais que facilitem a implementação das estratégias definidas. 


O presente relatório resulta de um estágio curricular na empresa FORCERA que decorreu entre outubro de 2023 e abril de 2024. A FORCERA é uma empresa tecnológica com foco na inovação e sustentabilidade da Administração Pública. Percebendo a crescente necessidade dos órgãos públicos em obter maior eficiência, inteligência e sustentabilidade sobre os seus processos organizacionais e com base na engenharia de software, analítica avançada e inteligência artificial, a empresa dedica-se à produção de soluções inovadoras em diferentes áreas de atuação como finanças, smart cities, ou energia. 
Com escritórios e centro de atividades estabelecido em Lisboa, a FORCERA possui um \textit{track record} de projetos de desenvolvimento e investigação, em parceria com diversas entidades e organizações públicas da Europa Central.

\begin{figure}[H]
	\centering
	\includegraphics[width=0.7\textwidth]{imagens/forcera.png}
	\caption{Presença da FORCERA na Europa.}
	\label{fig:forcera}
\end{figure}


Dando primazia a iniciativas inovadoras e ambiciosas, a FORCERA distingue-se dentro do setor público através de projetos com clientes de referência como a Câmara Municipal de Lisboa ou com organizações de ação e impacto social como a Better Future. A FORCERA conta com um portfólio de clientes nacionais e internacionais recorrentes (notavelmente a PortX, uma startup FinTech sediada em Londres), e destaca-se por dar respostas a desafios emergentes em diferentes vertentes de sustentabilidade (ambiental, social, energética, financeira, entre outras). Como prova disso, destacam-se os projetos em consórcios europeus como o DigiPrime, que consistiu na criação de uma plataforma de circularidade para equipamentos digitais na Administração Pública, ou o SoTecIn Factory, onde foi desenvolvido o sistema DATA2FORK que permite às organizações públicas obter um scan completo de sustentabilidade ambiental referente a todos os serviços de catering de refeições em espaços públicos.

De salientar que, para lá do enorme expertise tecnológico baseado em várias décadas de experiência acumulada no desenvolvimento de soluções digitais que facilitam a implementação de estratégias organizacionais, a FORCERA aposta muito do seu valor acrescentado através do aspeto consultivo dos serviços prestados. A sua equipa multidisciplinar, composta por programadores, gestores de projeto, consultores, investigadores, apoia a expansão do know-how dos parceiros através da cocriação de estratégias digitais, capacitação, oportunidades de financiamento, ferramentas de apoio, networking e matchmaking e acompanhamento contínuo e incondicional ao longo das várias etapas do projeto.



\section{Fraude e Corrupção na Contratação Pública}

A rápida evolução dos computadores e a crescente facilidade de acesso à internet, observada sobretudo nas últimas três decadas, foi acompanhado por uma produção de dados sem precedentes, gerados por entidades, com destaque para as empresas tecnológicas e instituições governamentais e científicas. Com a transição da tecnologia analógica para a digital, pode afirmar-se que o paradigma relativo à coleta de dados mudou, dando-se início à era da \textit{big data}. Estima-se que 90\% dos dados a nível mundial foram produzidos nos últimos dois anos e é expectável que esta massa de informação tenda a crescer sustentada nas mudanças que se impuseram no estilo de vida dos países tecnologicamente mais avançados. A democratização do acesso a \textit{gadgets} de uso pessoal, serviços de \textit{cloud} e \textit{streaming}, fez disparar o volume de dados /informação em circulação.


Uma das áreas onde se tem vindo a constatar um incremento do volume de dados é na contratação pública aberta. O uso de \textit{big data} nesta área, uma prática ainda relativamente recente, implica a existência de bases de dados com dados estruturados de forma lógica e coerente, embora tal pressuposto nem sempre se confirme. Esta problemática radica no insuficiente investimento em setores emergentes, na complexidade e dificuldade em construir sistemas de armazenamento e transmissão de dados de alta qualidade, nos elevados custos associados e num sistema de uniformização de dados contratuais com algumas fragilidades. Deste ponto de vista, afigura-se incontornável a mudança de paradigma privilegiando-se a aposta/reforço numa lógica de colaboração multidisciplinar entre áreas de investigação distintas como o Direito, a Ciência Política, a Economia e a Informática\cite{inbook} \cite{ocp_brief}.


O recurso a dados, em quantidade e qualidade, em processos de contratação pública aberta é cada vez mais importante podendo revelar-se numa poderosa ferramenta na prestação de serviços de assessoria a órgãos judiciais e económicos, permitindo a deteção de comportamentos e padrões de conduta que indiciem práticas fraudulentas, muitas vezes antes de se consumar a celebração contratual\cite{redflags_guide}.



Os montantes em dinheiro envolvidos em processos de contratação pública assumem, não raras vezes, valores excepcionais. Estima-se que, a nível mundial, os governos gastam cerca 13 biliões de dólares americanos em contratos públicos para obras, bens e serviços. Deste valor, cerca de 10 biliões são gastos por 16 países, com destaque para a República Popular da China, com 4.2 biliões, seguido dos Estados Unidos da América, com 1.8 biliões. Porém, a informação disponibilizada e relativa à celebração destes contratos é escassa e, estima-se, que os contratos públicos disponibilizados totalizem um valor contratual de 362 mil milhões de dólares, o que corresponde a 2.8\% do valor total do mercado\cite{redflags_guide}\cite{govspent}\cite{stateogp}.



Empriricamente, está comprovado que tornar a contratação pública aberta fomenta o aumento da competitividade entre empresas no processo contratual, levando a uma valorização do dinheiro, a redução de custos estimada entre 10\% a 20\%, resultando numa mais eficiente alocação de fundos públicos, permite a participação de novos atores, aumenta a diversidade e inclusão, ao mesmo tempo que ajuda a reduzir a corrupção, atividades suspeitas e potenciais desvios de fundos comunitários\cite{govspent}\cite{stateogp}. Deste modo, a capacidade de supervisão, por parte de instituições governamentais e da sociedade civil, deve estar orientada para a prevenção de potenciais situações de corrupção e fraude, o que resulta na atribuição de mais qualidade à prestação de serviços. Tal depende do valor e qualidade dos dados contratuais disponíveis para consulta.



As evidências mostram que, apesar dos custos associados, os benefícios da contratação pública aberta superam os inconvenientes que deles possam derivar. O custo financeiro associado ao desenvolvimento de uma infraestrutura para transmissão e armazenamento de dados e à contratação de recursos humanos para o seu desenvolvimento, monitorização e manuntenção é elevado. Por outro lado, a redução de custos em dinheiro mal alocado superam, em muitos casos, o investimento inicial\cite{stateogp}.


A Organização para a Cooperação e Desenvolvimente Económico (OCDE) estima que, em média, o valor adjudicado na contratação pública ronda os 12\%-20\% do PIB de um país o que, além de representar uma fatia não negligenciável de recursos, representa a sua principal vulnerabilidade devido aos casos de fraude e corrupção\cite{ocp_brief}\cite{redflags_guide}. Desta forma, a disponibilização de dados contratuais torna acessível informação potencialmente sensível e pode gerar preocupações a nível de interesses privados e revelar possível conluio entre entidades, gerando, assim, alguma resistência por parte de algumas organizações internacionais e alguns governos em adotar este sistema de contratação pública aberto. Por sua vez, a disponibilização desta informação também pode fragilizar a confiança da sociedade civil nos órgãos governamentais, se se verificar a prática frequente de situações potencialmente fraudulentas e tendenciosas\cite{stateogp}.  A corrupção gera um impacto negativo em vários domínios da esfera social, desde o bem estar económico, social, político e cultural. Economicamente, afeta a produtividade, o lucro gerado e o crescimento económico. Politicamente, gera um sentimento de desconfiança e insatisfação perante os órgãos governamentais, o que leva a uma deterioração do sistema democrático e da sociedade civil. Este é um tema que, atualmente, se encontra mais presente, não só pelo aumento da cobertura mediática, mas também pelo crescente interesse da sociedade civil no tópico\cite{crime}, tendo em conta que no sistema tributário atual, uma grande porção do rendimento dos contribuintes é canalizado, através de impostos, para a celebração de contratos de prestação de bens e serviços públicos. 

Neste sentido, é importante aumentar a transparência na contratação pública. A Open Contracting Partnership (OCP) é uma organização responsável por estabelecer o canal de ligação entre governos, empresas, a sociedade civil e especialistas ao promover a transparência e a eficiência nos processos de contratação pública. Os objetivos da OCP traduzem-se na promoção\cite{redflags_guide}\cite{stateogp}:

\begin{my_itemize}
	
	\item do aumento da transparência e integridade do processo de contratação pública, tornando-o mais eficiente, justo e competitivo.
	
	\item da poupança de tempo e dinheiro para instituições governamentais.
	
	\item da ajuda no processo de deteção de situações anómalas, de fraude e corrupção.
	
	\item de uma maior confiança nos orgãos dirigentes ao ter conhecimento da utilização de fundos.
	
	\item do aumento da qualidade dos serviços e bens prestados à comunidade.
	
\end{my_itemize}


Para tal, esta organização incentiva a publicação de dados relativos a contratos públicos e desenvolveu a Open Contracting Data Standard (OCDS). A OCDS é um modelo que visa padronizar os dados relativos à contratação pública, incentivando-se a disponibilização de informação acerca de fornecedores, valores contratuais, termos e condições e progresso da execução, entre outros. Assim, os dados tornam-se acessíveis, compreensíveis e reutilizáveis por todos quanto aqueles que demonstram interesse, sejam cidadãos, jornalistas, organizações não-governamentais (ONG) e empresas. Ao conjunto de dados pradonizado podem ser aplicados um conjunto de indicadores de comportamento suspeito, também denominados de \textit{red flags}, desenvolvidos pela OCDS, ao longo de todas as etapas do processo contratual, zelando, de forma a que os trâmites procedimentais ocorram nos moldes corretos, garantindo, assim, uma gestão eficaz e responsável dos recursos financeiros \cite{redflags_guide}. As diferentes fases do processo contratual estão definidas na Tabela \ref{tab:fases}.


\begin{table}[ht]
	\centering
	\renewcommand{\arraystretch}{1.2}
	\setlength{\tabcolsep}{15pt}
	\resizebox{\textwidth}{!}{
	\begin{tabular}{ccccc}
		\hline
		\multicolumn{5}{c}{\cellcolor[HTML]{EFEFEF}\textbf{Fase}}     \\ \hline
		Planeamento & Proposta & Concessão & Contrato & Implementação \\ \hline
	\end{tabular}%
	}
	\caption{Etapas do processo de contratação pública.}
	\label{tab:fases}
\end{table}


Dada a natureza e vastidão da contratação pública, a aplicação de uma flag em função do contexto assume-se como prioridade da maior relevância. Por exemplo, o conceito de preço contratual \textit{muito} elevado depende de inúmeros fatores e, dessa forma, é necessário tê-los em consideração de forma a evitar \textit{false-flagging}. É crucial salientar que uma \textit{redflag} indicía que, num contrato, existe uma situação anómala/suspeita, podendo ser digno de uma investição mais pormenorizada. Assim, não prova, e não é suposto provar, qualquer tipo de comportamento ilícito. Estes comportamentos ilícitos, de natureza bastante complexa, pertencem a uma de três categorias: corrupção, fraude, e conluio. Por outro lado, detetar comportamentos suspeitos pode auxiliar na identificação de vulnerabilidades no sistema de contratação, levando a mudanças de políticas, e pode auxiliar na deteção de más práticas, ineficiências e erros de outra natureza, passíveis de serem corrigidos \cite{redflags_guide}. 


Neste sentido, o trabalho de estágio na FORCERA teve como objetivo principal a aplicação de indicadores desenvolvidos pela OCDS, a um conjunto de contratos públicos celebrados por organismos governamentais e publicados no \textit{website} Portal BASE, a fim de aferir da ocorrência de inconformidades. Desenvolvidos os indicadores, pretendeu-se criar um processo de automação de forma a que todos os contratos públicos adicionados ao Portal BASE fossem escrutinados com periodicidade diária.

Desde logo, impôs-se um trabalho de pesquisa relativo ao processo de contratação pública em Portugal, recorrendo ao Código dos Contratos Públicos (CCP), uma familiarização com os métodos e indicadores desenvolvidos pela OCP e OCDS bem como um trabalho de aprendizagem das ferramentas de gerenciamente de bases de dados PostgreSQL e do serviço de cloud Amazon Web Services (AWS). Na Figura \ref{fig:processo} é possível observar a organização do projeto elaborado ao longo deste estágio. 


\begin{figure}[H]
	\centering
	\includegraphics[width=0.85\textwidth]{imagens/processo_tese.png}
	\caption{Ilustração do conjunto de etapas envolvidas na elaboração do presente projeto. }
	\label{fig:processo}
\end{figure}

Ao longo deste texto, convencionou-se que para todos os números referentes a valores pecuniários dos ítens apresentados, o separador decimal será representado por vírgula e o separador dos milhares por ponto. Assim, o valor $20.315,42$ € corresponde a vinte mil, trezentos e quinze euros e quarenta e dois cêntimos. 

O desenvolvimento em software foi feito recorrendo à linguagem de programação Python (versão 3.11.8), ao sistema gerenciador de base de dados PostgreSQL (versão 8.1), à plataforma de controlo de versão Github, ao sistema de \textit{containerization} Docker (versão 25.0.3) e ao serviço de \textit{cloud} AWS. 




\section{Organização da tese}

Este relatório é composto por sete capítulos organizados da seguinte forma: 

\begin{my_itemize}


	\item \textbf{Capítulo 1 - Introdução}. Nesta capítulo é definido o problema a que se pretende dar resposta, identificada a necessidade de resolvê-lo  e apontados os procedimentos e os moldes em que se executou. Por último, é apresentada a organização do presente relatório. 
	
	
	\item \textbf{Capítulo 2 - Contratação Pública em Portugal}. Neste capítulo é feita uma apresentação do Código dos Contratos Públicos em Portugal, especificando os principais objetivos deste documento e definidos os procedimentos e constituintes da formação de um contrato público.

	
	\item \textbf{Capítulo 3 - Noções Matemáticas}. Neste capítulo são apresentadas ferramentas básicas utilizadas recorrentemente na análise de dados em Estatística. %É apresentada a noção geral de \textit{outlier} e uma alternativa de deteção para distribuições assimétricas.

	
	\item \textbf{Capítulo 4 - Base de Dados}. Este capítulo encontra-se dividido em duas partes. Na primeira é descrito o \textit{website} de onde são recolhidos os dados, o modo como se encontra estruturado e as entidades envolvidas na regulação do mesmo. Na segunda parte é descrita a base de dados, o processo de recolha de dados, as colunas que revelaram maior interesse ao longo do desenvolvimento do projeto e, por último, a organização dos dados em função das diferentes variáveis adotadas, tais como o tipo de procedimento, valor adjudicado, categoria do contrato e local de execução.


	\item \textbf{Capítulo 5 - Aplicação de \textit{Red Flags}}. No início deste capítulo apresenta-se o método de seleção de \textit{red flags}. De seguida, são caracterizadas as tabelas auxiliares da base de dados utilizadas ao longo do processo de construção das \textit{flags}. Por fim, é descrito o processo de construção de cada uma das flags desenvolvidas ao longo deste estágio e apresentados os resultados, após aplicação das mesmas. Por fim, é descrito o processo de automação recorrendo à ferramenta de \textit{containerization} Docker e ao serviço de \textit{cloud} AWS.

	
	\item \textbf{Capítulo 6 - Resultados}. Neste capítulo são sintetizados as principais conclusões obtidas, da aplicação das flags ao conjunto de dados.

	
	\item \textbf{Capítulo 7 - Conclusão}. Neste capítulo, são sintetizadas as conclusões extraídas após a construção e aplicação das \textit{red flags}, e apontadas possíveis melhorias e desenvolvimentos a considerar futuramente. 


\end{my_itemize}




