\section{Estruta da tese e   Etapas da Resolução do Problema}

1. Analisar conjunto de dados. Ver nr de contratos, ver nr de procedimentos e nr de tipo de contratos, etc

2. Categorizar as flags de acordo com a facilidade de implementação e valor

3. Focar, numa fase inicial, num conjunto de contratos. Selecionaram-se apenas ajustes diretos e contratos públicos referentes a CPV's que dizem respeito a serviços de IT

4. Contruiu-se a primeira flag que compara preço base com preço contratual 

5. Construi-se função que verifica se os precos contratuais dentro dos ajustes diretos estã



\section{Estágio}

INSERIR TEXTO FORCERA

\section{Fraude e Corrupção na Contratação Pública}

FALAR SOBRE O QUE É FEITO A NÍVEL EUROPEU PARA COMBATER FRAUDE E CORRUPÇÃO ( OCD, RED FLAGS, ETC)
\\
\\


\section{Contratação Pública em Portugal}



FALAR SOBRE COMO FUNCIONA A CONTRATAÇÃO PÚBLICA EM PORTUGAL
\\
\\


A Contratação Pública em Portugal pode ser classificada de duas formas : aberta e fechada. As regras presentes no CCP dizem respeito aos contratos públicos celebrados entre uma entidade adjudicante pública e uma entidade adjudicatária.

%sendo esta composta por atos e formalidades relativos à formação, conclusão e produção de uma plena eficácia jurídica de um contrato público. A eficácia jurídica - ao contrário da eficácia social - é um conceito teórico, segundo o qual uma norma definida de acordo com a lei se torna eficaz em termos jurídicos. \\


O ato de adjudicar consiste em conferir o direito de algo a alguém, conceder algo ao maior licitante ou atribuir algo a alguém por concurso ou por ajuste. 
%Este é um termo essencial na área de contratação pública, sendo esta constituída pelas entidades adjudicantes e entidades adjudicatárias. 

O CCP é aplicado a entidades adjudicantes públicas, tais como o Estado, Regiões Autónomas, Autarquias locais, Institutos públicos, Entidades Administrativas Independentes, Banco de Portugal, Fundações Públicas, Associações Públicas, Associações de que façam parte uma ou várias pessoas coletivas referidas anteriormente e que sejam maioritariamente financiadas por estas. Além destas, são consideradas entidades adjudicantes organismos de direito público, pessoas coletivas e associações \footnote{nos termos do artigo 2.º n.º 2, alíneas a), b) e d)}. São consideradas, também, entidades adjudicantes organismos com atuação nos setores especiais da água, energia, tranposrtes e serviços postais \footnote{artigo 7.º n.º 1.º}. Existe, também, a possibilidade de aplicar o CCP a entidades não adjudicantes que pretendem celebrar determinados contratos de empreitadas de obras públicas ou de serviços associados a obras \footnote{artigo 275.º}.



% Associações de que façam parte uma ou várias pessoas coletivas referidas anteriormente, desde que sejam maioritariamente financiadas por estas, estejam sujeitas ao seu controle de gestão ou tenham um órgão de administração, de direção ou de fiscalização cuja maioria dos titulares seja, direta ou indiretamente, designada pelas mesmas

% São ainda entidades adjudicantes organismos de direito público, pessoas coletivas e associações, independentemente da sua natureza pública ou privada, nos termos do artigo 2.º n.º 2, alíneas a), b) e d).

% Para além das entidades adjudicantes referidas no artigo 2º, são também entidades adjudicantes as referidas no artigo 7.º n.º 1.º, concretamente as pessoas coletivas que realizam atividades nos seguintes sectores especiais da água, energia, transportes e serviços postais.

% O CCP aplica-se ainda a entidades que não sendo adjudicantes, se encontrem nas situações previstas no artigo 275.º, ou seja, entidades que pretendam celebrar determinados contratos de empreitadas de obras públicas ou de serviços associados a obras, desde que estes contratos sejam subsidiados diretamente em mais de 50\% do respetivo preço contratual por entidades adjudicantes, sempre que o preço contratual for igual ou superior aos limiares comunitários.


Existem duas fases principais no processo de contratação pública. 
A primeira fase é a \textbf{fase preparatória} em que é feita a decisão de realizar um contrato e inclui uma fase preparatória do procedimento e uma fase instrutória que terminará no ato de ajudicação. A segunda fase é a \textbf{fase conclusiva} em que é concluído e celebrado o contrato. Existe também uma \textbf{fase complementar} que pode ser necessária na eventualidade do contrato público depender de atos posterioes à sua celebração tais como a aprovação, visto e publicidade. \\

\subsection{Tipos de Procedimento de Contratação Pública}

Aquando da formação dos contratos, as entidades adjudicantes devem adotar um dos seguintes tipos de procedimentos : 

\begin{enumerate}
	\setlength\itemsep{.01cm}
	
	\item Ajuste Direto
	\begin{enumerate}
		\setlength\itemsep{.01cm}
		\item Regime Geral
		\item Regime Geral ao abrigo do artigo 7º da Lei n.º 30/2021, de 21.05
		\item Simplificado
	\end{enumerate}
	
	\item Consulta Prévia
	\begin{enumerate}
		\setlength\itemsep{.01cm}
		\item ao abrigo do artigo 7º da Lei n.º 30/2021, de 21.05
	\end{enumerate}
	
	\item Concurso Público
	\begin{enumerate}
		\setlength\itemsep{.01cm}
		\item Regime Geral
		\item Simplificado
	\end{enumerate}
	
	\item Concurso limitado por prévia qualificação
	\begin{enumerate}
		\setlength\itemsep{.01cm}
		\item Regime Geral
		\item Simplificado
	\end{enumerate}
	
	\item Procedimento de negociação
	
	\item Diálogo concorrencial
	
	\item Parceria para a inovação
	
	\item Ao abrigo de acordo-quadro (art.º 258.º) 

	\item Ao abrigo de acordo-quadro (art.º 259.º)
		
	\item Disponibilização de bens móveis	
	
	\item Serviços sociais e outros serviços específicos
	
	\item Concurso de conceção simplificado
	
\end{enumerate}

% Existem diferentes tipos de procedimentos de contratação pública : \textit{ajuste direto - regime geral e simplificado-, consulta prévia, concurso público - normal e urgente, concurso limitado por prévia qualificação, procedimento de negociação, diálogo concorrencial, parceria para a inovação, disponibilização de bens móveis, serviços sociais e outros serviços específicos, concurso de conceção simplificado e concurso de ideias simplificado}.\\

A natureza e designação do tipo de contrato que é possível realizar para cada um dos procedimentos anteriormente enumerados é : 

\begin{enumerate}
	\setlength\itemsep{.01cm}
	\item Empreitadas de obras públicas
	\item Concessão de obras públicas
	\item Concessão de serviços públicos
	\item Locação ou aquisição de bens móveis
	\item Aquisição de serviços
	\item Sociedade
\end{enumerate}


\section{Objetivo do estágio}

FALAR DO OBJETIVO GERAL 
\\
\\




