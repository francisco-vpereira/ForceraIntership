\section{Estágio}

INSERIR TEXTO FORCERA

\section{Contratação Pública em Portugal}

A Contratação Pública em Portugal pode ser classificada de duas formas : aberta e fechada. De modo a garantir a eficácia jurídica, é fundamental conhecer o Código dos Contratos Públicos (CCP). As regras presentes no CCP dizem respeito aos contratos celebrados entre uma entidade adjudicante pública e uma entidade adjudicatária.

%sendo esta composta por atos e formalidades relativos à formação, conclusão e produção de uma plena eficácia jurídica de um contrato público. A eficácia jurídica - ao contrário da eficácia social - é um conceito teórico, segundo o qual uma norma definida de acordo com a lei se torna eficaz em termos jurídicos. \\


O ato de adjudicar consiste em conferir o direito de algo a alguém, entregar algo ao maior licitante ou atribuir algo a alguém por concurso ou por ajuste. 
%Este é um termo essencial na área de contratação pública, sendo esta constituída pelas entidades adjudicantes e entidades adjudicatárias. 
As entidades adjudicantes definidas no CCP são as seguintes : {Estado, Regiões Autónomas, Autarquias locais, Institutos públicos, Entidades Administrativas Independentes, Banco de Portugal, Fundações Públicas, Associações Públicas. ( COMPLETAR )\\


Existem regras que devem ser cumpridas ao longo de todas as fases do processo de contratação pública. A primeira fase é a \textbf{fase preparatória} em que é feita a decisão de realizar um contrato e inclui uma fase preparatória do procedimento e uma fase instrutória que terminará no ato de ajudicação. A segunda fase é a \textbf{fase conclusiva} em que é concluído e celebrado o contrato. Existe também uma \textbf{fase complementar} que pode ser necessária na eventualidade do contrato público depender de atos posterioes à sua celebração tais como a aprovação, visto e publicidade. \\

Existem diferentes tipos de procedimentos de contratação pública : \textit{ajuste direto - regime geral e simplificado-, consulta prévia, concurso público - normal e urgente, concurso limitado por prévia qualificação, procedimento de negociação, diálogo concorrencial, parceria para a inovação, disponibilização de bens móveis, serviços sociais e outros serviços específicos, concurso de conceção simplificado e concurso de ideias simplificado}.\\