%\section{Estruta da tese e   Etapas da Resolução do Problema}
%
%1. Analisar conjunto de dados. Ver nr de contratos, ver nr de procedimentos e nr de tipo de contratos, etc
%
%2. Categorizar as flags de acordo com a facilidade de implementação e valor
%
%3. Focar, numa fase inicial, num conjunto de contratos. Selecionaram-se apenas ajustes diretos e contratos públicos referentes a CPV's que dizem respeito a %serviços de consultoria de IT - 72

%4. Contruiu-se a primeira flag que compara preço base com preço contratual 
%
%5. Construi-se função que verifica se os precos contratuais dentro dos ajustes diretos caem dentro de um intervalo em torno do preco base
%
%6. Nesta mesma função, inclui-se a presença de um parametro de racio para comparar com precobase/precocontratual. Nos casos em que este quociente é mt alto vai ser verificado se ha ou nao presenca de vários lotes no contrato
%
%7. Construcao de uma funcao que vai analisar ajustes diretos. Priemiro verifica quais os ids dos contratos q ultrapassam o valor maximo permitido por lei q é 20k€. Se ultrapssar dispara a flag. Se num ajuste direto nao houver fundamentação é disparada a flag tambem
%
%8. Os ajustes diretos foram ordenados por ordem crescente de nt de celebracao. Comparou-se o nr de ajustes diretos celebrados com o nt de ajustes suspeitos, calculou-se o racio entre os 2. 
%
%9. No caso dos concursos publicos : qnd o preco base é mt maior que o preco contratual verificar se existem varios id's associados a um mesmo nr de anuncio, somar os preços base e contratual e verificar se o rácio ainda é muito grande

%10. Atribuiu-se um valor de flag contínuo (entre 0 e 1) ao contratos públicos onde é disparada um flag

%11. Contruíu-se uma função que dispara um flag caso o valor do prazo de candidatura de um concurso público seja inferior a 6 dias




\section{Estágio}

A FORCERA é uma empresa criada em 2021, sediada na região Centro e, embora o seu método de colaboração seja maioritariamente o trabalho remoto, disponibiliza escritório em Lisboa para os seus trabalhadores. A FORCERA enquadra o estatuto de microempresa, contabilizando à data menos de 10 trabalhadores efetivos. Ciente da crescente necessidade dos órgãos públicos em obter maior eficiência, visibilidade e inteligência sobre os seus processos organizacionais, atua ao nível na inovação, sustentabilidade e tecnologia com foco na Administração Pública. Em sintonia com os decisores e respetivas equipas, são identificados os desafios, delineados os objetivos a fim de facilitar o cumprimento das metas estabelecidas. Em paralelo, tem como objetivo construir o expertise tecnológico para o desenvolvimento de novas soluções digitais que facilitem a implementação das estratégias definidas. 

\section{Fraude e Corrupção na Contratação Pública}

FALAR SOBRE O QUE É FEITO A NÍVEL EUROPEU PARA COMBATER FRAUDE E CORRUPÇÃO ( OCD, RED FLAGS, ETC)
\\
\\



\section{Estrutura da tese}
FALAR DA ESTRUTURA DA TESE\\



