O objetivo principal deste projeto consistiu no desenvolvimento de um conjunto de indicadores, definidos pela OCDS, capazes de revelar práticas que sugerem inconformidades. O segundo objetivo consistiu na aplicação automatizada dos indicadores desenvolvidos, através do serviço de \textit{cloud} AWS, diariamente, a todos os contratos coletados do \textit{website} Portal BASE.\\

Desenvolveram-se, no total, 9 indicadores de suspeição, 8 deles a serem aplicados em concursos públicos e o remanescente em ajustes diretos em regime geral. O desenvolvimento, aplicação e análise do \textit{output} gerado por estes indicadores tendo em consideração duas variáveis de categorização de contratos, a saber: divisão do CPV e distrito de celebração do contrato. 

Concluíu-se que existem duas divisões de CPV com uma fração significativa, quer da totalidade de contratos celebrados, quer da totalidade de contratos sinalizados por cada flag, sendo estas referentes a contratos de aquisição de \textit{Equipamento médico, medicamentos e produtos para cuidados pessoais} - divisão 33 do CPV - e contratos de \textit{Trabalhos de construção} - divisão 45 do CPV. Existe uma predominância de contratos celebrados, independentemente da divisão de CPV, no distrito de Lisboa, seguindo-se o do Porto. Por serem estes os distritos com maior número de contratos celebrados, comprovou-se, também, que é nestes onde foram sinalizados o maior número de contratos, através das flags definidas, mantendo-se a mesma ordem hierárquica. \\

Com a crescente utilização do Portal BASE e, consequentemente, com um maior número de contratos disponibilizados, por ano, no \textit{website}, constatou-se, na generalidade, que o número de contratos sinalizados, por ano e por flag, acompanhou esta tendência.\\

Da totalidade de concursos públicos analisados, mais de metade não foram \textit{classificados} em inconformidade à luz dos indicadores propostos neste estudo, tal como não foi detetado nenhum contrato em que as 6 flags fossem sinalizadas em simultâneo. Verificou-se, também, que aumentando o número de flags em simultâneo, o número de contratos sinalizados diminuía rapidamente, sinalizando-se 39386 (30.3\%) contratos com uma flag ativada e apenas 21 ($ < 0.5\%$) com 5 flags ativadas.\\


Acrescente-se que, a flag R051 por ser a última a ser construída, requereria uma abordagem mais trabalhada. Por outro lado, o resultado da aplicação da flag nos contratos, revela informação relativa à identificação das entidades envolvidas, o que a torna especialmente sensível. 



Concluído este projeto, pode afirmar-se que o processo de automatização implementado resultou positivamente. Findo este processo, considera-se imporntate afinar os indicadores desenvolvidos e, futuramente, criar outros que complementem o processo de deteção de situações potencialmente anómalas.








\section{Limitações e Trabalho Futuro}

A disparidade dos valores de cada um das variáveis consideradas fundamentais, no mecanismo de identificação de inconformidades em contratação pública, torna o processo de construção de indicadores dependente do contexto/especificidade dos contratos. \\

A categorização por divisão e/ou grupo de CPV foi considerada a abordagem mais conveniente. Contudo, e porque dentro de cada divisão de CPV existe uma enorme variabilidade dentro dos parâmetros, tais como, preços contratuais, número de entidades concorrentes e de contratos celebrados, não será de excluir uma abordagem que contemple a categorização do CPV através da respetiva classe e categoria. Deste modo, seria possível desenvolver uma análise mais fina aos dados em questão. \\


Para a deteção de outliers pertencentes a distribuições assimétricas, o método da constante de Medcouple provou ser um bom ponto de partida, embora necessite de ser refinado em resultado da grande variabilidade dos valores das variáveis consideradas na construção dos indicadores R017 e R019.  \\


Num projeto futuro , poderia trabalhar-se a flag RF2 de modo a distinguir entre os contratos que sofreram alterações no preço contratual e os contratos cujo preço base, na plataforma Portal BASE, se encontra discriminado por unidade (metro quadrado, hora, dia, quilómetro). \\

Da mesma forma, para a flag R031, sugere-se uma modelação dos níveis de tolerância de acordo com a especificidade pretendida, por forma a obter maior nível de granularidade, ou seja, introduzindo mais dígitos do código CPV.




















